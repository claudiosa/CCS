
\documentclass{beamer}
\usepackage{beamerthemeshadow}
\usepackage{pstricks}              % PSTricks package
\usepackage{graphicx}
\usepackage{listings}
\usepackage{grffile}
\usepackage[utf8]{inputenc}
%\usepackage[portuges]{babel}

\usepackage{xcolor}

\definecolor{comment1}{HTML}{7e7e7e}
\definecolor{comment2}{HTML}{7e7e7e}
\definecolor{comment3}{HTML}{770000} % adicional à IDE do processing
\definecolor{function1}{HTML}{006699}
\definecolor{function2}{HTML}{006699}
\definecolor{function3}{HTML}{669900}
\definecolor{function4}{HTML}{006699}
\definecolor{invalid}{HTML}{7e7e7e}
\definecolor{keyword1}{HTML}{72a7c2}
\definecolor{keyword2}{HTML}{72a7c2}
\definecolor{keyword3}{HTML}{669900}%669900}
\definecolor{keyword4}{HTML}{ee6699}%ff6699}
\definecolor{keyword5}{HTML}{e27139}
\definecolor{labelcolor}{HTML}{77997e}
\definecolor{literal1}{HTML}{7D4793}
\definecolor{literal2}{HTML}{617952}
\definecolor{operator}{HTML}{006699}

\lstset{language=Java}
\lstset{frame=,
  framesep=5pt,
  basicstyle=\scriptsize,
  keywordstyle=[1]\color{keyword1},
  identifierstyle=\color{black},
  commentstyle=\normalfont\color{comment1},
  stringstyle=\color{purple}\ttfamily,
  columns=fullflexible,
  fontadjust=true,
}

\lstset{
morecomment=[l][\color{comment1}]{//},
morecomment=[s][\color{comment1}]{/*}{*/},
morecomment=[s][\color{comment3}\bf]{/*-}{*/},
morecomment=[s][\color{comment2}\bf]{/**}{*/}
}
\lstset{
emph={
    },emphstyle={[1]\color{literal1}},
emph={[2]HALF_PI,PI,QUARTER_PI,TAU,TWO_PI 
    },emphstyle={[2]\color{literal2}},
emph={[3]setup,draw},emphstyle={[3]\color{function1}\bf},
emph={[4]size,append,reverse,sort,beginContour,endContour,endShape,clear,blend,copy,filter,get,loadPixels,set,dist,lerp,mag,
    },emphstyle={[4]\color{function2}},
%emph={[5]},emphstyle={[5]\color{function3}},
emph={[6]mouseClicked,mouseDragged,mouseMoved,mouseReleased,mouseWheel,keyReleased,keyTyped,keyPressed,mousePressed,keyPressed,
    },emphstyle={[6]\color{function4}\bf},
emph={[7]break,case,continue,default
    },emphstyle={[7]\color{keyword1}\bf},
%emph={[8]},emphstyle={[8]\color{keyword2}},
emph={[9]for,while,else,if,switch,do
    },emphstyle={[9]\color{keyword3}},
emph={[10]frameRate,height,width,mouseButton,mouseX,mouseY,pmouseX,pmouseY,key,keyCode,%keyPressed,mousePressed
pixels},emphstyle={[10]\color{keyword4}},
emph={[11]boolean,byte,char,color,double,float,int,long,Array,FloatDict,FloatList,HashMap,IntDict,IntList,
JSONArray,JSONObject,String,StringDict,StringList,Table,TableRow,XML,PShape,
BufferedReader,PrintWriter,PImage,PGraphics,PShader,PFont,PVector
    },emphstyle={[11]\color{keyword5}}, 
emph={[12]P2D,P3D,PDF,HSB,RGB,CENTER,CORNER,ROUND,BEVEL,RADIUS,CORNERS,SQUARE,PROJECT,MITER,POINTS,LINES,TRIANGLES,TRIANGLE_STRIP,TRIANGLE_FAN,QUADS,QUAD_STRIP,CLOSE
      },emphstyle={[12]\color{labelcolor}},
emph={[13]cursor,noCursor,binary,boolean,byte,char,float,hex,int,str,unbinary,unhex,%frameRate
join,match,matchAll,nf,nfc,nfp,nfs,split,splitTokens,trim,arrayCopy,concat,expand,shorten,splice,
subset,createShape,loadShape,arc,ellipse,line,point,quad,rect,triangle,bezier,bezierDetail,bezierPoint,
bezierTangent,curve,curveDetail,curvePoint,curveTangent,curveTightness,box,sphere,sphereDetail,ellipseMode,
noSmooth,rectMode,smooth,strokeCap,strokeJoin,strokeWeight,beginShape,bezierVertex,curveVertex,quadraticVertex,
vertex,shape,shapeMode,,createInput,createReader,loadBytes,loadJSONArray,loadJSONObject,loadStrings,loadTable,
loadXML,open,parseXML,saveTable,selectFolder,selectInput,day,hour,millis,minute,month,second,year,print,println,
save,saveFrame,beginRaw,beginRecord,createOutput,createWriter,endRaw,endRecord,saveBytes,saveJSONArray,saveJSONObject,
saveStream,saveStrings,saveXML,selectOutput,applyMatrix,popMatrix,printMatrix,pushMatrix,resetMatrix,rotate,
rotateX,rotateY,rotateZ,scale,shearX,shearY,translate,ambientLight,directionalLight,lightFalloff,lights,
lightSpecular,noLights,normal,pointLight,spotLight,beginCamera,camera,endCamera,frustum,ortho,perspective,
printCamera,printProjection,modelX,modelY,modelZ,screenX,screenY,screenZ,ambient,emissive,shininess,specular,
background,colorMode,fill,noFill,noStroke,stroke,alpha,blue,brightness,color,green,hue,lerpColor,red,saturation,
createImage,image,imageMode,loadImage,noTint,requestImage,tint,texture,textureMode,textureWrap,,updatePixels,
blendMode,createGraphics,loadShader,resetShader,shader,createFont,loadFont,text,textFont,textAlign,textLeading,
textMode,textSize,textWidth,textAscent,textDescent,abs,ceil,constrain,exp,floor,log,map,max,min,norm,pow,round,
sq,sqrt,acos,asin,atan,atan2,cos,degrees,radians,sin,tan,noise,noiseDetail,noiseSeed,random,randomGaussian,randomSeed
    },emphstyle={[13]\color{function1}}    
}%

\newcommand*{\identifier}[1]{\lstinline!#1!}

\begin{document}

\title{Primeiros passos em Computação 1}  

\author{Luiz Ernesto Merkle}
\institute{Universidade Tecnol\'{o}gica Federal do Paran\'{a} \linebreak
         Programa de P\'{o}s-Graduação em Tecnologia \linebreak
	  Departamento Acadêmico de Informática \linebreak 
         Grupo de Pesquisa em Ci\^{e}ncias Humanas, Tecnologia e Sociedade \linebreak
         Estúdio Xuê}
\date{} 

\begin{frame}
\titlepage
\end{frame}

\begin{frame}[fragile]
\frametitle{0.01 - Uma linha depois de outra \dots }
  \begin{columns}[T]
    \begin{column}{.3\textwidth}
    \includegraphics[width=\textwidth]{./examples/Books/Processing Handbook2/Units 00-17/00 Using Processing/Ex_01/Ex_01.png}
    \end{column}
    \begin{column}{.7\textwidth}
      \lstinputlisting[language=java]{"./examples/Books/Processing Handbook2/Units 00-17/00 Using Processing/Ex_01/Ex_01.pde"}
    \end{column}
  \end{columns}
\end{frame}

\begin{frame}[fragile]
\frametitle{0.02 Para controlar o traço \dots}
  \begin{columns}[T]
    \begin{column}{.4\textwidth}
    \includegraphics[width=\textwidth]{./examples/Books/Processing Handbook2/Units 00-17/00 Using Processing/Ex_02/Ex_02.png}
    \end{column}
    \begin{column}{.6\textwidth}
      \lstinputlisting[language=java]{"./examples/Books/Processing Handbook2/Units 00-17/00 Using Processing/Ex_02/Ex_02.pde"}
    \end{column}
  \end{columns}
\end{frame}


\begin{frame}[fragile]
\frametitle{Exemplo passo a passo 0-03}
  \begin{columns}[T]
    \begin{column}{.4\textwidth}
    \includegraphics[width=\textwidth]{./examples/Books/Processing Handbook2/Units 00-17/00 Using Processing/Ex_03/Ex_03.png}
    \end{column}
    \begin{column}{.6\textwidth}
      \lstinputlisting[language=java]{"./examples/Books/Processing Handbook2/Units 00-17/00 Using Processing/Ex_03/Ex_03.pde"}
    \end{column}
  \end{columns}
\end{frame}


\begin{frame}[fragile]
\frametitle{0.03 Variáveis agilizam o controle \dots}
  \begin{columns}[T]
    \begin{column}{.4\textwidth}
    \includegraphics[width=\textwidth]{./examples/Books/Processing Handbook2/Units 00-17/00 Using Processing/Ex_04/Ex_04.png}
    \end{column}
    \begin{column}{.6\textwidth}
      \lstinputlisting[language=java]{"./examples/Books/Processing Handbook2/Units 00-17/00 Using Processing/Ex_04/Ex_04.pde"}
    \end{column}
  \end{columns}
\end{frame}


\begin{frame}[fragile]
\frametitle{0.04 E uma pitada de interação \dots}
  \begin{columns}[T]
    \begin{column}{.4\textwidth}
    \includegraphics[width=\textwidth]{./examples/Books/Processing Handbook2/Units 00-17/00 Using Processing/Ex_05/Ex_05.png}
    \end{column}
    \begin{column}{.6\textwidth}
      \lstinputlisting[language=java]{"./examples/Books/Processing Handbook2/Units 00-17/00 Using Processing/Ex_05/Ex_05.pde"}
    \end{column}
  \end{columns}
\end{frame}

\begin{frame}[fragile]
\frametitle{Enxuge tudo com funções \dots}
  \begin{columns}[T]
    \begin{column}{.4\textwidth}
    \includegraphics[width=\textwidth]{./examples/Books/Processing Handbook2/Units 00-17/00 Using Processing/Ex_06/Ex_06.png}
    \end{column}
    \begin{column}{.6\textwidth}
      \lstinputlisting[language=java]{"./examples/Books/Processing Handbook2/Units 00-17/00 Using Processing/Ex_06/Ex_06.pde"}
    \end{column}
  \end{columns}
\end{frame}

\begin{frame}[fragile]
\frametitle{0.07 Tempere com conjuntos de dados \dots}
  \begin{columns}[T]
    \begin{column}{.4\textwidth}
    \includegraphics[width=\textwidth]{./examples/Books/Processing Handbook2/Units 00-17/00 Using Processing/Ex_07/Ex_07.png}
    \end{column}
    \begin{column}{.6\textwidth}
      \lstinputlisting[language=java]{"./examples/Books/Processing Handbook2/Units 00-17/00 Using Processing/Ex_07/Ex_07.pde"}
    \end{column}
  \end{columns}
\end{frame}

\begin{frame}[fragile]
\frametitle{Apimente com classes de objetos  \dots}
  \begin{columns}[T]
    \begin{column}{.4\textwidth}
    \includegraphics[width=\textwidth]{./examples/Books/Processing Handbook2/Units 00-17/00 Using Processing/Ex_08/Ex_08.png}
    \end{column}
    \begin{column}{.6\textwidth}
      \lstinputlisting[language=java,firstline=1,lastline=15]{"./examples/Books/Processing Handbook2/Units 00-17/00 Using Processing/Ex_08/Ex_08.pde"}
   continua \dots
    \end{column}
  \end{columns}
\end{frame}

\begin{frame}[fragile]
\frametitle{Apimente com classes de objetos  \dots}
  \begin{columns}[T]
    \begin{column}{.4\textwidth}
    \includegraphics[width=\textwidth]{./examples/Books/Processing Handbook2/Units 00-17/00 Using Processing/Ex_08/Ex_08.png}
    \end{column}
    \begin{column}{.6\textwidth}
      \lstinputlisting[language=java,firstline=17]{"./examples/Books/Processing Handbook2/Units 00-17/00 Using Processing/Ex_08/Ex_08.pde"}
   \end{column}
  \end{columns}
\end{frame}

\begin{frame}[fragile]
\frametitle{Para abrir uma tela \dots}
\begin{lstlisting}{language=java}
// Two forward slashes are used to denote a comment.
// All text on the same line is a part of the comment.
// There must be no spaces between the slashes. For example,
// the code "/ /" is not a comment and will cause an error
// If you want to have a comment that is many
// lines long, you may prefer to use the syntax for a
// multiline comment

/* 
  A forward slash followed by an asterisk allows the 
  comment to continue until the opposite 
*/ 
 
// All letters and symbols that are not comments are translated 
// by the compiler. Because the following lines are not comments, 
// they are run and draw a display window of 200 x 200 pixels 
size(200, 200); 
background(102);   
\end{lstlisting}
\end{frame}


\begin{frame}[fragile]
\frametitle{Variáveis}
\begin{lstlisting}{language=java}
size(200, 200); // Runs the size() function

/** declara uma variável do tipo inteira */
int x; // Declares a new variable x

/** atribui o valor 102 à variável x */
x = 102; // Assigns the value 102 to the variable x

background(x); // Runs the background() function

/** calcula 2*x e atribui a cor de prenchimento das formas */
fill(2*x); 
rect(10,10, 180,180); 
\end{lstlisting}
\end{frame}

\begin{frame}[fragile]
\frametitle{Procure arrumar seu código de modo legível \dots}
\begin{lstlisting}{language=java}
/** Espaços não fazem diferença para o compilador */
		  size
// jose foi a cidade comprar mandioca
(   200,
// comentario fora de lugar


  200)      
  
  ;
background   (       
102)
           ;
/*- Mas fazem uma enorme diferença para quem o le */
\end{lstlisting}
\end{frame}

\begin{frame}[fragile]
\frametitle{Para imprimir mensagens \dots}
\begin{lstlisting}{language=java}
// To print text to the screen, place the desired output in quotes
println("Processing..."); // Prints "Processing..." to the console

// To print the value of a variable, rather than its name, don't put
// the name of the variable in quotes.
int x = 20;
println(x); // Prints "20" to the console

// While println() moves to the next line after the text
// is output, print() does not.
print("10");
println("20"); // Prints "1020" to the console
println("30"); // Prints "30" to the console

// The "+" operator can be used for combining multiple text
// elements into one line.
int x2 = 20;
int y2 = 80;
println(x2 + " : " + y2); // Prints "20 : 80" to the message window 
\end{lstlisting}
\end{frame}

\begin{frame}[fragile]
\frametitle{Variáveis de tipos básicos}
\begin{lstlisting}{language=java}
int x; 		// Declare the variable x of type int
float y; 	// Declare the variable y of type float
double yy; 	// Decalra uma variavel do tipo double 
boolean b; 	// Declare the variable b of type boolean

x = 50; 	// Assign the value 50 to x
y = 12.6; 	// Assign the value 12.6 to f
b = true;

//x = 12.6 	// Error - Não é possivel atribuir um float a um inteiro
x = int(12.6); //  Converte um float em um inteiro
\end{lstlisting}
\end{frame}

\begin{frame}[fragile]
\frametitle{Variáveis de tipos básicos}
\begin{lstlisting}{language=java}
int inteiro; 		// Declare the variable x of type int
long inteiro_maior; 

float real; 	// Declare the variable y of type float
double real_preciso; 	// Decalra uma variavel do tipo double 

boolean b; 	// Declare the variable b of type boolean
byte B; 

char letra; 
color vermelho; 

inteiro = 50; 	// Assign the value 50 to x
real = 12.6; 	// Assign the value 12.6 to f
b = true;
letra = 'c';

String  S = "Maria";

//inteiro = 12.6 	// Error - Não é possivel atribuir um float a um inteiro
inteiro = int(12.6); //  Converte um float em um inteiro
\end{lstlisting}
\end{frame}

\begin{frame}[fragile]
\frametitle{Variáveis de tipos básicos}
\scriptsize
\begin{itemize}
\item [boolean] Pode assumir o valor true ou o valor false
\item [char] 	Caractere em notação Unicode de $16$ bits. Serve para a armazenagem de dados alfanuméricos. Também pode ser usado como um dado inteiro com valores na faixa entre $0$ e $65535$.
\item [byte] 	Inteiro de 8 bits em notação de complemento de dois. Pode assumir valores entre $-2^7=-128$ e $2^{7}-1=127$.
\item [short] 	Inteiro de $16$ bits em notação de complemento de dois. Os valores possívels cobrem a faixa de $-2^{-15}=-32.768$ a $2^{15}-1=32.767$
\item [int]	Inteiro de $32$ bits em notação de complemento de dois. Pode assumir valores entre $-2^{31}=2.147.483.648$ e $231-1$=$2.147.483.647$.
\item [long]	Inteiro de $64$ bits em notação de complemento de dois. Pode assumir valores entre $-2^{63}$ e $2^{63}-1$.
\item [float]	Representa números em notação de ponto flutuante normalizada em precisão simples de $32$ bits em conformidade com a norma IEEE 754-1985. O menor valor positivo represntável por esse tipo é $1.40239846e-46$ e o maior é $3.40282347e+38$
\item [double] 	Representa números em notação de ponto flutuante normalizada em precisão dupla de $64$ bits em conformidade com a norma IEEE 754-1985. O menor valor positivo representável é $4.94065645841246544e-324$ e o maior é $1.7976931348623157e+308$
\end{itemize}
Sadao Massago e Waldeck Schützer (Sem data) Tipos de dados. In: Programação Java. Disponível em {http://www.dm.ufscar.br/~waldeck/curso/java/part22.html}
\end{frame}

\begin{frame}[fragile]
\frametitle{Variáveis de tipos básicos}
\begin{lstlisting}
    double numeroDecimal = 5.0;
    numeroDecimal = 5d;
    numeroDecimal = 0.5;
    numeroDecimal = 10f;
    numeroDecimal = 3.14159e0;
    numeroDecimal = 2.718281828459045D;
    numeroDecimal = 1.0e-6D;
    \\ http://en.wikibooks.org/wiki/Java_Programming/Literals
\end{lstlisting}
\end{frame}

\begin{frame}[fragile]
\frametitle{Operadores}
\scriptsize
$+$ (adição), $-$ (subtração), $*$ multiplicação, $/$ (divisão), $\%$ (modulo ou resto)

$()$ (parenteses)

$++$ (incremento), $--$ (decremento), $+=$ (adiciona e atribui), $-=$ (subtrai e atribui)

$*=$ (multiplica e atribui), $/=$ (divide e atribuin), $-$ (negação)

ceil(), floor(), round(), min(), max()

\normalsize

\begin{lstlisting}{language=java}
int a = 8;
int b = 10;
line(a, 0, a, height);
line(b, 0, b, height);
strokeWeight(4);
line(a*b, 0, a*b, height);
\end{lstlisting}
\end{frame}


\begin{frame}[fragile]
\frametitle{Operadores Lógicos e Relacionais}
\scriptsize
$\&\&$ (E lógico), $||$ (OU lógico), $!$ (Negação), $<$ (menor), $<=$ ( menor igual)

$>$ (maior), $>=$ (maior igual), $==$ (igual)

\normalsize

\begin{lstlisting}{language=java}
 	

for(int i=5; i<=95; i+=5) { 
  if((i > 35) && (i < 60)) { 
    stroke(0);    //Atribui a variável color a cor preta
  } else { 
    stroke(255);  //Atribui a variável color a cor branca
  } 
  line(30, i, 80, i); 
} 

\end{lstlisting}
\end{frame}

\begin{frame}
\frametitle{Revisão após recesso}
\begin{itemize}
 \item Instruções;
 \item Blocos; 
 \item Variáveis;
 \item Operadores aritméticos;
 \item Operadores realacionais;
 \item Condicionais;
 \item Laços e repetições;
 \item Laços e repetições aninhadas. 
\end{itemize}
\end{frame}

\begin{frame}[fragile]
\frametitle{Instruções}
Toda instrução termina por um \emph{ponto e vírgula}, inclusive a vazia. 
\begin{lstlisting}{language=java}
// isto é um esboço com seis instruções vazias 	
// É um esboço que faz nada seis vezes
;
;
;
;
;
;
\end{lstlisting}
\end{frame}


\begin{frame}[fragile]
\frametitle{Instruções}
Toda instrução termina por um \emph{ponto e vírgula}, inclusive a vazia. 
\begin{lstlisting}{language=java}
// isto é um esboço com seis instruções não vazias 	
int a; // declara uma variável a
a=50; // atribui o valor 5 à variável a
print(a); // imprime o valor de a
point(a,a); // chama a função que desenha um ponto em a,a 
\end{lstlisting}
\end{frame}


\begin{frame}[fragile]
\frametitle{Instruções}
Toda instrução termina por um \emph{ponto e vírgula}, inclusive a vazia. 
CUIDADO COM O PONTO E VIRGULA DEPOIS DO LAÇO FOR(){} e do condicional if(){}
\begin{lstlisting}{language=java}
// isto é um comando que repete 10 vezes uma instrução vazia:
for (int = 0; i<10; i++);
  xxx ; // instrução qualquer
\end{lstlisting}
Como espaços não fazem diferença, equivale a:
\begin{lstlisting}{language=java}
// isto é um comando que repete 10 vezes uma instrução vazia:
for (int = 0; i<10; i++)
  ;
  xxx ; // instrução qualquer
\end{lstlisting}
\end{frame}


\begin{frame}[fragile]
\frametitle{Instruções}
Toda instrução termina por um \emph{ponto e vírgula}, inclusive a vazia. 
{\scriptsize \color{red} CUIDADO COM O PONTO E VIRGULA DEPOIS DOs CONDICIONAIS!}
\begin{lstlisting}{language=java}
// Se verdadeiro, faz-se nada, pois o ponto e virgula é uma instrução vazia.
if  (true);
  xxx ; // instrução qualquer
\end{lstlisting}
Equivale à:
\begin{lstlisting}{language=java}
// Se verdadeiro, faz-se nada, pois o ponto e virgula é uma instrução vazia.
if  (true)
      ;
  xxx ; // instrução qualquer
\end{lstlisting}
O certo seria, sem o ponto e vírgula:
\begin{lstlisting}{language=java}
// Se verdadeiro, faz-se nada, pois o ponto e virgula é uma instrução vazia.
if  (true)
  xxx ; // instrução qualquer
\end{lstlisting}
\end{frame}

\begin{frame}[fragile]
\frametitle{Blocos}
Blocos são conjuntos de instruções cercados por chaves {}
\begin{lstlisting}{language=java}
// exemplo
{ 
println("mouseX =" + mouseX);
println("mouseY =" + mouseY);
}
\end{lstlisting}
\end{frame}

\begin{frame}[fragile]
\frametitle{Blocos}
Blocos são usados para separar conjuntos de instruções em condicionais, laços, funções, classes. 
\begin{lstlisting}{language=java}
if (true) { 
  println("verdadeiro");
  println("Até daqui a pouco");
  }
else 
  {
  println("falso");
  println("Hasta la vista, Baby");
  }
\end{lstlisting}
\end{frame}


\begin{frame}[fragile]
\frametitle{Blocos}
Blocos são usados para separar conjuntos de instruções em condicionais, laços, funções, classes, ou ao longo de um esboço, para separar variáveis
\begin{lstlisting}{language=java}

int a = 10; // esta variável pode ser acessada de qualquer lugar

a += 10;
{
  int a = 10; // esta variavel a só pode ser acessada de dentro destas chaves
  a *= 10;
}
a *= 10; 
// qual o valor de a, 100 ou 1000?
println(a);

\end{lstlisting}
\end{frame}


\begin{frame}[fragile]
\frametitle{Variáveis}
Variáveis armazenam informação de vários tipos na memória do computador.
Podem representar números inteiros(int), de ponto flutuante(float, double), caracteres(char), sequências de caracteres(String), cores (color), valores de ferdade (true, false).  
\begin{lstlisting}{language=java}
// Se declaradas fora de um bloco, podem ser acessadas de dentro dele, 
// desde que não haja outra variável de mesmo nome dentro dele. 
// Um contador global pode ser usado para isto.
int xy=0;

void seput(){ // inicialização
 strokeWeight(5)
 };
void draw(){ //repetição
 point(xy*10,xy*xy/100);
 if(xy==10)
   xy=0
   x++; // como a variável xy foi declarada fora do bloco, seu valor é atualizado.
      // Se decalra dentro do bloco, ela deixa de existir
 } 
\end{lstlisting}
\end{frame}

\begin{frame}[fragile]
\frametitle{Operadores aritiméticos}
Operadores aritiméticos são utilizados para calcular o valor de expressões aritméticas;
\begin{lstlisting}{language=java}
rect(width/4, height/4, width/2, height/2); 
\end{lstlisting}
Antes de desenhar o retângulo, o programa substitui as variáveis por seus valores
\begin{lstlisting}{language=java}
rect(100/4, 100/4, 100/2, 100/2); 
\end{lstlisting}
Calcula o valor de cada expressão:
\begin{lstlisting}{language=java}
rect(25,25,100,100);
\end{lstlisting}
E chama a função rect(), que desenha um retângulo na tela de saída.
\end{frame}

\begin{frame}[fragile]
\frametitle{Funções}
A linguagem processing tem associada a ela uma biblioteca extensa de funções, 
para os mais diversos fins. 

Se não encontrar, procure nas bibliotecas associadas, que pode 
have alguma já desenvolvida para o que você precisa. 

Se não encontrar, muitas são de código aberto ou livre, e podem ser modificadas, desde que atribuída a autoria de quem a desenvolveu. 
\end{frame}

\begin{frame}[fragile]
\frametitle{Funções}
Funções em processing podem receber ferenciados, desde que programados para tal:
\begin{lstlisting}{language=java}
// color cor;
cor = color(127);
cor = color(125,255);
cor = color(127,127,127);
cor = color(127,127,127,255);
\end{lstlisting}
Que neste caso, resultam na mesma cor e opacidade. 
\end{frame}


\begin{frame}[fragile]
\frametitle{Condicionais}
Um comando if()\{\}, ou if()\{\}else\{\} controla a execução de um programa, podendo-se escolher o que vai se executar quando uma condição verdadeira. 
\begin{lstlisting}{language=java}
void setup(){}
void draw(){
  if(mouseX<height-1)
    background(255,0,0);
  else
    background (0,255,0);
}
\end{lstlisting}
\end{frame}


\begin{frame}[fragile]
\frametitle{Condicionais Aninhados}
Um condicional pode conter outro condicional, sucessivamente. 
\begin{lstlisting}
void setup(){}
void draw(){
  if(mouseX<height/2)
    if(mouseY<height/2 0)
       background(0);
    else 
       background(255,0,0);
  else
   if(mouseY<height/2 0)
       background(0,255,0);
    else 
       background(0,0,255);
}
\end{lstlisting}
\end{frame}

\begin{frame}[fragile]
\frametitle{Laços e Repetições}
O computador é uma máquina excelente para repetir instruções.
Os comandos for()\{\} e while()\{\} podem ser usados para tal.
\begin{lstlisting}{language=java}
for(int a=0; a<100; a+=10)
  {
  fill(a,127);
  rect(a,a,10,10);
  } 
\end{lstlisting}
\end{frame}

\begin{frame}[fragile]
\frametitle{Laços e Repetições}
O computador é uma máquina excelente para repetir instruções.
Os comandos for()\{\} e while()\{\} podem ser usados para tal.

Como com valores pré-determinados.
\begin{lstlisting}{language=java}
for(int a=0; a<100; a+=10)
  {
  fill(a,127);
  rect(a,a,10,10);
  } 
\end{lstlisting}
\end{frame}

\begin{frame}[fragile]
\frametitle{Laços e Repetições}
O computador é uma máquina excelente para repetir instruções.
Os comandos for()\{\} e while()\{\} podem ser usados para tal.

Como com valores pseudo-randômicos.
\begin{lstlisting}{language=java}
for(int a=0; a<10000; a+=10)
  {
  fill(random(255),random(255),random(255));
  rect(random(100), random(100),random(10,20),random(5,30));
  } 
\end{lstlisting}
\end{frame}

\begin{frame}[fragile]
\frametitle{Laços e Repetições Aninhadas}
Quando se precisa repetir uma repetição,
usam-se laços aninhados

Como com valores pseudo-randômicos.
\begin{lstlisting}{language=java}
for(int y=0; y<100; y+=10)
  {
  for(int x=0; x<100; x+=15)
    {
    fill(random(255),random(255),random(255));
    rect(x,y, random(5,10),random(5,15));
    }
  }
\end{lstlisting}
\end{frame}

\begin{frame}[fragile]
\frametitle{Seleção Múltipla}
Quando se têm multiplas escolhas a fazer.
Exemplo 1:

\begin{lstlisting}{language=java}
int num = 1;

switch(num) {
  case 0: 
    println("Zero");  // Does not execute
    break;
  case 1: 
    println("One");  // Prints "One"
    break;
  default:
}


\end{lstlisting}
\end{frame}

\begin{frame}[fragile]
\frametitle{Seleção Múltipla}
Quando se têm multiplas escolhas a fazer.
Exemplo 2:
\begin{lstlisting}{language=java}
char letter = 'N';

switch(letter) {
  case 'A':
  case 'a':
    println("Alpha");  // Does not execute
    break;
  case 'B':
  case 'b':
    println("Bravo");  // Does not execute
    break;
  default:             // Default executes if the case labels
    println("None");   // don't match the switch parameter
    break;
}

\end{lstlisting}
\end{frame}

\begin{frame}[fragile]
\frametitle{Seleção Múltipla}
Quando se têm multiplas escolhas a fazer.
Exemplo 3a:
\begin{lstlisting}{language=java}
void setup() {
  size(200, 200);
}

char letter='a';
void draw() {

  switch(letter) {
  case 'R':
  case 'r':
    background(255, 0, 0); 
    break;
  case 'G':
  case 'g':  // 
    background(0, 255, 0); 
    break;
  case 'B':
  case 'b':
    background(0, 0, 255, 0); 
    break;
  }
}
\end{lstlisting}
\end{frame}

\begin{frame}[fragile]
\frametitle{Seleção Múltipla}
Quando se têm multiplas escolhas a fazer.
Exemplo 3a (continuação):
\begin{lstlisting}{language=java}
void keyPressed()
{
  letter = key;
  switch(letter) {
  case 'R':
  case 'r':
    println("Encarnado");
    break;
  case 'G':
  case 'g': 
    println("Verde");  
    break;
  case 'B':
  case 'b':
    println("Azul"); 
    break;
  }
}

\end{lstlisting}
\end{frame}



\begin{frame}[fragile]
\frametitle{Vetores}
Vetores (Parte 1 de 2)
\begin{lstlisting}{language=java}
int pingos=150; 
int [] chuvax;
int [] chuvay;

void setup() {
  size(200, 200);
  chuvax = new int[pingos];
  chuvay = new int[pingos];
  for (int i=0; i<pingos; i++)
  {
    chuvax[i] = floor(random(width));
    chuvay[i] = floor(random(height));
  }
  smooth();
}
\end{lstlisting}
\end{frame}

\begin{frame}[fragile]
\frametitle{Vetores}
Vetores (Parte 2 de 2)
\begin{lstlisting}{language=java}
void draw() {
  stroke(150);
  for (int i=0; i<pingos; i++)
    line(chuvax[i], chuvay[i],chuvax[i]+3, chuvay[i]+10 );
}

void keyPressed()
{
  background(200);
  for (int i=0; i<pingos; i++)
  {
    chuvax[i] = floor(random(width));
    chuvay[i] = floor(random(height));
  }
}
\end{lstlisting}
\end{frame}

\begin{frame}[fragile]
\frametitle{Vetores}
Vetores (Parte 1 de 2)
\begin{lstlisting}{language=java}
void setup() {
  size(200, 200);
  }
  void draw() {
     fachada();
  }

  void fachada (int x, int y, int largura, int altura)
  {
  rect(x,y+altura,largura, altura);
  triangle(x,y,x+largura,y, x+largura/2, y-altura/2);
  }
  
\end{lstlisting}
\end{frame}

\begin{frame}[fragile]
\frametitle{Vetores}
Vetores (Parte 2 de 2)
\begin{lstlisting}{language=java}

void keyPressed()
{
 
}
\end{lstlisting}
\end{frame}





\begin{frame}[fragile]
\frametitle{Imagens, Formas, Texto, etc.}
\begin{lstlisting}{language=java}
PShape 
PImage
PFont
PGraphics
PVector // só em java mode
\end{lstlisting}
\end{frame}


\begin{frame}[fragile]
\frametitle{Dados compostos}
\begin{lstlisting}{language=java}
Array
String
ArrayList
*Dict 
\end{lstlisting}
\end{frame}



\begin{frame}[fragile]
\frametitle{Bibliotecas}
Recursos pré-programados de funcionalidade interessante.
\begin{lstlisting}{language=java}
Video
DXF 
PDF export
Audio
Rede
\end{lstlisting}
\end{frame}







% \begin{frame}[fragile]
% \frametitle{Laços Aninhados}
% T
% \begin{lstlisting}{language=java}
% // exemplo;
% \end{lstlisting}
% \end{frame}




% 
% \begin{frame}[fragile]
% \frametitle{Estrutura de um esboço\dots}
% \begin{lstlisting}{language=java}
%  
% \end{lstlisting}
% \end{frame}
% 
% 
% \begin{frame}[fragile]
% \frametitle{Estrutura de um esboço\dots}
% \begin{lstlisting}{language=java}
%  
% \end{lstlisting}
% \end{frame}
% 

%  \begin{frame}[fragile]
% \begin{lstlisting}{language=java}
% // The next line is needed if running in JavaScript Mode with Processing.js
% /* @pjs preload="moonwalk.jpg"; */ 
% 
% PImage img;
% int smallPoint, largePoint;
% 
% void setup() {
%   size(640, 360);
%   img = loadImage("moonwalk.jpg");
%   smallPoint = 4;
%   largePoint = 40;
%   imageMode(CENTER);
%   noStroke();
%   background(255);
% }
% 
% void draw() { 
%   float pointillize = map(mouseX, 0, width, smallPoint, largePoint);
%   int x = int(random(img.width));
%   int y = int(random(img.height));
%   color pix = img.get(x, y);
%   fill(pix, 128);
%   ellipse(x, y, pointillize, pointillize);
% }
%   \end{lstlisting}
% \end{frame}
% 
% 
% 
% 
% \begin{frame}[fragile]
% \begin{lstlisting}{language=java}
% //keyword5
% boolean byte char color double float int long Array FloatDict FloatList HashMap 
% IntDict IntList JSONArray JSONObject  String StringDict StringList Table TableRow 
% XML PShape BufferedReader PrintWriter PImage PGraphics PShader PFont PVector 
% //label 
% P2D P3D  PDF HSB RGB CENTER CORNER ROUND BEVEL RADIUS 
% CORNERS ROUND SQUARE PROJECT MITER POINTS LINES 
% TRIANGLES TRIANGLE_STRIP TRIANGLE_FAN QUADS QUAD_STRIP CLOSE 
% //function2
% size() append() reverse() sort() beginContour() endContour() endShape() 
% clear() blend() copy() filter() get() loadPixels()  set()  dist() lerp() mag()
% //function3  none
% //function4
% mouseClicked() mouseDragged() mouseMoved()  mouseReleased() mouseWheel() 
% mousePressed() keyPressed() keyReleased() keyTyped() 
%   \end{lstlisting}
% \end{frame}
% 
% \begin{frame}[fragile]
% \begin{lstlisting}{language=java}
% // function1 
% cursor()  frameRate()  noCursor() binary() boolean() byte() char() float() hex() int() str() 
% unbinary() unhex() join() match() matchAll() nf() nfc() nfp() nfs() split() splitTokens() 
% trim() arrayCopy() concat() expand() shorten() splice() subset() createShape() loadShape() 
% arc() ellipse() line() point() quad() rect() triangle()bezier() bezierDetail() bezierPoint() 
% bezierTangent() curve() curveDetail() curvePoint() curveTangent() curveTightness()box() 
% sphere() sphereDetail()ellipseMode() noSmooth() rectMode() smooth() strokeCap() 
% strokeJoin() strokeWeight() beginShape() bezierVertex() curveVertex() quadraticVertex() 
% vertex() shape() shapeMode() createInput() createReader() loadBytes() loadJSONArray() 
% loadJSONObject() loadStrings() loadTable() loadXML() open() parseXML() saveTable() 
% selectFolder() selectInput() day() hour() millis() minute() month() second() year() print() 
% println() save() saveFrame() beginRaw() beginRecord() createOutput() createWriter()
% endRaw()endRecord() saveBytes() saveJSONArray() saveJSONObject() saveStream() 
% saveStrings() saveXML() selectOutput() applyMatrix() popMatrix() printMatrix() 
% pushMatrix() resetMatrix() rotate() rotateX() rotateY() rotateZ() scale() shearX() shearY()
% translate() ambientLight() directionalLight() lightFalloff() lights() lightSpecular() noLights() 
% normal() pointLight() spotLight() beginCamera() camera() endCamera() frustum() ortho() 
% perspective() printCamera() printProjection() modelX() modelY() modelZ() screenX()
% screenY() screenZ() ambient() emissive() shininess() specular()background() colorMode() 
% fill()noFill() noStroke() stroke() alpha() blue() brightness() color() green() hue() lerpColor() 
% red() saturation() createImage()image() imageMode() loadImage() noTint() requestImage() 
% tint()texture() textureMode() textureWrap() updatePixels() blendMode() createGraphics() 
% loadShader() resetShader() shader() createFont() loadFont() text() textFont()textAlign() 
% textLeading() textMode() textSize() textWidth() textAscent() textDescent() abs() ceil() 
% constrain()  exp() floor() log()map() max() min() norm() pow() round() sq() sqrt() acos() 
% asin() atan() atan2() cos() degrees() radians() sin() tan()
% noise() noiseDetail() noiseSeed() random() randomGaussian() randomSeed()
%   \end{lstlisting}
% \end{frame}
% 
% \begin{frame}[fragile]{2. Discussion of Hello world}
%   Note that \lstinline!cout << "Hello"! in the previous example means
%   ``console out''.
% \end{frame}
% 
% \begin{frame}{3. Discussion of Hello world}
%   Note that \identifier{cout << "Hello"} in the previous example means
%   ``console out''.
% \end{frame}
\end{document}