
\section{O Curso}
\begin{frame}


\frametitle{Disciplina}

\begin{block}{Sistemas Multiagentes -- OSIM001}

\begin{itemize}
\item \emph{\textbf{Turma:}} 
\item \emph{\textbf{Professor:}} Claudio Cesar de Sá
  \begin{itemize}
  \item \texttt{claudio.sa@udesc.br}
  \item Sala 13 Bloco F
  \end{itemize}
\item \emph{\textbf{Carga horária:}} 72 horas-aula 
\textcolor{red}{$\bullet$}~Teóricas: 36 \textcolor{red}{$\bullet$}~Práticas: 36
\item \emph{\textbf{Curso:}} BCC
\item \emph{\textbf{Requisitos:}} Vários -- IA, LMA, TEC, SO, PRP, ...
\item \emph{\textbf{Período:}} 1º semestre de 2017
\item \emph{\textbf{Horários:}}
  \begin{itemize}
  \item 6ª 10h10 (2 aulas) - F-104  -- aula expositiva
  \item 6ª 18h00 (2 aulas) - F-306 -- lab
  
  \end{itemize}

\end{itemize}

\end{block}

\end{frame}


%-------------------------------------
\begin{frame}
\frametitle{Ementa}

\begin{block}{Ementa}
 Motivação do paradigma. Agentes reativos e cognitivos. Teoria e arquitetura de agentes. Sistema multiagentes (SMA) reativo e cognitivo. Linguagens e protocolos de comunicação. Coordenação e negociação. Metodologias para desenvolvimento de SMAs. Ambientes de desenvolvimento.

\end{block}

\end{frame}

%-------------------------------------
\begin{frame} [allowframebreaks=0.9]
\frametitle{Objetivos}

\begin{itemize}
\item \emph{\textbf{Geral:}} 
 Apresentar o  conceito de inteligência artificial distribuída: 
 desenvolvimento de agentes e abordagens para coordenação de sistemas multiagentes,
 permitindo ao aluno ser capaz de modelar problemas de forma a modularizar sua solução
 de forma distribuída.

\newpage
\item \emph{\textbf{Específicos:}} 

  \begin{itemize}
  \item Descrever o histórico e quadro atual da Inteligência Artificial -- Moderna. 
  \item Compreender a noção de Teoria de Problemas, computabilidade e complexidade na ótica de IA e IAD.
  \item Diferencia IAD (orientação a divisão de problemas) versus SMA (orientação a 
  coordenação de agentes)
  \item  Conhecer diferentes arquiteturas de agentes
  \item   Modelar problemas computacionais através de aplicação de agentes.
  \item   Descrever o processo de tomada de decisão e aprendizagem computacional baseado em 
  sistemas multiagentes.
  \item   Conceber, projetar e construir sistemas computacionais capazes de aplicar sistemas 
  multiagentes como técnica de resolução.
  
  \end{itemize}

\end{itemize}

\end{frame}



%-------------------------------------
\begin{frame}
\frametitle{Conteúdo programático}


    \begin{itemize}
      \item Conceitos de SMA (há muitos correlacionados há áreas diversas)
      \item Ferramentas: Netlogo e Picat
      \item Aplicação: voces escolhem
      \item Um artigo $ \equiv $ projeto
      \item Um artigo OUTRO da área a ser apresentado: ficha técnica
      
    \end{itemize}
\end{frame}


%-------------------------------------

\subsection{Ferramentas}
\begin{frame}

    \frametitle{Ferramentas}

    \begin{itemize}
      \item PICAT (com suporte)
      \item NETLOGO
       \url{http://ccl.northwestern.edu/netlogo/docs/} (escondido in WEB)
      
    \end{itemize}
\end{frame}

%%%%%%%%%%%%%%%%%%%%%%%%%%%%%%%%%%%%%%%%%%%%%%%%%%%%%%%%%%%%%%%%%%%%%

\subsection{Metodologia e avaliação}  

\begin{frame}[allowframebreaks=0.9]

\frametitle{Metodologia e avaliação}

\textbf{Metodologia:} \\

\textit{As aulas serão expositivas e práticas. A cada novo assunto tratado, exemplos 
 são demonstrados utilizando ferramentas computacionais adequadas para consolidar os conceitos 
 tratados. 
  As aulas nas sextas-feiras a tarde poderão ser realizadas, também, na forma de estudo dirigido.}


\newpage
    \textbf{Avaliação}

    \begin{itemize}
      \item Duas provas (conceituais) -- $\approx$  25\%\\
      
	\quad \textcolor{red}{$\bullet$}~$P_1$: 25/mar\\
	\quad \textcolor{red}{$\bullet$}~$P_2$: 25/maio (provão: todo conteúdo)


      \item Exercícios de laboratório  -- $\approx$ 10\%
      \item Implementação de um protótipo  -- $\approx$ 20\%
       \item O artigo (resultados da implementação)  -- $\approx$ 30\%

      \item Para o artigo: muito material será fornecido em \LaTeX ...

      \item Apresentação de um artigo estudado sobre SMA -- $\approx$ 15\%
 
      \item Presença e participação
      
      \item Média para aprovação: 5,0 (cinco)
      
    \end{itemize}

\end{frame}



\subsection{Dinâmica}
\begin{frame}

    \frametitle{Dinâmica de Aula}

    \begin{itemize}
      
      \item Teoria na parte da manhã -- 10:00 hrs -- F-104
      \item \textit{Ralação} a tarde -- LAB -- estudar o NetLogo -- vídeo-aulas
      
    \end{itemize}
\end{frame}



%%%%%%%%%%%%%%%%%%%%%%%%%%%%%%%%%%%%%%%%%%%%%%%%%%%%%%%%%%%%%%%%%%%%%

\subsection{Referências}  
%[allowframebreaks=0.9]

%-------------------------------------
\begin{frame}[allowframebreaks=0.9]
\frametitle{Bibliografia}  

\textbf{Básica:} 
\begin{itemize}

\item \scriptsize{ALVARES, L. O., SICHMAN, J. \textit{Introdução aos Sistemas
 Multiagentes}, Anais do EINE – Escola de Informática do Nordeste,
  Sociedade Brasileira de Computação – SBC, Brasil, 1997.}

\item \scriptsize{FERBER, J. \textit{Multi-Agent Systems: An Introduction 
to Distributed Artificial Inteligence}. Harlow, England, Addison-Wesley, 1999.}


\item \scriptsize{WOOLDRIDGE, M.. \textit{An introduction to
 MultiAgent Systems}, John Wiley, 2001}


\item \url{https://github.com/claudiosa/CCS/tree/master/https://github.com/claudiosa/CCS/tree/master/sistemas_multiagentes}

\end{itemize}

\textbf{Complementar:}

\begin{itemize}
\item \scriptsize{Nikos Vlassis,
\textit{A Concise Introduction to Multiagent Systems and Distributed Artificial Intelligence}
Synthesis Lectures on Artificial Intelligence and Machine Learning
2007, 71 pages -- guia dos tópicos destes slides
}


\item \scriptsize{
O’HARE, G.; JENNINGS, N. (Editors) \textit{Foundations of distributed artificial intelligence}, New York, NY: John Wiley, 1996.}


\item \scriptsize{
WEISS, G. \textit{Multiagent Systems: A Modern Approach to Distributed Artificial Intelligence}. London, MIT Press, 2001.
}




\item \scriptsize{Russell, S., Norvig, Peter; "Inteligência Artificial", Ed. Campus-Elsevier; Brasil, 2010  -- em inglês.}



\item \scriptsize{Bittencourt, G.; “Inteligência Artificial, ferramentas e teorias”; 3. ed. UFSC; Florianópolis, SC; 2006.}

\item \scriptsize{Barreto. J.M.; “Inteligência Artificial, uma abordagem híbrida”; 3a. ed.; RoRoRo; Florianópolis, SC; 2001}

\item \scriptsize{Eberhart, R; Simpson, P.; Dobbins, R.; "Computational Intelligence PC Tools"; AP Professional; 1996; ISBN 0-12-228630-8.}

\item \scriptsize{Fausett, Laurene; Fundamentals of Neural Networks; Prentice Hall Ind.; N. Jersey; 1994.}

\item \scriptsize{Freeman, J. A.; Skapura, D. M.; “Neural networks – Algorithms, Applications and Programming Techniques”; Addison- Wesley Pub. Co.; New York; 1991.}

\item \scriptsize{Luger, George F.; Inteligência Artificial; Artmed Ed. S.A.; P. Alegre; 2004.}

\item \scriptsize{Mitchell, M.; “An introduction to genetic algorithms”; The MIT press; London; 1966.}

\item \scriptsize{Rabuske, R. A.; Inteligência Artificial; UFSC; Florianópolis; 19??}

\item \scriptsize{Resende, Solange O., Sistemas Inteligentes - Fundamentos e aplicações, Ed. Manole (www.manole.com.br), 200?}

\item \scriptsize{Rich, E.; “Artificial Intelligence”; McGraw-Hill Book Company; USA; 1983.}
\item \scriptsize{Material didático disponível em: www.inf.ufsc.br/\~~falqueto}

\end{itemize}

\end{frame}




%%%%%%%%%%%%%%%%%%%%%%%%%%%%%%%%%%%%%%%%%%%%%%%%%%%%%%%%%%%%%%%%%%%%%
