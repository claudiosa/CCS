
\section{O Curso}
\begin{frame}

    \frametitle{Conteúdo do Curso}

    \begin{itemize}
      \item Conceitos de SMA (há muitos correlacionados há áreas diversas)
      \item Ferramentas: Netlogo e Picat
      \item Aplicação: voces escolhem
      \item Um artigo $ \equiv $ projeto
      \item Um artigo OUTRO da área a ser apresentado: ficha técnica
      
    \end{itemize}
\end{frame}

\subsection{Ferramentas}
\begin{frame}

    \frametitle{Ferramentas}

    \begin{itemize}
      \item PICAT (com suporte)
      \item NETLOGO
       \url{http://ccl.northwestern.edu/netlogo/docs/} (escondido in WEB)
      
    \end{itemize}
\end{frame}


\subsection{Avaliação}
\begin{frame}

    \frametitle{Avaliação}

    \begin{itemize}
      \item Duas provas (conceituais) -- 25\%
      \item Exercícios de laboratório  -- 10\%
      \item Implementação de um protótipo  -- 20\%
       \item O artigo (resultados da implementação)  -- 30\%

      \item Para o artigo: muito material será fornecido em \LaTeX ...

      \item Apresentação de um artigo estudado sobre SMA -- 15\%
 
      \item Presença e participação
      
    \end{itemize}
\end{frame}



\subsection{Dinâmica}
\begin{frame}

    \frametitle{Dinâmica de Aula}

    \begin{itemize}
      
      \item Teoria na parte da manhã -- 10:00 hrs -- K-107
      \item \textit{Ralação} a tarde -- LAB -- estudar o NetLogo -- vídeo-aulas
      
    \end{itemize}
\end{frame}



%%%%%%%%%%%%%%%%%%%%%%%%%%%%%%%%%%%%%%%%%%%%%%%%%%%%%%%%%%%%%%%%%%%%%

\subsection{Referências}
\begin{frame}
    \frametitle{Referências}
    \begin{itemize}
     \item \url{https://github.com/claudiosa/CCS/tree/master/https://github.com/claudiosa/CCS/tree/master/sistemas_multiagentes}
     \item lista de discussão da aula
    \end{itemize}
\end{frame}

%%%%%%%%%%%%%%%%%%%%%%%%%%%%%%%%%%%%%%%%%%%%%%%%%%%%%%%%%%%%%%%%%%%%%
