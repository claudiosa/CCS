


\section{Projetos de SMAs}

\begin{frame}

\begin{center}
{\huge Capítulo 6 -- Projetos de SMAS}
\end{center}

\end{frame}


%------------------------------------------------------------------------
\begin{frame} %[allowframebreaks=0.9]


\frametitle{O que vai ter neste capítulo}

\begin{itemize}
  \item Alguma metodologia?
  \item Ambiente simulado: SIM (implementar)
  
  \item O que considerar na construção de SMAs
  \item Perspectivas: reais e visionárias
\end{itemize}


\end{frame}

%-----------------------------------------------------------------------------------

\section{Implementação de SMAs}
\begin{frame}

    \frametitle{Agentes: Metodologia de desenvolvimento}
    \begin{itemize}
    \pause
      \item Decompõem o problema em: \\
      percepções, ações, objetivos, ambiente e outros agentes
\pause
      \item Decompõem tipo de conhecimento em:
      \begin{itemize}
        \item Quais são as propriedades relevantes do mundo?
        \item Como o mundo evolui?
        \item Como identificar os estados desejáveis do mundo?
        \item Como interpretar suas percepções?
        \item Quais as conseqüências de suas ações no mundo?
        \item Como medir o sucesso de suas ações?
        \item Como avaliar seus próprios conhecimentos?
    
      \end{itemize}
      
      \item O resultado dessa decomposição indica a arquitetura e o método de resolução de problema (raciocínio)
      
    \end{itemize}
\end{frame}

%-----------------------------------------------------------------------------------

\subsection{Simulação de Ambientes}
\begin{frame}

\begin{block}{Em geral a construção do agent segue um programa tal como:}
  

  funcao simulaAmbiente (estado, funcaoAtualizacao, 
                         agentes, final)
repita
	para cada agente em agentes faça
		 Percept[agente] := pegaPercepcao(agente,estado)
	para cada agente em agentes faça
		 Action[agente] := Programa[agente] (Percept[agente])
      estado := funcaoAtualizacao(acoes, agentes, estado)
	  scores := avaliaDesempenho(scores,agente,estado) 
	            opcional
ateh\_final

  
 \textcolor{red}{ Cuidado para não cair em tentação e ''\textit{roubar}” \/ do ambiente a descrição do que aconteceu. Usar a memória do agente!}
 
 
 
\end{block}  
    
  
\end{frame}
%-----------------------------------------------------------------------------------

\begin{frame}
\frametitle{Desenvolvendo um agente inteligente:}
\begin{block}{}

\begin{description}
  \item[Projeto]
  
  \begin{itemize}
    \item Modelar tarefa em termos de ambiente, percepções, ações, objetivos e utilidade
    \item Identificar o tipo de ambiente
    \item Identificar a arquitetura de agente adequada ao ambiente e tarefa
  \end{itemize}
  
  \item[Implementação]
   
   \begin{itemize}
     \item O simulador de ambientes
     \item Componentes do agente 
     \item Testar o desempenho com diferentes instâncias do ambiente
   \end{itemize}
      
\end{description}

\end{block}  
 
\end{frame}
%-----------------------------------------------------------------------------------

\begin{frame}

    \frametitle{Simulação de Ambientes}

     ESTRUTURAS ... ver NORVIG -- pg 34 da T

\end{frame}
%----------


%-----------------------------------------------------------------------------------

\begin{frame}

    \frametitle{Simulação de Ambientes}

     ESTRUTURAS ... ver NORVIG -- pg 34 da T

\end{frame}
%-----------------------------------------------------------------------------------








\begin{comment}
\sectio{Aprendizagem}
\begin{frame}

    \frametitle{Aprendizagem}
    \begin{itemize}
    \pause
      \item 
\pause
      \item cap 7
    
    \end{itemize}
\end{frame}
\end{comment}
