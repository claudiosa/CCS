
\section{Teoria de Jogos e Coordenação}
\begin{frame}

\begin{center}
{\huge Capítulo 4 -- Teoria de Jogos e  Coordenação}

3 partes fundamentais:

\begin{enumerate}
  \item Teoria de Jogos
  \item Coordenação
  \item Planejamento (capítulo 5)
\end{enumerate}
Nesta ordem
\end{center}

\end{frame}

%-----------------------------------------------------------


%-----------------------------------------------------------


\section{Estratégias de Jogos}
\begin{frame}

    \frametitle{Teoria de Jogos}
    \begin{itemize}
    \pause
      \item SMA como um comunidade que coopera, \textit{compete},  negocia, etc
      \pause
      \item Asssim é necessária uma métrica 
      entre todos os agentes (\textit{local})
        para levantar um valor \textit{global} de eficiência $\Rightarrow \Rightarrow $ Teoria de Jogos (TJ)
        
        \pause
      \item TJ: uma área a parte que coopera com SMAs
    
    \end{itemize}
\end{frame}


\begin{frame}

    \frametitle{Contextualizando a Teoria de Jogos}
    \begin{itemize}
    \item SMA tem uma visão social $\Rightarrow $ o que seria bom para comunidade toda de agentes?
    
    \item Em TJ, como a decisão \textbf{individual} afeta coletivamente
    \item Estabelecida as regras do jogo (\textit{mecanismo de negociação}),
    qual é a estratégia que seria adotada pelo agente racional?
    
    \item Hoje: TJ via exemplo do \textit{dilema do prisioneiro},
    identificando conceitos tais como: estratégias dominantes, de equilíbrio, equilíbrio de Nash etc.
    
    \end{itemize}
\end{frame}


%-----------------------------------------------------------

\subsection{Teoria de Jogos Aplicado a SMA}

\begin{frame}

\frametitle{Dilema do Prisioneiro}

    
\begin{figure}[!ht]
\centering
\includegraphics[height =.6\textheight, width=.7\textwidth]{figuras/dilema_prisioneiros01.jpg}
\caption{Memorize a idéia}
%\label{ag_01}
\end{figure}


\end{frame}





\begin{frame}

    \frametitle{Dilema do Prisioneiro}
   
    
\begin{quotation}
        Houve um assassinato e existem dois suspeitos, A e B. Se não se conseguir provar quem foi o assassino, a pena seria de apenas 6 meses por porte de arma.
Se um suspeito acusa (trair ou delatar) o outro e este não se defender (fica calado), então  será condenado a 10 anos. Assim, o traidor que colabora com a polícia sairá livre. Se os dois se acusarem mutuamente a pena é de 5 anos para cada um, pois a
polícia não acredita em nenhum dos dois.
    \end{quotation}
    

   
\end{frame}
    
\begin{frame}

    \frametitle{Dilema do Prisioneiro}
   
\begin{itemize}
  
  \item Como os suspeitos não conhecem a \textit{Teoria dos Jogos}, o normal será que se acusem mutuamente (agentes racionais)
    
  \item Os suspeitos serão interrogados em separado e não tem acesso a(s) resposta(s) do outro

\item Sim, plural: \textit{rodadas} ou \textit{jogadas} de respostas!

\item Muitas \textit{rodadas} leva a pontos como

\begin{itemize}
  \item \textbf{Estratégias puras}: não mudam ao longo do jogo = \textbf{determinísticas}. Exemplo: sempre mover a peça a frente
  
  
  \item Estratégias mistas: mudam ao longo do jogo 
  
  \item Podem olhar a última jogada sua e  de seu adversário
  
  \item Podem olhar o seu histórico de jogadas e  de seu adversário
  
  \item Nestes 3 últimos casos: complexidade $cresce^{\nearrow }$ muito
\end{itemize}


\end{itemize}

   
\end{frame}

%-----------------------------------------------------------

\begin{frame}

    \frametitle{Coletando os Dados: Dilema do Prisioneiro}
    
    
      \begin{center}
        \begin{tabular}{p{2cm} || p{3cm} | p{3cm}} \hline \hline
  & Prisioneiro B nega   &  Prisioneiro B delata   \\ \hline \hline
        Prisioneiro A nega (silêncio ou não-confessa)  &  Ambos condenados a 6 meses  & A é condenado a 10 anos e B é livre   \\ \hline 
         Prisioneiro A delata (acusa ou trai)     & B é condenado a 10 anos e A é livre & Ambos são condenados a 5 anos \\ 
         \hline \hline
        \end{tabular}
      \end{center}

\end{frame}
%-----------------------------------------------------------


\begin{frame}

    \frametitle{Hipóteses: Dilema do Prisioneiro}
    
    
    \begin{itemize}
      \item  Vamos supor que ambos os prisioneiros são completamente egoístas e a sua única meta é reduzir a sua própria estadia na prisão. Como prisioneiros têm duas opções:
      
      \begin{enumerate}
         \item cooperar com o seu cúmplice e permanecerem calados
         \item ou trair o seu cúmplice e confessar que foi o outro (agentes racionais)
       \end{enumerate}   
       
       \begin{itemize}
         \item   O resultado de cada escolha depende da escolha do cúmplice.     
         \item   Infelizmente, um não sabe o que o outro escolheu fazer. 
       
         \item Eis o \textit{impasse}: é um dilema, qual a melhor jogada?
         
       \end{itemize}
       
       \item Inclusive se pudessem falar entre si, não poderiam estar seguros de confiar um no outro (um diria ao outro que era melhor ficarem calados, e depois trairiam o outro)        
       
    \end{itemize}


\end{frame}

%-----------------------------------------------------------

\begin{frame}
    \frametitle{Analisando os fatos:}

%\begin{small}
\begin{quotation}
Se  esperar que o cúmplice escolha cooperar com ele e permanecerem em silêncio, a opção ótima (individualmente) para o primeiro seria confessar, o que significaria que seria libertado imediatamente, enquanto o cúmplice terá que cumprir uma pena de 10 anos. Se espera que seu cúmplice decida confessar, a melhor opção é confessar também, já que ao menos não receberá a pena completa de 10 anos, e apenas terá que esperar 5, tal como o cúmplice. Se ambos decidirem cooperarem entre si e permanecerem em silêncio, ambos serão libertados em apenas 6 meses.
\end{quotation}
%\end{small}


\end{frame}

%-----------------------------------------------------------

\begin{frame}
    \frametitle{Análise: Dilema do Prisioneiro}

   \begin{itemize}
     \item Acusar o outro (confessar que foi o outro)  é uma \textit{estratégia dominante} para ambos os jogadores. Seja qual for a escolha do outro jogador, podem reduzir sempre sua sentença acusando o outro. 
   
  \item Por infelicidade para os prisioneiros, isto conduz a um resultado ruim, no qual ambos acusam seus companheiros e ambos recebem longas condenações. \textcolor{red}{Aqui se encontra o ponto-chave do dilema ou impasse} 
  
  \item O resultado das acusações individuais produz um resultado que não é ótimo no sentido de Pareto; existe uma situação tal que a utilidade de um dos detidos poderia melhorar (ou mesmo a de ambos) sem que isto implique uma piora para o resto. 
  
  \item \textit{Pareto dominante} é quando há soluções muito boas para um ou mais jogadores. Exemplo: acusar o outro é uma resultado interessante!
  
  \item Ou seja, o resultado no qual ambos os detidos não confessam (silêncio) dominam o resultado no qual os dois escolhem confessar.


   \end{itemize}
   
\end{frame}


%-----------------------------------------------------------


\begin{frame}
    \frametitle{Ótimo de Pareto}

 \begin{itemize}
  
 
   \item Perspectiva de interesse ótimo para o grupo (i.é. dois prisioneiros), o resultado correto é que  ambos cooperassem (entre si, ficando calados).   Pois,  isto reduziria o tempo total de pena do grupo a um total de um ano (6 meses para cada um).

   \item  Qualquer outra decisão seria pior para ambos se considerar \textbf{conjuntamente}.

\item Se um jogador tiver uma oportunidade de castigar o outro jogador ao confessar (delatar o outro), então um resultado cooperativo (um vai para prisão apenas?) pode manter-se.

\pause
   
   
\item Neste jogo tem como solução do ponto de vista \textit{Ótimo de Pareto} a estratégia: A e B negam (ficam calados!)

\item Aqui  é o caso de \textit{Pareto dominado}, pois ambos jogadores prefeririam esta solução, as soluções de \textit{Pareto dominante} (agentes racionais $\Rightarrow $ querem o melhor para si!)
     
 \end{itemize}
 
 
\end{frame}
%-----------------------------------------------------------
 
 
\begin{frame}
    \frametitle{John Nash}
 
     
\begin{figure}[!ht]
\centering
\includegraphics[height =.6\textheight, width=.7\textwidth]{figuras/john_nash01.jpg}
%\caption{Memorize a idéia}
%\label{ag_01}
\end{figure}

\end{frame}
%-----------------------------------------------------------

 
 
\begin{frame}
    \frametitle{Equilíbrio de Nash}

 \begin{itemize}
   
\item Este jogo possui como \textit{Equilíbrio de Nash} de estratégias: 
\textbf{A e B delatam}! Neste caso, o \textit{equilíbrio é dominante}.

\item Porquê é chamado de \textit{Equilíbrio de Nash}?
\pause
 
 \item John Nash: todo jogo tem pelo menos um equilíbrio (mas que não necessariamente domina as estratégias dominantes)
 
\item \textit{Equilíbrio de Nash}: \textit{se a estratégia adotada por A é a melhor dada à estratégia adotada por B e a estratégia adotada por B é a estratégia ótima dada a adotada por A. Ou seja, nenhum dos jogadores pode aumentar seu ganho alterando, de \textbf{forma unilateral}, sua estratégia.}

%%  $u_i(a) \ge u_i(a_{-i}, a^*_i)$
    
  \item   Então o EN é um perfil de uma  ação $a$ tal que $u_i(a) \ge u_i(a_{-i}, a^*_i)$ para cada jogador $i$ e toda ação $a^*$ de $i$
\pause
 
\item No exemplo, assuma que A ou B escolham trair. O que resta para o outro fazer? Trair também! 

\item Ou seja, nenhum dos dois jogadores é benevolentes em sua estratégia (sim, leia-se: jogada)
 %%Apesar disso, se continuarem no seu próprio interesse egoísta, cada um dos dos prisioneiros receberá uma dura pena.
 
% A forma iterada de este jogo (mencionada mais abaixo) oferece uma oportunidade para este tipo de castigo. Nesse jogo, se o cúmplice trai e confessa uma vez, pode-se castigá-lo traindo-o na próxima. Assim, o jogo iterado oferece uma opção de castigo que está ausente no modo clássico do jogo.

 \end{itemize}

\end{frame}
%-----------------------------------------------------------


\begin{frame}
    \frametitle{Formulação: Dilema do Prisioneiro}

\begin{itemize}
  \item $Jogadores = \{A,B\}$
  \item Espaço de estratégia do jogo: $S_A = \{confessa, \:\: nega\}$ e  $S_B = \{confessa, \:\: nega\}$
  \item Espaço do jogo: $S = \{ (nega, nega), (nega,confessa), (confessa, nega), (confessa, confessa) \}$
  \item Função de utilidade: $u_i: S \rightarrow R$ onde $i \in \{A,B\}$
  
  \item No caso $u_A (s): S \rightarrow R$ e $u_B (s): S \rightarrow R$

  \item Assim o mapeamento desta \textbf{função de utilidade} dos jogadores $A$ e $B$:
  
  \begin{itemize}
    \item Para o jogador A:  $u_A (nega, nega) = -1$, $u_A(nega, confessa) = 0$, $u_A(confessa, nega) = -10$  e  $u_A(confessa, confessa) = -5$
  
    \item  Para o jogador B:  $u_B(nega, nega) = -1$, $u_B(nega, confessa) = -10$, $u_B(confessa, nega) = 0$  e  $u_B(confessa, confessa) = -5$
    
  \end{itemize}
  
\end{itemize}


\end{frame}


%-----------------------------------------------------------
\begin{frame}
\frametitle{Construindo uma Matriz de Ganhos:}

\begin{table}[!ht]
  \caption{Função Utilidade}
  \begin{center}
  
    \begin{tabular}{c||c|c}
    \hline \hline
               & B Nega & B Delata  \\     \hline \hline
     A Nega\footnote{Ficar calado.}    &   (-1,-1)     &  (-10, 0)  \\     \hline 
     A Delata\footnote{Trair o outro.}  &    (0, -10)   &  (-5,-5)   \\
         \hline \hline
    \end{tabular}
  \end{center}

  \label{tab_01}

\end{table}


\begin{itemize}
  \item Sendo \textit{egoísta}, a melhor estratégia é delatar seu companheiro e este não se defender
  
  \item Assim, seja qual for a opção do adversário, individualmente, qualquer um se sai melhor traindo ou delatando o outro
  
  \item Ambos chegam individualmente a conclusão \textbf{racional}: \textit{trair}!
  
  \item Nesta lógica individual, as penas somam 10 anos de prisão!
    
  \end{itemize}


\end{frame}

%----------------------------------------------------------

\begin{frame}
\frametitle{Reflexões: Dilema do Prisioneiro}

\begin{itemize}
  \item E se os jogadores aprendessem a \textit{cooperar} após sucessivas jogadas?

  \item Assim suge o \textit{princípio da reciprocidade}, onde o jogador $B$   deve cooperar com $A$ seguindo-o na sua escolha

\pause 
  \item Do livro \textit{A evolução da cooperação: o dilema do prisioneiro e a teoria de jogos} (1984),  de Robert Axelrod, tem-se a terminologia do \textit{ganha-ganha}: 
    
  \begin{center}
    \begin{tabular}{c||p{3cm}|p{3cm}}
    \hline \hline
                & B Coopera & B Delata  \\     \hline \hline
     A Coopera  &   ganha -- ganha     & perda substancial -- ganho substancial  \\     \hline 
     A Delata   &    	ganho substancial -- perda substancial &  	perde -- perde   \\
         \hline \hline

    \end{tabular}
  \end{center}

\end{itemize}

\end{frame}
%----------------------------------------------------------

\begin{frame}
\frametitle{Reflexões: Dilema do Prisioneiro}

\begin{itemize}
\item Outros exemplos: corrida de bicicleta (\textit{Tour de France}, Giro na Itália, etc)  onde há um revezamento na liderança para se poupar, nem se assume a liderança isolada para não se desgastar!

\item O problema da troca de malas

\item \textit{The refund}: recompensa em dizer a verdade 

\item ... ver outros exemplos e TODOS se identificam

\item Como integrar isto a SMAs?
\end{itemize}

\end{frame}

\section{Teoria de Jogos e SMA}


\begin{frame}
\frametitle{Correlação TJ e SMA}

\begin{itemize}
  \item Número de agentes (n): $n > 1$
  \item Cada agente escolhe \textbf{uma estratégia} ou \textbf{uma ação}: $a_i$
  \item Um \textbf{perfil de ações} ou  de \textbf{ações conjuntas} é definido pea tupla de ações individuais: $(a_1, a_2, ...a_n)$
  
     \item Define-se uma ação de todos agentes menos do agente $i$ por: $a_{-1}$
  
  \item Assim uma sequência $(a_i,a_{-1})$ indica que o agente $i$ fez uma ação, e no estado seguinte
  todos  (ou um só, no caso de jogo de adversário) fizeram uma ação conjunta, menos o agente $i$
  
  \item O jogo ocorre em estados  (discretos), com suas recompensas/punições fixas
  
  \item Cada agente tem uma função de avaliação, que indica a \textit{bondade}   ou \textit{intencionalidade} do agente
\end{itemize}

\end{frame}




\begin{frame}
\frametitle{Correlação TJ e SMA}

\begin{itemize}
  \item O estado é observável por todos agentes
  \item Há um \textit{conhecimento comum} e igual por todos agentes
  \item Cada agente ao realizar uma ação $\Leftrightarrow $ jogo de \textbf{\textit{tiro-único}} (visto anteriormente)
  \item Todos agentes escolhem as ações: simultaneamente e individualmente
  \item Nenhum agente é informado sobre o \textbf{{\em tiro}} ou \texttt{\textit{decisão}} assumida pelos demais agentes
  \item Resultado do jogo: \textit{\underline{depende da seleção conjunta de todas as ações}}!
  \item Uma solução de jogo: uma predição de resultado do jogo, assumindo que todos agentes
  são racionais em suas estratégias!
\end{itemize}

\end{frame}



\begin{frame}
\frametitle{Reflexão TJ e coordenação de SMAs}

\begin{itemize}
  \item Antes de coordenação ....
  \item SMA com a parcialidade de observação do mundo (ver figuras da introdução),
  força que agentes racionais interajam entre si!
  \item O que é o \textbf{conhecimento} comum (global) e o individual (local) entre os agentes $\Rightarrow $ comprometimento
  \item \textbf{\textit{Observalidade parcial}} dos agentes $\Rightarrow $ há várias consequencias nas tomadas de decisões dos agentes
  \item Exemplo: n-agentes com \textit{observalidade parcial} em um planejamento 
  de suas ações leva a um problema intratável
  \item Então a \textit{função utilidade} é uma medida ... um ponto de partida!
  
\end{itemize}

\end{frame}

%-----------------------------------------------------

\begin{frame}
\frametitle{OK, vamos jogar!}

 Cara e coroa:

    \begin{center}
      \begin{tabular}{c||c|c}
                 & Cara -- B & Coroa -- B \\ \hline \hline
      Cara -- A   & 1, -1     &  -1, 1      \\ \hline
      Coroa -- A   & -1, 1   & 1, -1         \\ \hline \hline
      \end{tabular}
    \end{center}

\pause 

\begin{itemize}
  \item Tabela de penalidade (\textit{pay-off}) de um jogo \textbf{estritamente competitivo} ou \textbf{soma-zero}
 \item Pois $u_1(a) + u_2(a) = 0$
\end{itemize}

\end{frame}

%----------------------------------------------------------

\begin{frame}
\frametitle{OK, vamos jogar!}



 Dois carros (motorista A e motorista B) em um cruzamento, querem cruzar \textit{logo}:

    \begin{center}
      \begin{tabular}{c||c|c}
                 & Avance -- B & Pare -- B \\ \hline \hline
      Avance -- A   & -1, -1     &  1, 0      \\ \hline
      Pare -- A     & 0, 1   & 0, 0         \\ \hline \hline
      \end{tabular}
    \end{center}
    
\pause 
\begin{itemize}
  \item Tabela de penalidade (\textit{pay-off}) para um jogo típico de \textbf{coordenação}. 
  
  \item Pois
se os dois carros desejarem avançar teremos: $u_1(a) + u_2(a) = -2$, consequentemente
uma batida.

\end{itemize}

\end{frame}

%----------------------------------------------------------

\begin{frame}
\frametitle{Exercícios}

\begin{enumerate}
  \item Construa a função de utilidade para jogo \textbf{pedra--papel--tesoura} para este se torne um jogo competitivo. Justifique suas escolhas:
  
    \begin{center}
      \begin{tabular}{l||c|c|c}
                 & Pedra -- B & Papel -- B & Tesoura -- B \\ \hline \hline
      Pedra -- A     &       &     &   \\ \hline
      Papel -- A     &       &       &   \\ \hline
      Tesoura -- A   &       &       &   \\ \hline \hline
      \end{tabular}
    \end{center}


\item Nos 3 últimos exemplos identifique os elementos e os justifique:
\begin{enumerate}
    \item O que é uma estratégia em cada jogo
    \item Há um Pareto dominante?
    \item Qual é o Pareto ótimo?
    \item Onde está o equilíbrio de Nash? (pode haver mais de um)
  \end{enumerate}  
  
\end{enumerate}

\end{frame}

%----------------------------------------------------------




%%% longe


\section{Coordenação}


\begin{frame}
\frametitle{Coordenação}
\begin{itemize}
  \item Um tema amplo em SMA ...
  \item Coordenação $\approx $ a não obstrução entre os agentes 
  \item ou decisões individuais dos agentes que possam levar
  há uma boa decisão conjunta do grupo
\end{itemize}


\end{frame}


%-----------------------------------------------------------

\begin{frame}
\frametitle{Coordenação}
\begin{itemize}
  \item Ainda ... a medida usando a TJ é o equlíbrio de Nash
  \item Exemplo: o exemplo dos carros no cruzamento, tem DOIS (pontos) de equlíbrio de Nash
  \item Neste caso: $u_A(para,avanca) = u_B(avanca, para)$
  \pause
  \item Neste sentido, em que todos agentes $n$ compartilham a mesma função de utilidade, eles serão colaborativos se:
  $u_1(\overrightarrow{acoes}) = u_2\overrightarrow{(acoes}) = \ldots \ldots  u_n(\overrightarrow{acoes})$
  
  \item A notação acima é uma simplificação de: 
  $u_i(a_i, \overrightarrow{a_{-i}})$ 
  
  \end{itemize}


\end{frame}

%-----------------------------------------------------------

\begin{frame}
\frametitle{Abordagens para  Coordenação}
 
 Estruturando o tema de \textit{coordenação}, o mesmo
 discutido segundo suas estratégias:

 \begin{enumerate}
   \item Jogos de Coordenação (fundamentação no capítulo anterior)
   \item Convenções (função) Sociais
   \item Papéis
   \item Grafos de Coordenação, subdivide-se em:
        \begin{enumerate}
          \item Coordenação por Eliminação de Variável
          \item Coordenação por Troca de Mensagem
        \end{enumerate}
   
   
 \end{enumerate}




\end{frame}



%-----------------------------------------------------------


\subsection{Jogos de Coordenação}

\begin{frame}
\frametitle{Jogos de Coordenação}

\begin{itemize}
  \item Exemplo de um casal $\{Abel, Beatriz\}$ que gostam de dançar
    \begin{center}
      \begin{tabular}{c || c | c}
                 & Decansar -- B & Dansar -- B \\ \hline  \hline
      Decansar -- A   & 1, 1     &  0, 0      \\ \hline
      Dansar -- A     & 0, 0     &  1, 1         \\ \hline
      \end{tabular}
    \end{center}

novamente, tem-se dois equilíbrios de Nash e ótimos de Pareto

\item Generalizando estes 2 exemplos (o outro era dos carros no cruzamento), 
define-se formalmente  \textbf{coordenação}
 
 \item \textbf{Coordenação:} \textit{é um processo no qual  um grupo de agentes escolhe um Pareto ótimo e um equílibrio de Nash no jogo}
 
 \item Compartilham a mesma função utilidade
  (\textit{payoff function} -- fugiu a tradução de penalidade), logo, eles  serão
  colaborativos!
 
  \item Vale ressaltar que nesta abordagem os agentes compartilham a mesma função de utilidade $\Rightarrow $ veja tabelas dos exemplos
  
 
\end{itemize}
\end{frame}
















%-----------------------------------------------------------
\subsection{Convenção Social}

\begin{frame}
\frametitle{Convenção Social}[allowframebreaks=0.9]

\begin{itemize}
  \item Como escolher a ação?
  \item Infelizmente, não há tal receita que conduza ao Equilíbrio de Nash (EN)!
  \item Assim receitas devem ser passadas aos agentes visando o EN

\pause
  \item Convenção social (ou lei social): é uma \textit{receita} que visa
  restringir as ações  dos agentes.
  
  \item Assim, os agentes seriam \textit{conduzidos} a terem um comportamento
  social, visando um EN!
  
  \item Que \textit{receitas} seriam estas? Exemplos:
  \pause
  \begin{enumerate}
    \item Agente \textbf{A} escolhe depois de \textbf{B} (afinal no par, \textbf{B} é a Beatriz)
    \item A agente \textbf{B} é animada, sempre quer dançar!
  \end{enumerate}
  
\item A partir disto, o EN é investigado dado os $n$ agentes da comunidade


  \item Sob esta Convenção Social (CS) o conhecimento é comum e nenhum agente pode se
  beneficiar disto para sempre!
 
\end{itemize}


\end{frame}
%-----------------------------------------------------------


\begin{frame}
\frametitle{Convenção Social -- Comentários}[allowframebreaks=0.9]

\begin{itemize}
  \item Coordenação que usam CS podem se fazer uso de ordenamento dos agentes -- fácil de implementar
  
  \item xxx
  
  
\end{itemize}



\end{frame}



%-----------------------------------------------------------

\subsection{Papel Social}

\begin{frame}
\frametitle{Papel Social}

EM 2 semanas
pag 25


\end{frame}



%-----------------------------------------------------------
\subsection{Grafos de Coordenação}

\begin{frame}
\frametitle{Grafos de Coordenação}

pag 26


\begin{huge}

\textbf{ \textcolor{red}{Falta terminar}}
 

\end{huge}
\end{frame}
%-----------------------------------------------------------


\subsubsection{Coordenação por Eliminação de Variáveis}

\begin{frame}
\frametitle{Coordenação por Eliminação de Variáveis}

pag 28


\end{frame}
%-----------------------------------------------------------


\subsubsection{Coordenação por Troca de Mensagens}

\begin{frame}
\frametitle{Coordenação por Troca de Mensagens}

pag 28


\end{frame}
%-----------------------------------------------------------

\section{Planejamento}


\begin{frame}
\frametitle{Fundamentos de Planejamento}



\end{frame}
%-----------------------------------------------------------


\subsection{Abordagens ao Planejamento Multiagente -- SMAs}

\begin{frame}
\frametitle{Abordagens ao Planejamento de SMAs}

\begin{block}{}
 
\begin{itemize}
  \item Coordenação central: controla todos os subplanos
  \item Esquemas de controle distribuído\\
        Conhecimento parcial dos planos de outros agentes
  \item Planejamento Global Negociado

\begin{itemize}
  \item Compartilhamento de todos os planos
  \item Ajuste local para a realização de objetivos comuns

\end{itemize}

\item Modelagem Explícita da Equipe de Agentes
\begin{itemize}
  \item Compromissos conjuntos
   \item Crenças, desejos e intenções comuns

\end{itemize}
\end{itemize}
\end{block}

\end{frame}

%-----------------------------------------------------


\subsection{Exemplos de Coordenação SMAs}

\begin{frame}
\frametitle{Exemplo de Coordenação SMAs}

\begin{figure}[!ht]
\centering
\includegraphics[height =.6\textheight,width=.7\textwidth]{figuras/coordenacao_agentes01.png}
\caption{Coordenação de agentes $\equiv $   SMA}
%\label{ag_01}
\end{figure}
 \end{frame}

%-----------------------------------------------------------

\begin{frame}
\frametitle{Exemplo de Coordenação SMAs}

\begin{figure}[!ht]
\centering
\includegraphics[height =.6\textheight,width=.7\textwidth]{figuras/coordenacao_agentes02.png}
\caption{Coordenação de agentes $\equiv $   SMA}
%\label{ag_01}
\end{figure}
 
\end{frame}


%-----------------------------------------------------------
