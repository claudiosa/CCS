\documentclass[12pt,a4paper]{article}
\usepackage{t1enc}
\usepackage[latin1]{inputenc}
\usepackage[portuges]{babel}
\pagestyle{headings}
\markright{}
\topmargin-1cm
\oddsidemargin20mm
\textwidth17cm
\textheight24cm
\setlength{\parskip}{0.5cm}
\setlength{\parindent}{1cm}

\begin{document}
\baselineskip0.5pt

 Sistema de Tele-Pizza

A proposta deste sistema � gerenciar uma pequena pizzaria de ``{\em tele-entregas}". Ou seja, um cliente liga � pizzaria, faz o seu pedido, recebe a pizza em sua casa, e a baixa no pedido � feita pelo entregador quando este recebe o dinheiro.

Como o pedido � feito por telefone, � ``{\em natural}" que os clientes sejam cadastrados pelo n�mero do telefone. Caso um n�mero ainda n�o exista, este ser� inclu�do mediante um pedido.

As pizzas s�o de v�rios sabores e pre�os variados. Estas se apresentam sob forma de menu, com c�digos a partir de 1, 2, 3, \ldots . A cada pizza h� um pre�o caracter�stico.  Esta estrutura pode ser fixa internamente. Tipo um vetor de estruturas.

Um pedido tem uma demanda de uma ou mais pizzas.
Logo, na medida que o pedido � feito, h� o c�lculo de um sub-total, e um total ao final do mesmo.


``{\em Lay-out}" no arquivo mestre:
\begin{verbatim}
structuc reg_tipo_1
        {
          char fone[8];
          char nome[20];
          char endereco[40];
          char descricao_pizzas[30];
          int qtidade;
          float valor_total;
          char pend�ncia;
          char status;
        }
\end{verbatim}
          
\begin{verbatim}
structuc reg_tipo_2
        {
          char fone[8];
          int posicao;
         }
\end{verbatim}

Quando um pedido for realizado, o campo da pend�ncia segue para ``{\bf S}", ou seja, o pedido est� pendente.
Quando o pedido for entregue e pago, o campo da pend�ncia segue para ``{\bf N}". 


Em resumo, o �ltimo pedido de um cliente sempre fica armazenado no sistema.

Obviamente, que os m�dulos de altera��o, exclus�o s�o partes do sistema. A altera��o visa as mudan�as dos dados do cliente, e a exclus�o � para remover um cliente do sistema (caso este tenha se mudado de cidade).






\end{document}