Pattern matching is used in functional languages. As Picat supports explicit unification, it is trivial to translate a Prolog clause into Picat if the clause does not contain cuts. For a clause p(A1,A2,��,An):-Body, you can translate it to 

P(NA1,NA2,��,NAn) ?=> NA1=A1,NA2=A2,��,NAn=An,Body.

However, the resulting program is not efficient since rules are not indexed.

For a pattern like p(X,[X]), it matches a call like p(a,[a]) but not p(a,[b]) or p(X,[a]) or p(a,[X]). In general, a pattern P matches a term T if there is a substitution theta such that P_theta is identical to T. For general purpose programming, matching is sufficient although there are rare cases where unification is more convenient.

The compress/2 should be written in Picat as:
compress([],Y) => Y=[].
compress([X],Y1) => Y1=[X].
compress([X,X|Xs],Zs) => compress([X|Xs],Zs).
compress([X,Y|Ys],Z1) => Z1=[X|Zs], compress([Y|Ys],Zs).
All the rules should be non-backtrackable and the condition X!==Y is not needed in the fourth rule since the third rule takes care of the case when X and Y are equal. If you compare this definition with the Prolog definition, you will find that this definition is much faster.
For the predicate transform, it depends on how it is used. If the first argument is input, then it can be defined in Picat as follows:
transform([],Zs) => Zs=[].
transform([C@[X|_]|Ys],Zs) => Zs=[[C.length,X]|ZsR], transform(Ys,ZsR).
