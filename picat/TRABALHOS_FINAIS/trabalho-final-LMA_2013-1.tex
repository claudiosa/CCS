\documentclass[a4paper,11pt]{article}

\usepackage[a4paper,left=27mm,right=27mm,top=10mm,bottom=15mm]{geometry}
\usepackage{graphicx,url}
\usepackage{color,comment}
\usepackage{amssymb}
\usepackage[utf8]{inputenc}
\usepackage[brazilian]{babel}
\usepackage[T1]{fontenc}

\begin{comment}
\usepackage[T1]{fontenc}
\usepackage[utf8]{inputenc}
\usepackage{lmodern}
\usepackage[francais]{babel}
\end{comment}

\title{Lógica Matemática -- Trabalho Final}
\author{Rogério Eduardo da Silva e Claudio Cesar de Sá}
\date{\today}

\graphicspath{{/figures/}}   
\DeclareGraphicsExtensions{{.jpg},{.png}}


\begin{document}
\maketitle

\begin{flushleft}

Tarefa: Escolham 02 (dois) dos 3 problemas propostos abaixo.\\
Entrega: a definir com Prof. Rogério \\ 

\end{flushleft}

\begin{enumerate}

\item Dicas de como se resolve manualmente:\\
\url{http://www.valdiraguilera.net/problema-de-logica-esquema.html}
\end{enumerate}


\begin{description}

\item [Amigas na Piscina] Fonte do problema proposto:   \url{http://rachacuca.com.br/logica/problemas/amigas-na-piscina/}(tem a montagem da tabela para irem entendendo e depurando o problema).\\

Quatro amigas combinaram de aproveitar o dia de Sol na piscina. Elas estão lado a lado, prontas para mergulharem. Cada uma está com um maiô de uma cor e está usando um protetor solar de determinado fator de proteção (FPS).

Cabe a você seguir as dicas para encontrar os gostos e as características delas.

\begin{enumerate}
  \item     Na terceira posição está a menina que gosta de Cachorros.
  \item     Quem gosta de Peixes está em uma das pontas.
  \item     A garota que gosta de Gatos está na primeira posição.
  \item     Ana usa protetor solar de FPS 50.
  \item     Na segunda posição está a menina que usa filtro solar com FPS 55.
  \item     A garota mais nova está ao lado da que usa protetor solar de menor FPS.
  \item    Quem gosta de suco de Morango está na quarta posição.
  \item     A menina que gosta de suco de Maracujá está ao lado da que gosta de Pássaros.
  \item     A menina que gosta de limonada está ao lado da que gosta de suco de Maracujá.
  
  \item   Quem gosta de suco de Laranja está em uma das pontas.
  \item     A menina de maiô Azul está em algum lugar à esquerda da menina de 9 anos.
  \item     A garota de 8 anos está na quarta posição.
  \item     A garota de 11 anos está em uma das pontas.
  \item     Vivian gosta de Pássaros.
  \item     Raquel está na primeira posição.
  \item     A menina de maiô Verde está na quarta posição.
  \item     A menina de maiô Branco está em uma das pontas.
   
  
\end{enumerate}




  \item[Sala de Espera] Fonte do problema proposto:
  \url{http://rachacuca.com.br/logica/problemas/sala-de-espera/} (tem a montagem da tabela para irem entendendo e depurando o problema).\\
  Cinco homens aguardam o antendimento com hora marcada no dentista. Cada um deles tem consulta com uma dentista diferente, num horário diferente e veio em busca de um tratamento específico.
Siga as dicas e descubra o nome dos pacientes, qual revista cada um está lendo, o horário do atendimento e o motivo da consulta. 
Sejam os fatos:
  
  \begin{enumerate}
  \item A pessoa que veio fazer uma obturação está lendo a revista Caras.
  \item  Quem está lendo a Veja está em algum lugar entre o Eduardo e quem está com a revista Galileu, nessa ordem.
  \item  Quem veio ajustar o Aparelho está em algum lugar entre quem está lendo Isto É e quem veio fazer a restauração, nessa ordem.
  \item  O paciente da Dra. Raquel está lendo a revista Galileu.
  \item  O Álvaro é o paciente da Priscila.
  \item  O Marcos está em algum lugar entre o paciente de camisa Amarela e o Rodrigo, nessa ordem.
  \item  O paciente que veio fazer uma Restauração está em algum lugar entre quem veio tratar o Canal e quem veio fazer uma Obturação, nessa ordem.
  \item A pessoa de Amarelo está em algum lugar à esquerda do paciente agendado às 10:30.

    \item Quem tem consulta às 10:15 está sentado em uma das pontas.
    \item     O primeiro paciente será atendido pela Silvia.
    \item     O homem de camisa Azul está em algum lugar à esquerda de quem tem consulta as 11h.
    \item     A pessoa de camisa Verde está em algum lugar entre a pessoa com camisa Vermelha e a pessoa de camisa Azul, nessa ordem.
    \item     O quarto paciente será atendido as 10h.
        \item  O paciente da Dra Adriana está em algum lugar entre os pacientes das doutoras Raquel e Priscila, nessa ordem.
    \item     O paciente da Dra. Pricila está exatamente à direita de quem está com a revista Caras.
    \item     O homem de camisa Amarela está em algum lugar entre quem está lendo Isto É e quem veio de camisa Branca, nessa ordem.

  \end{enumerate}  




  \item[Nadadores Olímpicos] Fonte do problema proposto:
  \url{http://rachacuca.com.br/logica/problemas/nadadores-olimpicos/} (tem a montagem da tabela para irem entendendo e depurando o problema).\\

Descubra a nacionalidade, a idade e a cor da touca dos 4 nadadores olímpicos, junto com a sua especialidade e o número de medalhas conquistadas em sua carreira.

\begin{enumerate}
  \item  O nadador mais velho do grupo conquistou o maior número de medalhas.
   \item Quem usa a touca Azul está em alguma raia à esquerda do nadador com 19 anos.
   \item O nadador com 20 anos está em uma das raias das pontas.
   \item Quem gosta de beber limonada está exatamente à esquerda de quem é especialista em nado Crawl.
   \item  O atleta que bebe suco de Maracujá está em alguma raia à direita de quem usa a touca Azul.
   \item Na segunda raia está o nadador que bebe suco de Laranja.
   \item O mais velho dos nadadores está na raia número 3.
   \item Quem ganhou 3 medalhas está na segunda raia.
   \item Quem ganhou 5 medalhas está posicionado exatamente à esquerda de quem gosta de suco de Laranja.
\item  O especialista em nado Peito está na raia número 2.
\item  O especialista em nado Costas está exatamente à direita de quem conquistou 3 medalhas.
\item  O Brasileiro conquistou o maior número de medalhas entre os 4 nadadores.
\item  O nadador da China está exatamente à esquerda do nadador do Brasil.
\item  O nadador dos EUA conquistou 5 medalhas.
\item  Na última raia está o nadador com a touca branca.
\item  O especialista em nado Borboleta está ao lado de quem usa a touca Azul.
\item  O nadador Estadunidense está usando a touca verde.

 \end{enumerate}  
 
\end{description}
\end{document}
