\documentclass[sans]{beamer}
\usepackage[utf8]{inputenc}
\usepackage[T1]{fontenc}
\usepackage[brazilian]{babel}
\usepackage{fixltx2e}
\usepackage{graphicx,svg}
%\usepackage{subfig}
\usepackage{longtable}
%\usepackage{wrapfig}
%\usepackage{soul}
\usepackage{textcomp}
\usepackage{amsmath}
%\usepackage{mathtools}
\usepackage{marvosym}
\usepackage{wasysym}
\usepackage{latexsym}
\usepackage{amssymb}
\usepackage{multicol}
\usepackage{pifont} %,bbding}%%,dingbat} %%% ver manual de simbolos
\usepackage[final]{listings}

\usepackage{hyperref,url}
\usepackage{comment}

%%%\usepackage{scrhack}
%\usepackage{minted}
\definecolor{azulclaro}{rgb}{0.9,0.9,1}
\definecolor{mygreen}{rgb}{0,0.6,0}
\definecolor{mygray}{rgb}{0.5,0.5,0.5}
\definecolor{mymauve}{rgb}{0.58,0,0.82}

\lstdefinestyle{myPrologstyle}
{
    language=Prolog,
    basicstyle = \ttfamily\color{blue},
    moredelim = [s][\color{black}]{(}{)},
    literate =
        {:-}{{\textcolor{black}{:-}}}2
        {,}{{\textcolor{black}{,}}}1
        {.}{{\textcolor{black}{.}}}1
}

\usetheme{Boadilla}
%Global Background must be put in preamble
\usebackgroundtemplate%
{%
    \includegraphics[width=\paperwidth,height=\paperheight]{fundo-amarelo.jpeg}%
}



\graphicspath{{/home/ccs/Dropbox/figs_genericas/}{figuras/}}

\title[Tutorial de P.I.C.A.T]{Tutorial de P.I.C.A.T}
\author[Paulo, Claudio, Lu]{Paulo Victor de Aguiar\\
	Claudio Cesar de Sá e Outros\\ 
	Universidade do Estado de Santa Catarina -- UDESC\\
	Departamento de Ciência da Computação -- DCC\\
	Joinville -- SC}

\lstset{
  basicstyle=\small
  }


\begin{document}



\begin{frame}[fragile]   %%%% indica que o ambiente  FRAME é frágil
\maketitle
\end{frame}

%%%%%%%%%%%%%%%%%%%%%%%%%%%%%%%%%%%%%%%%%%%%%%%%%%%%%%%%%%%%%%%%%%%%%%%%

\begin{frame}[fragile]   %%%% indica que o ambiente  FRAME é frágil
\frametitle{Indice}
\tableofcontents
\end{frame}

%%%%%%%%%%%%%%%%%%%%%%%%%%%%%%%%%%%%%%%%%%%%%%%%%%%%%%%%%%%%%%%%%%%%%%%%

\section{Introdução}
\begin{frame}[fragile]   %%%% indica que o ambiente  FRAME é frágil
\frametitle{Introdução}
\begin{block}{O que é P.I.C.A.T?}
 \begin{description}
 \item [\textit{\underline{P}attern-matching}:] Utiliza o conceito de casamento de padrão. 
  
 \item [\textit{\underline{I}ntuitive}:] O Picat oferece atribuições e laços de repetição (\textit{loops}) para a programação dos dias de hoje.
 
 \item [\textit{\underline{C}onstraints}:] Picat suporta a programação por restrições.
 
 \item [\textit{\underline{A}ctors}:] Atores são chamadas orientadas a eventos.
 
 \item [\textit{\underline{T}abling}:] É possível guardar o resultado de certas operações na memória.
 
 \end{description}
\end{block}
\end{frame}

%%%%%%%%%%%%%%%%%%%%%%%%%%%%%%%%%%%%%%%%%%%%%%%%%%%%%%%%%%%%%%%%%%%%%%%%

\section{Tipo de dados}
\begin{frame}[fragile]   %%%% indica que o ambiente  FRAME é frágil
\frametitle{Tipo de dados}

\end{frame}

%%%%%%%%%%%%%%%%%%%%%%%%%%%%%%%%%%%%%%%%%%%%%%%%%%%%%%%%%%%%%%%%%%%%%%%%

\subsection{Variável}
\begin{frame}[fragile]   %%%% indica que o ambiente  FRAME é frágil
\frametitle{Variável}

\end{frame}

%%%%%%%%%%%%%%%%%%%%%%%%%%%%%%%%%%%%%%%%%%%%%%%%%%%%%%%%%%%%%%%%%%%%%%%%

\subsection{Átomo}
\begin{frame}[fragile]   %%%% indica que o ambiente  FRAME é frágil
\frametitle{Átomo}

\end{frame}

%%%%%%%%%%%%%%%%%%%%%%%%%%%%%%%%%%%%%%%%%%%%%%%%%%%%%%%%%%%%%%%%%%%%%%%%

\subsection{Número}
\begin{frame}[fragile]   %%%% indica que o ambiente  FRAME é frágil
\frametitle{Número}

\end{frame}

%%%%%%%%%%%%%%%%%%%%%%%%%%%%%%%%%%%%%%%%%%%%%%%%%%%%%%%%%%%%%%%%%%%%%%%%

\subsection{Termo Composto}
\begin{frame}[fragile]   %%%% indica que o ambiente  FRAME é frágil
\frametitle{Termo Composto}

\end{frame}

%%%%%%%%%%%%%%%%%%%%%%%%%%%%%%%%%%%%%%%%%%%%%%%%%%%%%%%%%%%%%%%%%%%%%%%%

\section{Exemplos}
\begin{frame}[fragile]   %%%% indica que o ambiente  FRAME é frágil
\frametitle{Exemplos}

\end{frame}


\section{Motivações}
\begin{frame}[fragile]
\frametitle{Motivações}


1. Qual característica do P.I.C.A.T é mais chamativa?
R: multi-paradigma ...(fora objetos), leve  como Python

Em quais aplicações você usaria P.I.C.A.T?
R: ensino de uma primeira linguagem 

Quais são os pontos positivos e negativos do P.I.C.A.T que você identifica?
R: negativo .... geração de codigo (ainda eh byte-code) e interface com outras linguagens (ainda incipiente)

Se pudesse melhorar algo no P.I.C.A.T, o que melhoraria?
R: a interface com outras linguagens 

O P.I.C.A.T pode substituir alguma linguagem?
R: Prolog sim ...

\end{frame}




\end{document}
