%\documentclass{beamer}

\documentclass[12pt,a4paper]{article}

\usepackage{graphicx,hyperref,url}
\usepackage[utf8]{inputenc}
\usepackage[T1]{fontenc}
\usepackage[portuges,brazilian]{babel}
%%%\usepackage{wrapfig}
\usepackage{caption}
\usepackage{subfigure}
%\usepackage{subcaption}
\usepackage{latexsym}
\usepackage{amssymb, amsmath}
\usepackage{multicol}
\usepackage{pifont}%,bbding}%%,dingbat} %%% ver manual de simbolos
\usepackage[final]{listings}
\usepackage{comment, color}


\definecolor{azulclaro}{rgb}{0.9,0.9,0.9}
\definecolor{mygreen}{rgb}{0,0.6,0}
\definecolor{mygray}{rgb}{0.5,0.5,0.5}
\definecolor{mymauve}{rgb}{0.58,0,0.82}
\definecolor{darkgray}{rgb}{.4,.4,.4}
\definecolor{purple}{rgb}{0.65, 0.12, 0.82}


\begin{document}

\begin{center}
{\LARGE  Roteiro de Laboratório - PICAT}\\
Antes de tudo, verifique se:

\end{center}



\section{Atenção}
\begin{enumerate}
\item Tens em mãos o manual de comandos Linux
\item Anotado as notas de aula de professor
\item \textbf{\textcolor{red}{Comece com os exercícios feitos em sala de aula}}
\end{enumerate}


\section{Exercícios}

\begin{enumerate}
\item Construa a árvore geneológica de sua família, tendo as relações (predicados): filho e pai;

\item A partir do programa anterior, construa as relações: irmão, avô e ancestral. Um ancestral generaliza os conceitos de pai, avô, bisavô, etc;

\item Quais dos predicados abaixo casam, e se a unificação ocorrer, quais são os resutados. Escreva os seus significados, tomando por base os conceitos de termos ou átomos, ``matching''\/ (``casamento''\/), e unificação:
\begin{itemize}
\item Picat> 2 > 3.
\item Picat>  >(2, 3).
\item Picat>  2 == 2.
\item  Picat>   a(1, 2) = a(X, X). 
\item  Picat>   a(X, 3) = a(4, Y). 
\item  Picat>   a(a(3, X)) = a(Y).  /* deixar para depois */
\item  Picat>  1+2 = 3. 
\item  Picat>   X = 1+2. 
\item  Picat>   a(X, Y) = a(1, X). 
\item  Picat>   a(X, 2) = a(1, X). 
\item  Picat> 1+2 = 3
\item  Picat> X + 2 = 3 * Y.
\item  Picat> X+Y = 1+2.
\item  Picat> 1+Y = X + 3.
\item  Picat> lectures(X, ai) = lectures(alison, Y).
\item  Picat> X+Y = 1+5, Z = X.
\item  Picat> X+Y = 1+5, X=Y.
\end{itemize}

\item Seja o programa abaixo:
\begin{verbatim}
       b(1).                             
       b(z).                            
       d(3).
       d(11).
       c(3).
       c(z).
       a(X) => b(X),  c(X),  false.
       a(X) =>  c(X),  d(X).
\end{verbatim}
Encontre os valores para\/ ``{\em a(Y)}''\/ e explique como foi o esquema de busca. Onde e porquê ocorreu
\/ ``{\it backtraking}''?


\item Repasse TODOS exercícios apresentados em LPO, na sala, e 
reescreva-os em PICAT. \textcolor{red}{Não avance aos próximos exercícios sem fazer \textbf{todos} exercícios acima.}
 



\item Resolva por recursão os seguintes problemas:
\begin{itemize}

\item Cálculo do fatorial (Obs: fat(0) é 1, fat(N) é fat(N-1) * N);

\item Soma intervalar a partir de um valor\/ ``{\em n1}''\/ até\/ ``{\em n2}''\/  tal que\/ ``{\em n2 > n1}''\/.

\item Seja F a versão recursiva da função de Fibonacci.  A função de Fibonacci é definida assim:  
 \texttt{F(0) = 0}, \texttt{ F(1) = 1}   e   \texttt{F(n) = F(n-1) + F(n-2)}   para 
 \texttt{n > 1}.  O cálculo do valor da expressão \texttt{F(3)} provocará a seguinte sequência de invocações da função:
\begin{verbatim}
    F(3)
     F(2)
       F(1)
       F(0)
     F(1)
 \end{verbatim}
  Qual a sequência de invocações da função durante o cálculo de \texttt{F(5)}?
  Implemente-a em PICAT.


\item Implemente em PICAT um programa que imprima um retângulo de (X,Y) de asteriscos,
onde X é o número de linhas e Y é o número de colunas.


\item Implemente em PICAT um programa que imprima um triângulo de X de asteriscos, na primeira linha,
X-1 na linha seguinte, e assim sucessivamente, até 1 asterisco na última linha.

\item Reescreva o problema acima, mas com o triângulo 
invertido.


\item  Implemente em PICAT um programa que imprima um triângulo de X de asteriscos, na primeira linha,
contudo, este deverá ter um formato de um triângulo
equilátero.

\item Reescreva o problema acima, mas com o triângulo 
equilátero invertido.


\item Write a program in Picat to print first 50 natural numbers using recursion.

\item Write a program in Picat to calculate the sum of numbers from 1 to n using recursion.

\item  Write a program in Picat to Print Fibonacci Series using recursion. 
\item Write a program in Picat to print even or odd numbers in given range using recursion. Example:
\begin{verbatim}
Test Data : 
Input the range to print starting from 1 : 10 
Expected Output :

All even numbers from 1 to 10 are : 2  4  6  8  10                              
All odd numbers from 1 to 10 are : 1  3  5  7  9   

\end{verbatim}

\item The digital root of a non-negative integer $n$ is computed as follows. Begin by summing the digits of $n$. The digits of the resulting number are then summed, and this process is continued until a single-digit number is obtained. For example, the digital root of 2019 is 3 because \texttt{2+0+1+9=12} and \texttt{1+2=3}. Write a recursive function digitalRoot(n) which returns the digital root of $n$.\\
Assume that a working definition of digitalSum will be provided for your program.



\item The hailstone sequence starting at a positive integer n is generated by following two simple rules. If n is even, the next number in the sequence is $n/2$. If n is odd, the next number in the sequence is $3*n+1$. Repeating this process, we generate the hailstone sequence. Write a recursive function hailstone(n) which prints the hailstone sequence beginning at n. Stop when the sequence reaches the number 1 (since otherwise, we would loop forever \texttt{1, 4, 2, 1, 4, 2, ...}) 
For example, when \texttt{n=5}, your program should output the following sequence:
\begin{verbatim}
5
16
8
4
2
1

\end{verbatim}

%\item 
%\item 

\end{itemize}

\end{enumerate}
\end{document}
