
\documentclass[12pt, a4paper,final]{article}
\usepackage{t1enc}
\usepackage[latin1]{inputenc}
\usepackage[portuges]{babel}
\usepackage{amsmath}
\usepackage{amsfonts}
\usepackage{amssymb}

%\usepackage{graphicx}
\topmargin       -1cm
 \headheight      17pt
 \headsep  1cm

\textheight      24cm

\textwidth       16.3cm

\oddsidemargin   2mm

\evensidemargin  2mm

\pagestyle{empty}

\begin{document}
\begin{center}
\framebox[\textwidth][c]{L�gica e Programa��o em
L�gica  (LPL) - 1a. Parte Pr�tica}
\newline

\end{center}

\vskip1cm Alunos (at� 03): \hrulefill
%%%\noindent

\begin{enumerate}
\item  Construa a �rvore geneol�gica de sua fam�lia, tendo as rela��es
(predicados): filho e pai;

\item  A partir do programa anterior, construa as
rela��es: irm�o, av�,  bisav�, tio,  etc;


\item ($\ast $)  Quais dos predicados abaixo que
casam, e se a unifica��o ocorrer, quais s�o os
seus resultados? Qual a sua interpreta��o dos
mesmos? Comente:

\begin{itemize}
\setlength{\itemsep}{-5pt}
 \item  ?- 2 > 3. \item
?- >(2, 3).

\item  ?- 2 == 2. /* experimente com letras */

 \item ?- 2=2. /* experimente com letras */

 \item ?- 2 is 2. /* experimente com letras */

 \item  ?- a(1, 2) = a(X,
X).

\item  ?- a(X, 3) = a(4, Y).

\item  ?- a(a(3, X)) = a(Y). /* deixe para depois
caso n�o entenderes */

\item  ?- 1+2 = 3.

\item  ?- X = 1+2.

\item?-a(X, Y) = a(1, X).

\item  ?- a(X, 2) = a(1, X).

\item  ?- 1+2 = 3

\item  ?- X + 2 = 3 * Y.

\item  ?- X+Y = 1+2.

\item  ?- 1+Y = X + 3.

\item  ?- lectures(X, ai) = lectures(alison, Y).

\item  ?- X+Y = 1+5, Z = X.

\item  ?- X+Y = 1+5, X=Y.
\end{itemize}

\item ($\ast $)  Seja o programa abaixo:
\begin{verbatim}
b(1).
 b(2).
  d(3).
   d(4).
    c(5).
     c(6).
 a(W) :-  b(X),  c(Y), W is(X+Y), fail.
a(W) :-  c(X),  d(Y), W is (X+Y).
\end{verbatim}

Encontre os valores para ``\emph{a(Y)}'' e
explique como foi o esquema de busca. Onde e
porqu� ocorreu ''\textit{backtraking}''?

\end{enumerate}

$\ast $ Observa��o: Apenas os 2 (dois) �ltimos
exerc�cios s�o para entrega {\bf em papel}.
Rascunho, ou a m�o, mas em {\bf papel} !

\end{document}
