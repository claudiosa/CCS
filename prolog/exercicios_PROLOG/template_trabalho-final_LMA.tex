\documentclass[a4paper,12pt]{article}
\usepackage[utf8]{inputenc}
\usepackage[brazilian]{babel}
\usepackage[T1]{fontenc}
\usepackage[a4paper,left=30mm,right=30mm,top=15mm,bottom=15mm]{geometry}
\usepackage{graphicx,url}
\usepackage{color,comment, pifont}
\usepackage{amssymb}

\begin{comment}
\usepackage[T1]{fontenc}
\usepackage[utf8]{inputenc}
\usepackage{lmodern}
\usepackage[francais]{babel}
\end{comment}

\title{Lógica Matemática -- Trabalho Final -- 2014-2}
\author{Rogério Eduardo da Silva, Claudio Cesar de Sá\\ Gabriela Nandi}
\date{\today}

\graphicspath{{/figures/}}   
\DeclareGraphicsExtensions{{.jpg},{.png}}


\begin{document}
\maketitle

\begin{flushleft}

\ding{224} Tarefa: Implementem os  02 (dois)  problemas dos problemas propostos abaixo.\\
Entrega: definir data e forma com Prof. Rogério \\ 

\ding{224} Implementação em SWI-Prolog ou Eclipse (www.eclipseclp.org) (ver a apostila do curso)\\

\ding{224} Além dos códigos, sob forma de cometários as entradas e saídas com os testes de seus programas\\

\ding{224} Os testes exaustivos no próprio código fonte vão demonstrar que seu programa está fazendo o que se solicita.


\ding{224} Alguns fontes e materiais de apoio estão em:
 \url{https://www.dropbox.com/home/cursos/lma/exercicios_prolog} ou \\
  \url{http://www2.joinville.udesc.br/~coca/index.php/Main/LogicaMatematica}

\end{flushleft}

\begin{enumerate}
\setlength\itemsep{0.5cm}
\item Dicas de como se resolve manualmente:\\
\url{http://www.valdiraguilera.net/problema-de-logica-esquema.html}

\item Há exemplos bem detalhados do semestre passado\\

\item Use a lista da disciplina para as dúvidas ou procure
os professores \textbf{pessoalmente}\\

\item Para que o \textit{código de honra} (evitar cópias de trabalhos) seja mantido,
troquem os nomes dos personagens das estórias abaixo, por seus nomes e/ou de suas família/amigos etc. 

\end{enumerate}

\newpage
\tableofcontents



\newpage
\section{Teste de QI de Einstein}

 Fonte do problema proposto:
 \url{http://rachacuca.com.br/logica/problemas/teste-de-einstein/}
 (tem a montagem da tabela para irem entendendo e depurando o problema).\\

{\em Albert Einstein criou este teste de QI (raciocínio lógico) no século 
  passado e afirmou que 98\% da população mundial não é capaz de resolvê-lo}. 
Acho que Einstein não
faria tal afirmação, mas,  como o curso de LMA foi num bom nível este
semestre, acho que voces vão resolver este problema. Vamos lá!

Regras básicas para resolver o teste
\begin{enumerate}
\item Há 5 casas de diferentes cores;
\item Em cada casa mora uma pessoa de uma diferente nacionalidade;
\item Esses 5 proprietários bebem diferentes bebidas, fumam diferentes
      tipos de cigarros e têm um certo animal de estimação;
\item Nenhum deles têm o mesmo animal, fumam o mesmo cigarro ou bebem a mesma bebida.

\end{enumerate}


Sejam os fatos desta estória:
\begin{enumerate}
 %%  \item    \textbf{\textcolor{red}{Falta o enunciado dos FATOS}}
\item    O Norueguês vive na primeira casa.
\item    O Inglês vive na casa Vermelha.
\item    O Sueco tem Cachorros como animais de estimação.
\item    O Dinamarquês bebe Chá.
\item    A casa Verde fica do lado esquerdo da casa Branca.
\item    O homem que vive na casa Verde bebe Café.
\item    O homem que fuma Pall Mall cria Pássaros.
\item    O homem que vive na casa Amarela fuma Dunhill.
\item    O homem que vive na casa do meio bebe Leite.
\item    O homem que fuma Blends vive ao lado do que tem Gatos.
\item    O homem que cria Cavalos vive ao lado do que fuma Dunhill.
\item    O homem que fuma BlueMaster bebe Cerveja.
\item    O Alemão fuma Prince.
\item    O Norueguês vive ao lado da casa Azul.
\item    O homem que fuma Blends é vizinho do que bebe Água.
   
\end{enumerate}

Afinal, defina quem mora em cada casa, sua cor, nacionalidade, bebida predileta, 
marca de cigarro e o seu animal de estimação.

\newpage
  \section{Resoluções de Ano Novo}
  
  
   Fonte do problema proposto:
 \url{http://rachacuca.com.br/logica/problemas/resolucoes-de-ano-novo/} 
  (tem a montagem da tabela para irem entendendo e depurando o problema).

Embora o final do ano não tenha chegado, as férias estão quase ai. Tirando
um lado ``{\em nerd}'' da nossa área, eis um problema incrementado.
Descubra quais são as resoluções de ano novo (férias) de cinco amigas.

  
Sejam os fatos desta estória:
  
\begin{enumerate}
%%  \item    \textbf{\textcolor{red}{Falta o enunciado dos FATOS}}
%%  \item 
\item  A moça que quer viajar está ao lado de quem tem 24 anos.
\item   A Clarissa está em algum lugar à esquerda de quem tem 23 anos.
\item   A Vivian está exatamente à direita da mulher mais velha.
\item   A namorada do Otávio é a mulher mais nova do grupo.
\item   A moça que namora o Daniel está na última posição.
\item   A Fabiana está em algum lugar entre quem tem 20 anos e quem quer viajar, nessa ordem.
\item   Quem namora o Marcelo está ao lado da Vivian.
\item   A Vivian está em algum lugar à esquerda de quem pretende economizar dinheiro.
\item   Quem quer ler mais no ano novo está ao lado da Vivian.
\item   Quem gostaria de emagrecer está na segunda posição.
\item   A Thaís está ao lado da mulher que namora o Otávio.
\item   A moça de bolsa amarela está exatamente à esquerda da Ana.
\item   A mulher de 19 anos está ao lado da mulher de bolsa verde.
\item   Quem tem a bolsa branca está em algum lugar entre a moça de 20 
        anos e a de 26 anos, nessa ordem.
\item   A cor da bolsa da Ana é vermelha.
\item   A Thaís está ao lado da Fabiana.
\item   A mulher de 26 anos está exatamente à esquerda da mulher que namora o Alexandre.
\item   A mulher de bolsa verde está em algum lugar à direita da Thaís.

\end{enumerate}  



\end{document}
