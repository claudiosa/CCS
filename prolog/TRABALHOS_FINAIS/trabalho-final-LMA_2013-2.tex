\documentclass[a4paper,12pt]{article}
\usepackage[utf8]{inputenc}
\usepackage[brazilian]{babel}
\usepackage[T1]{fontenc}
\usepackage[a4paper,left=30mm,right=30mm,top=15mm,bottom=15mm]{geometry}
\usepackage{graphicx,url}
\usepackage{color,comment, pifont}
\usepackage{amssymb}


\begin{comment}
\usepackage[T1]{fontenc}
\usepackage[utf8]{inputenc}
\usepackage{lmodern}
\usepackage[francais]{babel}
\end{comment}

\title{Lógica Matemática -- Trabalho Final -- 2013-2}
\author{Rogério Eduardo da Silva e Claudio Cesar de Sá}
\date{\today}

\graphicspath{{/figures/}}   
\DeclareGraphicsExtensions{{.jpg},{.png}}


\begin{document}
\maketitle

\begin{flushleft}

\ding{224} Tarefa: Escolham 02 (dois) dos 3 problemas propostos abaixo.\\
Entrega: definir data e forma com Prof. Rogério \\ 

\ding{224} Implementação em SWI-Prolog ou Eclipse (www.eclipseclp.org) (ver a apostila do curso)\\

\ding{224} Além dos códigos, sob forma de cometários as entradas e saídas com os testes de seus programas\\

\ding{224} Os testes exaustivos no proprio código fonte vão demonstrar que seu programa está fazendo o que se solicita.

\end{flushleft}

\begin{enumerate}

\item Dicas de como se resolve manualmente:\\
\url{http://www.valdiraguilera.net/problema-de-logica-esquema.html}

\item Há exemplos bem detalhados do semestre passado\\

\item Use a lista da disciplina para as dúvidas ou procure
os professores \textbf{pessoalmente}\\

\item Para que o \textit{código de honra} (evitar cópias de trabalhos) seja mantido,
troquem os nomes dos personagens das estórias abaixo, por seus nomes e/ou de suas família/amigos etc. 

\end{enumerate}


%%%\tableofcontents


\begin{description}

\newpage
\item [Fila do Banco:] Fonte do problema proposto:   \url{http://rachacuca.com.br/logica/problemas/fila-do-banco/}(tem a montagem da tabela para irem entendendo e depurando o problema).\\
Cinco homens estão lado a lado aguardando na fila de um banco para pagar uma conta. Descubra qual a conta que cada um foi pagar.


Sejam os fatos desta estória:
\begin{enumerate}
 %%  \item    \textbf{\textcolor{red}{Falta o enunciado dos FATOS}}
   
  \item   Na quarta posição está o homem que gosta de Sinuca.
  \item Quem gosta de Futebol está na quinta posição.
  \item O homem que gosta de Basquete está na segunda posição.
  \item Quem gosta de Vôlei está de Vermelho.
  \item O dono do Peixe está exatamente à esquerda de quem pratica Natação.
  \item O homem de Branco é o dono do réptil.
  \item O Bombeiro está exatamente à direita do Samuel.
  \item O Veterinário está na quinta posição.
  \item O Pesquisador está na terceira posição.
  \item Douglas trabalha como Fotógrafo.
  \item José está na quinta posição.
    \item  O Pesquisador está exatamente à esquerda do Augusto.
    \item  Douglas tem um Cachorro.
    \item  Na quinta posição está o homem da conta de Telefone.
    \item  O homem de Branco vai pagar a conta de Água.
    \item  Na terceira posição está quem vai pagar a conta de Luz.
    \item  Ronaldo vai pagar a conta de Celular.
    \item  O homem de Verde tem um Pássaro.
    \item  O homem de Azul está em uma das pontas.
    \item  O homem que tem um Gato está exatamente à direita de que gosta de Sinuca.
    \item  O homem que tem um Pássaro está ao lado do que joga Sinuca.
   
\end{enumerate}


\newpage
  \item[Passeio no Zoológico:] Fonte do problema proposto:
  \url{http://rachacuca.com.br/logica/problemas/passeio-no-zoologico/} (tem a montagem da tabela para irem entendendo e depurando o problema).\\

Cinco amigas estão esperando na fila para entrar no zoológico. Descubra as preferências de cada uma delas.

%%{\Large    \textbf{\textcolor{red}{Falta o enunciado dos }}}
  
Sejam os fatos desta estória:
  
  \begin{enumerate}
%%  \item    \textbf{\textcolor{red}{Falta o enunciado dos FATOS}}
%%  \item 
  
  \item   A penúltima garota da fila gosta de Biologia.
  \item A menina de mochila Amarela está em algum lugar à esquerda da garota que gosta de Português.
  \item A menina que gosta de História está em algum lugar entre a Joana e a menina que gosta de Geografia, nesta ordem.
  \item A Pati está exatamente à esquerda da garota que quer ver a Girafa.
  \item A garota que que ver a Arara está ao lado da que quer ver o Leão.
  \item A menina da mochila Vermelha está em algum lugar à esquerda da que quer ver o Leão.
  \item A garota que quer ver o Macaco usa uma mochila Branca.
  \item A menina que trouxe uma Maçã está na quarta posição da fila.
  \item A garota que gosta de Geografia vai comer Salgadinho.
  \item A menina que trouxe Chocolate está na segunda posição.
  \item A garota que vai comer Sanduíche gosta de Português.
  \item A menina que gosta de suco de Morango está em uma das pontas da fila.

  \item A menina que gosta de suco de Limão está entre a garota que quer ver o Elefante e a que gosta de suco de Maracujá, nesta ordem.
  \item A garota da mochila Amarela gosta de tomar limonada.
  \item A menina que gosta de História gosta de suco de Laranja.
  \item A garota que prefere suco de Abacaxi está em uma das pontas da fila.
  \item A menina que gosta de suco de Morango está exatamente à direita de quem gosta de suco de Maracujá.
  \item A Renata está entre a menina que vai comer Banana e a Ana, nesta ordem.
  \item A Jéssica ocupa a segunda posição na fila.
  \item O suco preferido da Pati é o de Limão.
  \item A garota que gosta de Geografia está ao lado da de mochila Azul.
  \item A menina da mochila Vermelha está exatamente à direita da de mochila Verde.


  \end{enumerate}  



\newpage
  \item[Esquadrilha de Aviões:] Fonte do problema proposto:
  \url{http://rachacuca.com.br/logica/problemas/esquadrilha-de-avioes/} (tem a montagem da tabela para irem entendendo e depurando o problema).\\

Descubra a posição de cada um dos 5 aviões e suas anomalias, bem como seus pilotos e seus gostos pessoais.

Sejam os fatos desta estória:
\begin{enumerate}
  \item  O avião do Cel. Milton solta fumaça vermelha.
  \item O rádio transmissor do Ten. Walter está com problemas.
  \item O piloto do avião que solta fumaça verde adora pescar.
  \item O Major Rui joga futebol nos finais de semana.
  \item O avião que solta fumaça verde está imediatamente à direita do avião que solta fumaça branca.
  \item O piloto que bebe leite está com o altímetro desregulado.
  \item O piloto do avião que solta fumaça preta bebe cerveja.
  \item O praticante de natação pilota o avião que solta fumaça vermelha.
  
  \item  O Cap. Farfarelli está na ponta esquerda da formação.
  \item O piloto que bebe café voa ao lado do avião que está com pane no sistema hidráulico.
  \item O piloto que bebe cerveja voa ao lado do piloto que enfrenta problemas na bússola.
  \item O homem que pratica equitação gosta de beber chá.
  \item O Cap. Nascimento bebe somente água.
  \item O Cap. Farfarelli voa ao lado do avião que solta fumaça azul.
  \item Na formação, há um avião entre o que tem problema hidráulico e o com pane no altímetro.
  
 \end{enumerate}  
 
\end{description}
\end{document}
