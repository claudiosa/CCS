% -*- coding: utf-8 -*-

\documentclass[12pt,a4paper]{article}
\usepackage[utf8]{inputenc}
\usepackage[portuges,brazil]{babel}

\usepackage[pdftex]{graphicx}
\usepackage{amsmath}
\usepackage{amsfonts}
\usepackage{amssymb}
\usepackage[normalem]{ulem}
\usepackage{pifont, color, marvosym }

\graphicspath{{figuras/}}

\topmargin       -0.15cm
\headheight      0pt
\headsep         -0.5cm
\textheight      25cm
\textwidth       16.7cm
\oddsidemargin   0mm
\evensidemargin  0mm
\baselineskip     -13pt
\pagestyle{empty}

%\restylefloat{figure,table}
\begin{document}
%%%\framebox[\textwidth][c]{\normalsize
\begin{center}
\fbox{\fbox{
\parbox [c][2.0cm][c]{15cm} {
\centering {\Large ALP}\\
{\bf Exercícios com funções  (\today)}\\
\uwave{Nível Médio}\\
}
}}
\end{center}

%\vskip13pt Aluno(a): \rule{13cm}{0.4mm}
%\noindent%

%\section{Série de Taylor e outras séries -- antes de Cálculo III}

\begin{huge}
\begin{itemize}
  \item  \textcolor{red}{Faça TODOS exercícios abaixo usando funções.}

\item  \textcolor{red}{Revise todos exercícios do curso, e refaça-os  usando funções.}

\end{itemize}
\end{huge}

\begin{enumerate}

\item Crie um programa em C que tenha um vetor A com 
de 100 números inteiros quaisquer,
e um vetor B com 5 números, digamos 1, 2, 3,  4 e 5.
Seu programa deve dizer quantos números de 1 a 5 estão presentes
no vetor A.

\item Dado um vetor um vetor A de números inteiros quaisquer,
indique se o mesmo é um palíndromo ou não.


\item Crie um programa em C que peça 100 números, armazene eles em um vetor e diga qual elemento é o maior, e seu valor.




\item  Crie um programa em C que peça 100 números, armazene eles em um vetor e diga qual elemento é o menor, e seu valor.

\item  Crie um programa em C que peça 100 números, armazene eles em um vetor e diga qual elemento é o maior, qual é o menor e que seus valores.


\item  Crie um aplicativo em C que peça um número inicial ao usuário, uma razão e calcule os termos de uma P.A (Progressão Aritmética), armazenando esses valores em um vetor de tamanho 100.

\item  Crie um aplicativo em C que peça um número inicial ao usuário, uma razão e calcule os termos de uma P.G (Progressão Geométrica), armazenando esses valores em um vetor de tamanho 100.

\item  Escreve um programa que sorteio, aleatoriamente, 100 números e armazene estes em um vetor.
Em seguida, o usuário digita um número e seu programa em C deve acusar se o número digitado está no vetor ou não. Se estiver, diga a posição que está.

\item  Criando um tabuleiro de Jogo da Velha
Crie um tabuleiro de jogo da velha, usando uma matrizes de caracteres (char) 3x3, onde o usuário pede o número da linha (1 até 3) e o da coluna (1 até 3). A cada vez que o usuário entrar com esses dados, colocar um 'X' ou 'O' no local selecionado.


\item Como criar um programa que verifica se o número é palíndromo
Número palíndromo é aquele que, se lido de trás para frente e de frete para trás, é o mesmo.
Exemplos: 2112, 666, 2442 etc


\item Há vários exercícios sobre palíndromos e reversos... treine nestes temas para prova.
 
%\item
\end{enumerate}



%\vskip 13pt \noindent Considerações para avaliação: clareza, comentários sobre a
%resolução, isto  é: 0,5 pontos.
\begin{center}
\rule{15cm}{0.1cm}
\end{center}

\end{document}
