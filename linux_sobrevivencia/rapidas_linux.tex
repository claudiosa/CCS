\documentclass[final,a4paper]{article}
%%% iconv  --from-code=UTF-8 --to-code=ISO-8859-1 x.tex > x_latim.tex
%\usepackage{lucidabr}

\usepackage[T1]{fontenc}        % Permite digitar os acentos de forma normal
\usepackage[utf8]{inputenc} %% este caso o doc esteja em utf8
%% \usepackage[brazil]{babel}

\usepackage{mathtools}
%%% FOR SLIDES DIRECTly
\usepackage[ams]{pdfslide}
\usepackage{graphicx}
%%\overlay{fundo_amarelo.jpg}



%%%%%%%%%%%%%%%%%%%%%%%%%%%%%%%%%
%%\usepackage{amsmath, amsfonts ,amssymb}
%%%\usepackage{epsfig}   %float,
\usepackage[normalem]{ulem}
\usepackage{hyperref,url}
\usepackage{comment}
\usepackage{pifont}%,bbding}%%,dingbat} %%% ver manual de simbolos
\usepackage[final]{listings}

%%

\hypersetup{pdftex, colorlinks=true, linkcolor=blue, citecolor=blue, 
filecolor=blue, urlcolor=blue, pdftitle=1, pdfauthor=r4cnr, 
pdfsubject=, pdfkeywords=}
%%% COMMENT UP TO HERE .... 

%%\begin{comment}
\ifx\pdftexversion\undefined
\usepackage[dvips]{graphicx}
\else
\usepackage[pdftex]{graphicx}
\DeclareGraphicsRule{*}{mps}{*}{}
\fi
%%%\end{comment}


%\graphicspath{{./figuresdir1/}{./figuresdir2/}{./figuresdir3/}}
\graphicspath{{/home/ccs/Dropbox/figs_genericas/}{figuras/}}

%\restylefloat{figure,table}   %%%% obrigat#io... pelo [H]
\overlay{fundo_bege.jpg}
%%salmao.png}
\pagestyle{title}


%\color{black} 


\begin{document}
%\lstset{language=Prolog}
\lstloadlanguages{Prolog,Haskell,C++}.

%\vskip 3cm
\begin{tabular}{lr}
%\vskip 2cm & {\vskip 2cm}\\

\begin{minipage}[l]{0.5\textwidth}
\vskip 2cm  %% abaixar a figura....
%%%%%%%%%%% INCLUDING PDF FIGURES
%\includegraphics[scale = 0.5]{mobius_strip_small.pdf}
%\includegraphics[scale = 1.2]{logo2_coca_icone.jpg}
\includegraphics[height=0.8\textheight, width=0.8\textwidth]{varios_tugs.jpg}
%%\includegraphics[scale = 0.5, bb=0.0cm 6.0cm 3cm 3cm]{mobius_strip_small.pdf}
\end{minipage}
&
\begin{minipage}[r]{0.5\textwidth}

\vskip 1cm

\begin{flushleft}
 {\textsf {\Large \underline{Para Todos os Cursos}}} \\
\vskip 1cm
{\textsf {\Huge \underline{\textit{Fast}  Linux} } }\\
\vskip 1cm
{\textsf Grupo Colméia } \\ 
{\textsf \$ : linha de comando}
\end{flushleft}


\end{minipage}
\end{tabular}


\tableofcontents
\begin{comment}
\section{Professores do CoCA}
%%\textcolor{white}{xxxx}
\begin{description}

  \item[\ding{239}]  Claudio Cesar de Sá (\textsf{claudio@colmeia.udesc.br})
  \item[\ding{239}]  Cristiano Damiani  (\textsf{damiani@joinville.udesc.br})
  \item[\ding{239}]  Fernando Deeke Sasse (\textsf{fsasse@gmail.br} -- DMAT)
  \item[\ding{239}]  Rogério Eduardo da Silva (\textsf{dcc2res@joinville.udesc.br})
\end{description}



\section{Problemas de Alocação}

\begin{figure}[!htb]
\centering
\includegraphics[width=0.6\textwidth , height=0.5\textheight]{figuras/depositos.pdf}
%%%prolog/scale=0.47
%\label{fig_solutions_covering}
 \caption{Problemas de Abastecimento -- Alocação}
\end{figure}



\begin{figure}[!htb]
\centering
%%\includegraphics[width=0.6\textwidth , height=0.5\textheight]{figuras/trens.png}
\includegraphics[scale=0.9]{figuras/trens.pdf}
%%%prolog/scale=0.47
%\label{fig_solutions_covering}
 \caption{Problemas de escalonamento de trens/onibus}
\end{figure}



\end{comment}


\section{As Teclas}


\begin{description}


\item[\texttt{xterm} ou \texttt{lxterminal}:] aplicativo terminal (aqui faz tudo!)
\item[Cursor $+$ botão do meio:]  \textit{copy-paste} ao ter um texto marcado
\item[Alt $+$ Tab:] alterna entre as aplicações ativas

\item[$\uparrow $ e $\downarrow $:] repetem os comandos digitados neste terminal
\item[\texttt{\^}: ] é o \texttt{ENTER} o qual deve ser pressionado após os comandos
que se seguem
\end{description}


\section{Comandos Introdutórios}

\begin{description}


\item[\ding{248}] Vendo o diretório onde estou:
\begin{verbatim}
$ pwd
/home/udesc
\end{verbatim}


\pagebreak
\item[\ding{248}] Listando o conteúdo do diretório:
\begin{verbatim}
$ ls
append.txt      lab_inic.pl        rapidas_linux.out
aula-15-08a.pl  lab_recursao.c     rapidas_linux.pdf
aula-15-08.pl   lab_recursao.pl    rapidas_linux.tex
......................................................
$ 
TESTE ESTE
$ ls .*
\end{verbatim}


\pagebreak
\item[\ding{248}] Cria um diretório
\begin{verbatim}
$ mkdir seu_diretorio
$ ls    
TEM QUE APARECER seu_diretorio LAH
\end{verbatim}


\pagebreak
\item[\ding{248}] Entrando dentro de uma pasta/diretório:
\begin{verbatim}
$ cd seu_diretorio/    VAI para seu diretorio
$ cd ..                SOBE  um nivel acima
$ cd pgms_prolog/      VAI PARA BAIXO ou um dado diretorio
$ cd ~  ATEH RAIZ HOME
$ pwd
/home/udesc
$ 
\end{verbatim}



\pagebreak
\item[\ding{248}] Cria um arquivo e lista o conteúdo:
\begin{verbatim}
$ touch nome_arquivo.txt
$ ls -al *.txt
-rwxr-xr-x 1 udesc udesc 435 2011-08-29 15:34 append.txt
-rw-r--r-- 1 udesc udesc   0 2011-08-29 19:41 nome_arquivo.txt
\end{verbatim}


\pagebreak
\item[\ding{248}] Passos para os laboratórios da turma de ALP:


\begin{description}

 \setlength\itemsep{13pt}

  \item[\ding{242} \texttt{gterminal}:] é o terminal já mencionado
    \item[\ding{242} \texttt{mkdir SEU\_NOME \^}:] cria diretório
        \item[\ding{242} \texttt{cd SEU\_NOME \^}:] vai para o seu diretório
        \item[\ding{242} \texttt{touch programa.c \^}:] criou um arquivo chamado \texttt{programa.c}
        \item[\ding{242} \texttt{geany programa.c} \&  \^:] edita o \texttt{programa.c} com o \texttt{geany}
       \item[\ding{242} \texttt{g++ programa.c \^}:] compila o arquivo o \texttt{programa.c} 
    \item[\ding{242}  \texttt{ls a* \^}:] verifica se o programa executável foi gerado, é o \textbf{a.out}  
    \item[\ding{242}  \texttt{./a.out \^}:] executa o \textbf{a.out}  
    \item[\ding{242}  \texttt{ls \* \^}:] lista diretório corrente
       
\end{description}


\pagebreak
\item[\ding{248}] Erro recorrente da turma:



\begin{verbatim}
$ geany nome_arquivo.txt  ^ // processo PARADO
^Z
[1]+  Parado              //geany xxxx   PARADO
$

$ bg 1 ^                    // processo 1 PARADO
[1]+ geany nome_arquivo.txt & ^  // bg ATIVANDO-O

o correto eh:

$ geny nome_arquivo.txt    &  ^   
          
          // NAO ESQUECA O & comercial ao final 
          // ^ eh o ENTER ....

\end{verbatim}


\pagebreak
\item[\ding{248}] Listando processos na memória:
\begin{verbatim}
$ ps 
  PID TTY          TIME CMD
 4685 pts/2    00:00:00 bash
 4790 pts/2    00:00:00 ps

$ ps -aux | grep udesc
........


$ ps -aux | grep udesc | more
.....................
lista os processos por pagina
\end{verbatim}



\pagebreak
\item[\ding{248}] Processos na memória e seu estado:
\begin{verbatim}
$ gedit nome_arquivo.txt &
[1] 4801
$ ps aux | grep gedit
udesc     4809  3.1  0.4  58132 16668 pts/2    Sl   19:54   0:00 gedit nome_arquivo.txt
udesc     4814  0.0  0.0   3060   816 pts/2    S+   19:54   0:00 grep --color=auto gedit

$ gedit nome_arquivo.txt      // processo em modo parado
^Z
[1]+  Parado                  gedit nome_arquivo.txt    PARADO
$
$ bg 1
[1]+ gedit nome_arquivo.txt &   // posto em background

\end{verbatim}

\pagebreak
\item[\ding{248}] Apagar arquivo  e diretório:
\begin{verbatim}
$ touch x
$ rm x
$ mkdir cria_diretorio
$ rmdir cria_diretorio
\end{verbatim}


\pagebreak
\item[\ding{248}] Copiar um arquivo:
\begin{verbatim}
$ cp origem.txt destino.txt
$ cp casa.pdf /media/arch_linux/pgms_prolog/

\end{verbatim}

\pagebreak
\item[\ding{248}] Copiando recursivamente um diretório:
\begin{verbatim}
$ cp -R haskell/ /media/arch_linux/

\end{verbatim}

\pagebreak
\item[\ding{248}] Renomear um arquivo:
\begin{verbatim}
$ mv casa.pdf /media/arch_linux/pgms_prolog/
$ rename atual novo

\end{verbatim}




\pagebreak
\item[\ding{248}] Limpar tela:
\begin{verbatim}
clear
reset

\end{verbatim}


\pagebreak
\item[\ding{248}] Listar o conteúdo de um arquivo:
\begin{verbatim}
$ cat append.txt 
?- append([a,b,c],[d,e], X).
X = [a, b, c, d, e].
.............................................
.............................................
?- append([a,b,c], X , [a, b, c, d, e]).
--Mais--(32%)

\end{verbatim}


\pagebreak
\item[\ding{248}] Pesquisar um arquivo com um dado específico:
\begin{verbatim}
$ grep "Y" *.pl
aula-15-08a.pl:			    p(Y), 
aula-15-08a.pl:			    X \== Y, 
aula-15-08a.pl:			    Z is (X + Y) ,
........................................
udesc@matrizubuntu9:~/pgms_prolog$ 

\end{verbatim}

\pagebreak
\item[\ding{248}] Criar um link simbólico ou atalho (em geral se cria este atalho em \texttt{/usr/bin} ou \texttt{/usr/local/bin}). 

\begin{verbatim}
EXEMPLO:
$ ln -s caminho/minizinc minizinc
PERMISSAO DE EXECUCAO:
$ chmod +x minizinc  (em  /usr/bin)
caminho = onde foi instalado o ORIGINAL
\end{verbatim}

\emph{A soft link, or more common, a symlink, is link a shortcut to the targeted file or directory. So when is removed the original target stays present. This is the opposite of a hard link which is a reference to the target and so, if the hard link is removed, so is the target.}

\pagebreak
\item[\ding{248}] Remover um link simbólico ou atalho (não link físico)

\begin{verbatim}
APENAS para o SIMBOLICO ....
$ ln -s caminho/nota_Minizinc_IDE.txt TESTE.TXT
$ rm TESTE.TXT 
$ ls caminho/*.txt
$ nota_Minizinc_IDE.txt
\end{verbatim}

\pagebreak
\item[\ding{248}] Remover um link simbólico ou atalho (não link físico)
com segurança
\begin{verbatim}
$ unlink  link_simbolico_criado
\end{verbatim}
O original ficou intacto!

\end{description}




\newpage
\subsection{Contato:}
UDESC/CCT/DCC \\
Grupo de Hardware e Software Livre -- Colméia\\

\subsection{Sítio de Referência:}

\textsf{http://www.colmeia.udesc.br/}






\end{document}
