\documentclass{article}

\usepackage{fontspec}
\setmonofont{FreeMono}
\setsansfont{FreeSans}
\setmainfont{FreeSerif}

\usepackage{pdfslide}
\overlay{verdinho.pdf}

\usepackage{svg,pdftexcmds}
\makeatletter
\let\pdfstrcmp\pdf@strcmp
\let\pdffilemoddate\pdf@filemoddate

\begin{document}

\begin{tabular}{lr}
\begin{minipage}[l]{0.5\textwidth}
\vskip 2cm
%http://www.gnu.org/graphics/bwcartoon-m.svg
%\includesvg[width=0.85\textwidth]{bwcartoon-m}
%http://www.gnu.org/graphics/bwcartoon.html
\includegraphics[scale=0.7]{gnu-and-penguin-color-300x276.jpg}
\end{minipage}
&
\begin{minipage}[r]{0.5\textwidth}
{\Large \underline{Para Todos os Cursos}}\\
\vskip 1cm
{\Huge \underline{\textit{Fast} GNU}}\\
\vskip 1cm
Grupo Colméia \\
\$ : linha de comando
\end{minipage}
\end{tabular}
*gnome-terminal, cd (bash), pwd, ls, touch, grep, less, gedit, rm, mkdir, rmdir, cp, mv, cat, echo, chmod and ln are all gnu

\section{Teclas}
\begin{itemize}
\item[➸] S shift, C ctrl (\^{}), s super, M meta (alt)
\item[➸] <M-tab> (Alt+Tab) mudar aplicações
\item[➸] C-M-t (Ctrl+Alt+t) abrir gnome-terminal
\item[➸] C-d fim de input, encerrar interpretador
\item[➸] <tab> ... preencher comandos
\item[➸] Cursor $+$ botão do meio $=$ \textit{copy-paste}
\item[➸] 
\end{itemize}

\section{Gerenciar arquivos (substitui nautilus)}
➸ Vendo o diretório onde estou:
\begin{verbatim}
$ pwd
/home/udesc
\end{verbatim}

\pagebreak
➸ Listando o conteúdo do diretório:
\begin{verbatim}
$ ls -1
Desktop
Documents
Downloads
Music
Pictures
Public
Templates
Videos

TESTE ESTES
$ ls .*
$ ls -A
\end{verbatim}

\pagebreak
➸ Entrando em uma pasta/diretório:
\begin{verbatim}
$ cd ..          SOBE  um nivel acima
$ cd Downloads   PARA BAIXO
$ cd /           RAIZ
$ cd             HOME
\end{verbatim}

\pagebreak
➸ Apagar arquivo e diretório:
\begin{verbatim}
$ touch x
$ rm x
$ mkdir cria_diretorio
$ rmdir cria_diretorio
\end{verbatim}

\pagebreak
➸ Mover ou renomear arquivo:
\begin{verbatim}
$ mv atual novo
$ mv casa.pdf ~/Downloads
\end{verbatim}

\pagebreak
➸ Copiar um arquivo:
\begin{verbatim}
$ cp origem.txt destino.txt
$ cp casa.pdf ~/Downloads
\end{verbatim}

\pagebreak
➸ Copiando recursivamente um diretório:
\begin{verbatim}
$ cp -r haskell ~/Downloads
\end{verbatim}

\pagebreak
➸ Remover um link ou atalho
\begin{verbatim}
$ ln -s caminho/nota_Minizinc_IDE.txt TESTE.TXT
$ rm TESTE.TXT
$ ls caminho/*.txt
nota_Minizinc_IDE.txt
\end{verbatim}

\emph{WARNING: Link dereferenciation}
\begin{verbatim}
$ mkdir y
$ echo whatever >> y/a
$ cat y/a
whatever
$ ln -s y z
$ rm -rf z
$ cat y/a
whatever
$ ln -s y z
$ rm -rf z/
$ cat y/a
cat: y/a: No such file or directory
\end{verbatim}

\pagebreak
➸ Link simbolico (path) e link hard (conteudo)
\begin{verbatim}
$ echo whatever >> z
$ echo hell world >> y
$ ln -s z sym
$ ln z hard
$ rm -f z
$ cat sym
cat: sym: No such file or directory
$ cat hard
whatever
$ mv y z
$ cat sym
hell world
$ cat hard
whatever
\end{verbatim}

\section{Manipular arquivos texto}

➸ Cria um arquivo e lista o conteúdo:
\begin{verbatim}
$ touch nome_arquivo.txt
$ ls -l *.txt
-rwxr-xr-x 1 udesc udesc 435 2011-08-29 15:34 append.txt
-rw-r--r-- 1 udesc udesc   0 2011-08-29 19:41 nome_arquivo.txt
\end{verbatim}

\pagebreak
➸ Listar o conteúdo de um arquivo:
\begin{verbatim}
$ cat append.txt
?- append([a,b,c],[d,e], X).
X = [a, b, c, d, e].

?- append([a,b,c],[d,e], [a, b, c, d, e]).
true.
.............
$ less append.txt
?- append([a,b,c],[d,e], X).
X = [a, b, c, d, e].

?- append([a,b,c],[d,e], [a, b, c, d, e]).
true.

?- append([a,b,c], X , [a, b, c, d, e]).
\end{verbatim}

\pagebreak
➸ Pesquisar um arquivo com um dado específico:
\begin{verbatim}
$ grep "Y" *.pl
aula-15-08a.pl:			    p(Y),
aula-15-08a.pl:			    X \== Y,
aula-15-08a.pl:			    Z is (X + Y) ,
aula-15-08.pl:p(X,Y,Z) :- r(X), s(Y), t(Z).
aula-15-08.pl:x :- p(X,Y,Z), .....
lab_inic.pl:filho(X,Y) :- pai(Y, X).
lab_inic.pl:irmao(X,Y) :- pai(X,Z), pai(Y,Z),
lab_inic.pl:				X \== Y.
lab_inic.pl:avo(X, Y) :- pai(Z , X), pai(Y,Z).				
\end{verbatim}

\pagebreak
➸ Criar um comando sem permissões especiais:
\begin{verbatim}
$ cd
$ mkdir ~/bin
$ echo '#!/bin/bash
echo to write λ use C-S-u 3bb
echo does not display on xterm'>lambda
PERMISSAO DE EXECUCAO:
$ chmod +x lambda
CRIAR LINK SIMBOLICO:
$ ln -s ~/lambda ~/bin/lambda
FAZER O PROCESSO (bash) E FILHOS RECONHECEREM A PASTA:
$ export PATH=~/bin:$PATH
$ lambda
to write λ use C-S-u 3bb
does not display on xterm
\end{verbatim}

\section{Processos}
➸ Listando processos na memória:
\begin{verbatim}
$ ps
  PID TTY          TIME CMD
 4685 pts/2    00:00:00 bash
 4790 pts/2    00:00:00 ps

$ ps -aux | grep udesc
........
lista os processos

$ ps -aux | grep udesc | less
.....................
less é um visualizador de texto
\end{verbatim}
*less is more

\pagebreak
➸ Processos na memória e seu estado:
\begin{verbatim}
$ gedit nome_arquivo.txt &
[1] 4801
$ ps aux | grep gedit
udesc     4809  3.1  0.4  58132 16668 pts/2    Sl   19:54   0:00 gedit nome_arquivo.txt
udesc     4814  0.0  0.0   3060   816 pts/2    S+   19:54   0:00 grep --color=auto gedit

$ gedit nome_arquivo.txt
^Z
[1]+  Parado                  gedit nome_arquivo.txt    PARADO
$ bg
[1]+ gedit nome_arquivo.txt &
\end{verbatim}

\newpage
\subsection{Contato:}
UDESC/CCT/DCC \\
Grupo de Hardware e Software Livre -- Colméia\\

\subsection{Sítio de Referência:}
\textsf{http://www.colmeia.udesc.br/}

\end{document}
