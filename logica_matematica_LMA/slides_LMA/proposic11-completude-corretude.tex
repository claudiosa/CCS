\section{Completude e Corretude}
\begin{frame}[t]

\vskip 3cm

\begin{center}
{\Huge Teorema da Completude e da Corretude}
\end{center}

\end{frame}

\begin{frame}{Teorema da Completude e da Corretude}

\begin{itemize}

\item \textbf{Teorema da Completude}: seja $H$ uma fórmula da Lógica Proposicional. 
Se $H$ é uma tautologia então existe uma prova de $H$ por resolução.

A prova por resolução também é correta. Todo teorema é uma tautologia. 
Isto significa que dada uma fórmula $H$, se uma expansão por resolução associada 
a $\sim$H é fechada, então $H$ é uma tautologia. Os argumentos provados 
utilizando a resolução são válidos.


\item \textbf{Teorema da Corretude}: seja $H$ uma fórmula da Lógica Proposicional. 
Se existe uma prova de $H$, por resolução, então $H$ é uma tautologia.

\end{itemize}
\end{frame}
