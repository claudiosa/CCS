%% LyX 2.1.3 created this file.  For more info, see http://www.lyx.org/.
%% Do not edit unless you really know what you are doing.
\documentclass[brazil,english]{article}
\usepackage[T1]{fontenc}
\usepackage[latin9]{inputenc}
\usepackage{amsmath}

\makeatletter

%%%%%%%%%%%%%%%%%%%%%%%%%%%%%% LyX specific LaTeX commands.
%% Because html converters don't know tabularnewline
\providecommand{\tabularnewline}{\\}

\makeatother

\usepackage{babel}
\begin{document}
\selectlanguage{brazil}%

\section*{Sistema de Numera��o Bin�rio}

O sistema bin�rio utiliza-se de dois s�mbolos para representar um
n�mero qualquer. Tais s�mbolos s�o:

\begin{center}
$\varSigma_{b}=\{0,1\}$ ou $\varSigma_{b}=\{F,V\}$
\par\end{center}

Para um melhor entendimento, pode-se fazer um paralelo com o sistema
decimal, com o qual estamos mais habituados. Neste sistema, s�o utilizados
dez s�mbolos para a representa��o de um n�mero:

\[
\varSigma_{d}=\{0,1,2,3,4,5,6,7,8,9\}
\]


Assim, podemos representar, por exemplo, o n�mero $219$ em decimal.
O n�mero mais � direita, o $9$, representa a casa das unidades, enquanto
o $1$, a casa das dezenas, e o $2$, a casa das centenas, de modo
que:

\[
219=2\times10^{2}+1\times10^{1}+9\times10^{0}=200+10+9
\]


De forma semelhante, no sistema bin�rio:

\[
01011_{2}=0\times2^{4}+1\times2^{3}+0\times2^{2}+1\times2^{1}+1\times2^{0}=0+8+0+2+1=11_{10}
\]


Assim, $01011$ em bin�rio � o mesmo que $11$ em decimal. Vale notar
que o zero mais � esquerda pode ser omitido, pois ele n�o fornece
nenhuma contribui��o ao n�mero, ou seja, $01011_{2}=1011_{2}$, do
mesmo modo que $0219_{10}=219_{10}$.

A contagem no sistema bin�rio � realizada da mesma maneira que no
sistema decimal:

\begin{center}
\begin{tabular}{|c|c|}
\hline 
Bin�rio & Decimal\tabularnewline
\hline 
\hline 
0000 & 00\tabularnewline
\hline 
0001 & 01\tabularnewline
\hline 
0010 & 02\tabularnewline
\hline 
0011 & 03\tabularnewline
\hline 
0100 & 04\tabularnewline
\hline 
0101 & 05\tabularnewline
\hline 
0110 & 06\tabularnewline
\hline 
0111 & 07\tabularnewline
\hline 
1000 & 08\tabularnewline
\hline 
1001 & 09\tabularnewline
\hline 
1010 & 10\tabularnewline
\hline 
1011 & 11\tabularnewline
\hline 
1100 & 12\tabularnewline
\hline 
1101 & 13\tabularnewline
\hline 
1110 & 14\tabularnewline
\hline 
1111 & 15\tabularnewline
\hline 
\end{tabular}
\par\end{center}\selectlanguage{english}%

\end{document}
