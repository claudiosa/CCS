
\section{Planejamento}
\begin{frame}

\begin{center}
{\huge Capítulo 5 -- Planejamento}



\begin{enumerate}
  \item Contexto
  \item Definições
  \item Exemplos de Planejamento com o \textit{planner} do PICAT
\end{enumerate}

\end{center}

\end{frame}

%-----------------------------------------------------------


%-----------------------------------------------------------


\subsection{Contexto de Planejamento}
\begin{frame} [allowframebreaks=0.9]


    \frametitle{Contexto de Planejamento}
    \begin{itemize}
    
      \item Após a comunidade de SMA, os agentes terem \textit{cooperado} (decidem como não se    \textit{atrapalharem}), estes agentes entram na fase do \textit{planejar-executar}

%      \pause
      \item Área \textit{antiga} da IA, mas difícil de resolver
     
 %     \pause
      \item Área ativa ... com competições anuais entre os melhores \textit{planners}
             
  %   \pause
      \item NP-completo $+$ crescimento polinomial de espaço de memória $=$ classe PSPACE de problemas
      
   %   \pause
       \item Contudo, planejamento é 
       \textbf{semi-decidível}: o plano pode não existir ou não ser viável!
       
    %  \pause
      \item Planejamento $=$ escalonamento (sem o fator tempo), mas com crescimento de possibilidades no espaço de memória
      
    \end{itemize}
\end{frame}
%------------------------------------------------------
\subsection{Definições de Planejamento}
\begin{frame}

    \frametitle{Definições de Planejamento}
    \begin{itemize}
    
      \item Slides de apoio
       
       
    \end{itemize}
\end{frame}



\subsection{Abordagens ao Planejamento Multiagente -- SMAs}

\begin{frame}
\frametitle{Abordagens ao Planejamento de SMAs}

\begin{block}{}
 
\begin{itemize}
  \item Coordenação central: controla todos os subplanos
  \item Esquemas de controle distribuído\\
        Conhecimento parcial dos planos de outros agentes
  \item Planejamento Global Negociado

\begin{itemize}
  \item Compartilhamento de todos os planos
  \item Ajuste local para a realização de objetivos comuns

\end{itemize}

\item Modelagem Explícita da Equipe de Agentes
\begin{itemize}
  \item Compromissos conjuntos
   \item Crenças, desejos e intenções comuns

\end{itemize}
\end{itemize}
\end{block}

\end{frame}




%-----------------------------------------------------------
\subsection{Exemplos de Planejamento com PICAT}

\begin{frame}

    \frametitle{Exemplos de Planejamento com PICAT}

    \begin{itemize}
    
      \item Slides de apoio
       
       
    \end{itemize}
\end{frame}

%-----------------------------------------------------------
