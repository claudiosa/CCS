\documentclass[12pt]{article}

\usepackage[dvips]{color} %% incluido por Claudio
\usepackage{graphicx, subfigure, url}
\usepackage{amsmath,amsthm, amscd}
\usepackage[utf8]{inputenc} 
\usepackage[brazil]{babel}   

\usepackage{listings,comment}

\usepackage{colortbl}
\definecolor{olive}{rgb}{0,0.6,0}
\definecolor{brown}{rgb}{0.5,0.4,0}
\definecolor{orange}{rgb}{1,0.65,0.2}

\renewcommand{\lstlistingname}{Código}
\newenvironment{code}
{\begin{list}
{\setlength{\leftmargin}{1.5cm}
\setlength{\rightmargin}{1.5cm}
% \lstset{float = {hbtp},
\lstset{float = {H},
lineskip = -1pt,
captionpos = b,
numbers = left,
numberstyle = \tiny,
basicstyle = \scriptsize,
tabsize = 4}}
\item\scriptsize}
{\end{list}}

\newcommand{\nrainhas}{\emph{n-rainhas }}
\newcommand{\Nrainhas}{\emph{N-Rainhas }}
\newcommand{\PR}{Programação por Restrições}

\theoremstyle{definition}
\newtheorem{defn}{Definição}[section]
\hyphenation{si-mu-la-ted}

\sloppy

\begin{document}


{\huge \textcolor{red}{
A sequencia da apresentacao deve serguir
o seu tipo de trabalho
}}



\section{Implementação e Resultados}
\label{sec_implementacao}


{\Large \textcolor{red}{\underline {Aqui ... altere para o 
contexto de seu problema}}}

Falar do \textit{search} rapidamente ....
e enfatizar os doi pontos que serao explorados:

\begin{enumerate}

\item  Sel Var: Método de seleção das variáveis:\\
define a maneira em que as
variáveis serão selecionadas pelo solver para receberem um valor;



\item Atr Dom: Método de atribuição dos valores:\\
define a maneira em que
os valores dentro do domínio são escolhidos e atribuídos às variáveis durante o processo de exploração e busca;
\end{enumerate}


\begin{table}[!ht]

\caption{LEGENDA DA TABELA Eh sobre a tabela MESMO}
\label{tab01}
{\small
\begin{center}
\begin{tabular}{c||c|c|c} \hline\hline
$\frac{\textrm{Sel. Variável}}{\textrm{Atr. Dominio}}$  
& \texttt{first\_fail}
& \texttt{anti\_first\_fail}  & \texttt{occurrence} 
 \\ \hline\hline
\texttt{indomain\_min} & \ldots & \ldots & \ldots \\ \hline
\texttt{indomain\_max}  &  \ldots & \ldots  & \ldots \\ \hline
\texttt{indomain\_median}   & \ldots &  \ldots & \ldots \\ \hline
 \texttt{indomain\_split} &  \ldots & \ldots & \ldots \\ \hline
\texttt{indomain\_random} & \ldots & \ldots & \ldots \\ \hline
\texttt{indomain\_interval}   & \ldots & \ldots  & \ldots \\ \hline\hline

\end{tabular}
\end{center}
}

\end{table}





\begin{table}[!ht]

\caption{LEGENDA DE FIGURA é EMBAIXO ... aqui eh uma \textbf{tabela}}
\label{tab02}

{\small
\begin{center}
\begin{tabular}{c||c|c} \hline\hline
 $\frac{\textrm{Sel. Variável}}{\textrm{Atr. Dominio}}$ 
&  \texttt{most\_constrained}  & \texttt{max\_regret}
 \\ \hline\hline
\texttt{indomain\_min} &  \ldots & \ldots \\ \hline
\texttt{indomain\_max}  & \ldots  & \ldots \\ \hline
\texttt{indomain\_median}  &  \ldots & \ldots \\ \hline
 \texttt{indomain\_split} & \ldots & \ldots \\ \hline
\texttt{indomain\_random} & \ldots & \ldots \\ \hline
\texttt{indomain\_interval}  & \ldots  & \ldots \\ \hline\hline

\end{tabular}
\end{center}
}

\end{table}




\subsubsection{Solução em EcliPSE}



\subsubsection{Solução em PR}



{\Large \textcolor{red}{\underline {Aqui ... altere para o 
contexto de seu problema ECLIPSE}}}

{\Large \textcolor{red}{Aqui ... altere para o 
contexto de seu problema. Qual o objetivo do 
artigo mesmo? Retome o artigo aqui}}

O experimento consistiu em implementar .........................



 A tabela \ref{tab:Times} apresenta alguns destes valores quando executados em uma máquina padrão com 1.8 GHz de velocidade de CPU, 512 Kbytes de memória principal.


{\Large \textcolor{red}{\underline {Nao hah como 
fugir de uma tabela como esta que se segue}}}

{\Large \textcolor{red}{\underline { ESTA DEVE
SER MUITO, MAS MUITO MELHORADA}}}


\begin{table}[!ht]

\label{tab:Times}
\caption{Resultados}

\begin{center}
\begin{tabular}{c|c|c|c}
\hline \hline 
Tabuleiro & Outras TécnicasProLog & CP & Número de Combinações \\
\hline
\hline
$4 \times 4$ & $\cong$ 0.512 ms &  $\cong$ 2.979 ms & 2 \\
%% $5 \times 5$ & $< 1$ms & 16ms & 10 \\
$6 \times 6$ & $\cong$ 195.236 ms & $\cong$ 113.003 ms & 4 \\
%% $7 \times 7$ & 47ms & 46ms & 40 \\
$8 \times 8$ &  $\cong$ 1029.973 ms & $\cong$ 722.416 ms & 92 \\
%%$9 \times 9$ & 3.9s & 1.2s & 352 \\
$10 \times 10$ & $\cong$ 99.553 seg & $\cong$ 15.00 seg & 724 \\
$12 \times 12$ & $\geq$ 120 min &  $\cong$ 334.117 seg & 14200 \\
\hline \hline 
\end{tabular}

\end{center}

\end{table}


Embora tenha-se obtido tempos aceitáveis, a estatística dos tempos obtidos demonstram a complexidade exponencial deste problema. Para $N > 10$, este problema assume uma complexidade superior a $2^N$, logo, um problema NP-completo. Tratando-se de uma otimização, este é um clássico NP-difícil, do inglês, {\em NP-hard} \cite{sipser_1996}. Com tabuleiros de lado maior que $8$ unidades os tempos crescem por um fator exponencial. A partir de um tabuleiro de tamanho $7$, inicia as vantagens da utilização da CP comparada com a programação em lógica tradicional. Mesmo com instâncias pequenas, devido a natureza do problema, logo se começa a notar as diferenças de tempos de execução. Encontrar a melhor combinação não o fator mais complicado neste experimento, 
mas sim encontrar {\em todas} combinações. 


Em estratégias de buscas por melhoramentos como algoritmos genéticos, {\em simulated annealing}, são interessantes se o objetivo fosse encontrar uma única solução \cite{RusNorv}. Contudo, no problema aqui proposto,  a complexidade do controle em evitar as soluções duplicadas por estes métodos, deixaria de ser uma abordagem viável neste problema. Assim, a completude quanto as soluções
existentes de um dado problema, torna a CP como uma técnica atrativa.

\end{document}

