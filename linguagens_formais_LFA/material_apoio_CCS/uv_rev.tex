\documentclass[a4paper,12pt]{article}
\usepackage[utf8]{inputenc}
\usepackage{color}
\usepackage{amsmath,amssymb}
\usepackage{mathrsfs}

\title{LFA}
\author{Prova por Indução Matemática: Reverso}
\date{\today}
\begin{document}

\maketitle

\section{Exercício: $(uv)^r=v^ru^r$}

 Demonstre que $(uv)^r=v^ru^r$.\\
	$r:$ reverso ou invertido\\
	$\Lambda \equiv \epsilon \equiv  \curlywedge:$ vazio\\\\
	
	\vspace{1cm}
	BASE Indutiva:
	\begin{enumerate}
		
	
		\item Se $u=$ $\Lambda$ e $v=$ $\Lambda$ $\rightarrow$ k=0
		\item Se $u=a$ e $v=$ $\Lambda$ $\rightarrow$ k=1
		\item Se $u=$ $\Lambda$ e $v=a$ $\rightarrow$ k=1
		\item ...... 
		\item e $k=|uv|$\\ 
		
	
		
			
	\end{enumerate}
	
		\vspace{1cm}
	HIPÓTESE  Indutiva:
	\begin{enumerate}
	    \item $\mid u^rv^r \mid = n$
		\item Passo k: temos	$k=n \Rightarrow u^rv^r = (vu)^r$\\
		
	\end{enumerate}
	\vspace{1cm}	
			\textbf{$\Rightarrow$ Precisamos provar para $k=(n+1)$}. Por exemplo seja a palavra: $(aw)^r$ logo $\mid (aw)^r \mid = n+1$ (passo em $n+1$),
			 onde \underline{$w=uv$} então 
			$(auv)^r=v^ru^ra^r=v^ru^ra$ e $\mid u^rv^ra^r \mid = n+1$, ou ainda $\mid (auv)^r \mid = n+1$
			
	\vspace{1cm}	
	Passo (ou Prova)  Indutivo:
	\begin{enumerate}

		\item $(aw)^r= (auv)^r$  \hspace{2cm} assim, esta é a partida
		
		\item $(auv)^r=((au)v)^r$ \hspace{2cm} (1) Associatividade
		 
		 \item $((au)v)^r=v^r(au)^r$ \hspace{2cm}  (2) usando a Hipótese Indutiva
		 
		 \item $v^r(au)^r=v^r(u^ra^r)$ \hspace{2cm}  (3) novamente Hipótese Indutiva
		 
		 \item $v^r(u^ra^r)=v^ru^ra^r$ \hspace{2cm}  (4) Associatividade\\
		
		 
		 \item $(auv)^r=v^ru^ra^r=v^ru^ra$\hspace{2cm} 	 (5) C.Q.D.
		 
		 \item Esclarecendo: $a^r = a$ da definição do reverso, pois $a$ é símbolo do alfabeto. 
		 
		 \item Idem quanto $\curlywedge ^r = \curlywedge$

	
	
\end{enumerate}

\section{Notas:}
\begin{enumerate}

		 \item Esclarecendo: $a^r \equiv a$ da definição do reverso, pois $a$ é símbolo do alfabeto. 
		 
		 \item Idem quanto $\curlywedge ^r \equiv \curlywedge$
		 
		 \item Há uma outra solução desta prova no livro do Sudkamp, página 41.

		 \item Digitação inicial: Paula
	
	
\end{enumerate}





	
\end{document}
