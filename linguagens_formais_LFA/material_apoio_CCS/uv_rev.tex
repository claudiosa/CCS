\documentclass[a4paper,12pt]{article}
\usepackage[utf8]{inputenc}
\usepackage{color}
\usepackage{amsmath,amssymb}
\usepackage{mathrsfs}

\title{LFA}
\author{Prova por Indução Matemática: Reverso }
\date{\today}
\begin{document}

\maketitle

\section{Exercício}

 Demonstre que $(uv)^r=v^ru^r$.\\
	$r:$ reverso ou invertido\\
	$\Lambda=\epsilon = curly:$ vazio\\\\
	
	
	BASE Indutiva:
	\begin{enumerate}
		
	
		\item Se $u=$ $\Lambda$ e $v=$ $\Lambda$ $\rightarrow$ k=0
		\item Se $u=a$ e $v=$ $\Lambda$ $\rightarrow$ k=1
		\item Se $u=$ $\Lambda$ e $v=a$ $\rightarrow$ k=1
		\item ...... 
		\item e $k=|uv|$\\ 
		
	
		
			
	\end{enumerate}
	
	HIPÓTESE INDUTIVA:
	\begin{enumerate}
	    \item $\mid u^rv^r \mid = n$
		\item 	$k=n \Rightarrow u^rv^r = (vu)^r$\\
		

		\end{enumerate}
			\textbf{$\rightarrow$ Precisamos provar para $k=(n+1)$, por exemplo $(auv)^r=v^ru^ra^r=v^ru^ra$}  
	
	PROVA:
	\begin{enumerate}

		
		 \item$(auv)^r=((au)v)^r$ \hspace{2cm} (1) Associatividade
		 
		 \item$((au)v)^r=v^r(au)^r$ \hspace{2cm}  (2) usando a Hipótese Indutiva
		 
		 \item$v^r(au)^r=v^r(u^ra^r)$ \hspace{2cm}  (3) novamente Hipótese Indutiva
		 
		 \item$v^r(u^ra^r)=v^ru^ra^r$ \hspace{2cm}  (4) Associatividade\\
		
		 
		 \item$(auv)^r=v^ru^ra^r=v^ru^ra$\hspace{2cm} 	 (5) C.Q.D.

	
	
\end{enumerate}



	
\end{document}

