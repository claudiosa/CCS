\documentclass[a4paper,12pt]{article}
\usepackage[utf8]{inputenc}


\title{LFA}
\author{Prova por Indução Matemática: Reverso }
\date{1 de Março de 2018}
\begin{document}

\maketitle

\begin{enumerate}
	\item Demonstre que $(uv)^r=v^ru^r$.\\
	$r:$ reverso ou invertido\\
	$\Lambda=\epsilon:$ vazio\\\\
	
	
	BASE:
	\begin{itemize}
		
	
		\item Se $u=$ $\Lambda$ e $v=$ $\Lambda$ $\rightarrow$ k=0
		\item Se $u=a$ e $v=$ $\Lambda$ $\rightarrow$ k=1
		\item Se $u=$ $\Lambda$ e $v=a$ $\rightarrow$ k=1
		\item Se ...  \space\space\space\space\space\space\space\space\space\space\space $k=|uv|$\\ 
		
	
		
			
	\end{itemize}
	
	HIPÓTESE INDUTIVA:
	\begin{itemize}
		\item 	$k=n \rightarrow u^rv^r = (vu)^r$\\
		

		\end{itemize}
			\textbf{$\rightarrow$ Precisamos provar para k=(n+1) que $(auv)^r=v^ru^ra^r=v^ru^ra$}  \\\\
	
	PROVA:
	\begin{itemize}

		
		 \item$(auv)^r=((au)v)^r$
		 \space\space\space\space\space\space\space\space\space\space\space\space\space\space\space	 
		 (1) Associatividade
		 
		 \item$((au)v)^r=v^r(au)^r$
		 \space\space\space\space\space\space\space\space\space\space\space\space\space\space 
		 (2) Hipótese Indutiva
		 
		 \item$v^r(au)^r=v^r(u^ra^r)$
		 \space\space\space\space\space\space\space\space\space\space\space\space\space\space 
		 (3) Hipótese Indutiva
		 
		 \item$v^r(u^ra^r)=v^ru^ra^r$
		 \space\space\space\space\space\space\space\space\space\space\space\space\space\space\space 
		 (4) Associatividade\\
		
		 
		 \item$(auv)^r=v^ru^ra^r=v^ru^ra$
		  \space\space\space\space\space\space
		 (5) C.Q.D.
		 
		 
		 

		 
		 
			
\end{itemize}	
		

	
	
\end{enumerate}



\medskip


	
\end{document}

