
\documentclass[10pt]{beamer} 
%\usetheme{Warsaw}
\usetheme{CambridgeUS}

\usepackage{animate}

\usepackage{graphicx,hyperref,url}
\usepackage[utf8]{inputenc}
\usepackage[T1]{fontenc}
\usepackage[portuges,brazilian]{babel}
%%%\usepackage{wrapfig}
\usepackage{caption}
\usepackage{subfigure}
%\usepackage{subcaption}
\usepackage{latexsym}
\usepackage{amssymb, amsmath}
\usepackage{multicol}
\usepackage{pifont}%,bbding}%%,dingbat} %%% ver manual de simbolos
\usepackage[final]{listings}
\usepackage{comment}


\definecolor{azulclaro}{rgb}{0.9,0.9,0.9}
\definecolor{mygreen}{rgb}{0,0.6,0}
\definecolor{mygray}{rgb}{0.5,0.5,0.5}
\definecolor{mymauve}{rgb}{0.58,0,0.82}
\definecolor{darkgray}{rgb}{.4,.4,.4}
\definecolor{purple}{rgb}{0.65, 0.12, 0.82}

\newcommand{\minizinc}{MiniZinc}

\lstset{ 
  %  label={pgm_ex01},
    backgroundcolor=\color{azulclaro}, 
    language=erlang, %%Miranda,%%Perl,%%%Python, %%Mercury,
    showstringspaces=false,
    basicstyle=\bf\scriptsize\ttfamily,
%%      basicstyle= \footnotesize %%% TESTAR
%%      keywordstyle=\bfseries\color{green!40!black},
    keywordstyle=\textbf{\color{mygreen}}, 
    otherkeywords={*, \%, array, constraint, solve, output,  show, "/\", satisfy, set, of, if, then, elseif, float, search},
%%  keywordstyle=\color{blue},       % keyword style
%%    commentstyle=\itshape\color{purple!40!black},
      commentstyle=\color{orange},    % comment style
      identifierstyle=\color{blue},
      stringstyle=\color{orange},
      stringstyle=\color{mymauve},
      numbers=left,  % where to put the line-numbers; possible values are (none, left, right)
      numbersep=5pt,   % how far the line-numbers are from the code
      numberstyle=\tiny\color{magenta},
      keepspaces=true      
    % %caption={LEGENDA no source PASCAL ficou OK},
}


\graphicspath{{/home/ccs/Dropbox/figs_genericas/}{figuras/}{/home/ccs/Dropbox/CCS/picat}}
\DeclareGraphicsExtensions{.pdf,.png,.jpg}
%Global Background must be put in preamble
%\usebackgroundtemplate{\includegraphics[width=\paperwidth]{amarelinho.pdf}}
%%% \begin{frame}[allowframebreaks=0.8]

% The log drawn in the upper right corner.

%\logo{\centering
%\includegraphics[height=0.050\paperheight]{figuras/logo_SBPO_Peixe.png}
%%\hspace{9.6cm}
%\includegraphics[height=0.027\paperheight]{figuras/logo_udesc_horizontal.jpg}


%%%%%%%%%%%%%%%%%%%%%%%%%%%%%%%%%%%%%%%%%%%%%%%%%%%%%%%%%%%%%%%%%%%%%


\title[Picat]{\fontsize{20}{30}\selectfont \textcolor{black}{Resumindo e Conectando a Linguagens, Máquinas à Complexidade Computacional (em Figuras)}}

\author[]{Claudio Cesar de Sá\\
     {\small \url{claudio.sa@udesc.br}}}

\institute[UDESC]{
    Departamento de Ci\^encia da Computa\c{c}\~ao \\
    Centro de Ci\^encias e Tecnol\'ogias\\
   Universidade do Estado de Santa Catarina}

%%%%%%%%%%%%%%%%%%%%%%%%%%%%%%%%%%%%%%%%%%%%%%%%%%%%%%%%%%%%%%%%%%%%%

\begin{document}

\begin{frame}
    \titlepage
\end{frame}

%%%%%%%%%%%%%%%%%%%%%%%%%%%%%%%%%%%%%%%%%%%%%%%%%%%%%%%%%%%%%%%%%%%%%

%\begin{frame} [allowframebreaks=0.8]
%\frametitle{Sumário}
%\tableofcontents
%\end{frame}

%%%%%%%%%%%%%%%%%%%%%%%%%%%%%%%%%%%%%%%%%%%%%%%%%%%%%%%%%%%%%%%%%%%%%

\begin{frame}[fragile]

\frametitle{Contextualizando}

\begin{itemize}
	\item Algumas figuras vieram de vários autores \textit{by Google}
	\item Propositalmente, as mantive originais!
	\item Cabeçalho e sequência dos {\em slides} -- minha responsabilidade
	\item Comentários e ajuda: Cristiano Damiani
	\item Requisitos: finalizando um curso de LFA ou de TEC (preferenciamente)
	\item Esta apresentação disponível em \textbf{.tex} e \textbf{.pdf} em: \url{github.com/claudiosa/CCS/tree/master/linguagens_formais_LFA/extras_lfa_CCS}
	\item Programa de gravação utilizado: \url{http://www.maartenbaert.be/simplescreenrecorder/}
	\item Plataforma: Manjaro -- Linux
\end{itemize}

\end{frame}



%-----------------------------------------------------------------------------
\begin{frame}[fragile]

\frametitle{Resolvendo Problemas}

\begin{figure}[!ht]
	\centering
	\includegraphics[height =.65\textheight,width=\textwidth]
	{figuras/problem_implementation.pdf}
	\caption{O  quê o cientista busca?}
	%\label{ag_01}
\end{figure}



\end{frame}
%-----------------------------------------------------------------------------



%-----------------------------------------------------------------------------
\begin{frame}[fragile]

\frametitle{O \emph{Dilema} dos problemas  é:}

\begin{figure}[!ht]
\centering
\includegraphics[height =.65\textheight,width=.8\textwidth]
{figuras/dilema_eh.jpg}
%%%\caption{Agente situado versus a visão clássica de sistemas inteligentes}
%\label{ag_01}
\end{figure}

\end{frame}
%-----------------------------------------------------------------------------







%-----------------------------------------------------------------------------
\begin{frame}[fragile]

\frametitle{Problemas $\times$ Complexidade}
\begin{figure}[!ht]
\centering
\includegraphics[height =.65\textheight,width=.8\textwidth]
{figuras/os_problemas.png}
\caption{A área de Ciência da Computação (CC) se preocupa com os  dificeis (complicados) e os simples (os ingênuos) -- \emph{alguma estruturação!}}
%\label{ag_01}
\end{figure}

\end{frame}


%-----------------------------------------------------------------------------
\begin{frame}[fragile]

\frametitle{A área da CC se preocupa com:}
\begin{figure}[!ht]
	\centering
	\includegraphics[height =.65\textheight,width=.8\textwidth]
	{figuras/modelos_complexidade_computabilidade.png}
	\caption{Precisamos construir modelos e medir seu desempenho!}
	%\label{ag_01}
\end{figure}

\end{frame}


%-----------------------------------------------------------------------------




%-----------------------------------------------------------------------------


%-----------------------------------------------------------------------------
\begin{frame}[fragile]
\frametitle{Um escopo de formalismos da CC:}
\begin{figure}[!ht]
	\centering
	\includegraphics[height =.6\textheight,width=.9\textwidth]
%	{figuras/theoretical_computer_science.pdf}
		{figuras/theoretical_computer_science02.pdf}
\caption{Basicamente um guia desta apresentação}
	%\label{ag_01}
\end{figure}

\end{frame}




%-----------------------------------------------------------------------------










%-----------------------------------------------------------------------------


\begin{frame}[fragile]
\frametitle{LFA -- Linguagens  Formais}
\begin{figure}[!ht]
	\centering
	\includegraphics[height =.6\textheight,width=.5\textwidth]
	{figuras/Chomsky_Venn.jpg}
	%\caption{}
	%\label{ag_01}
\end{figure}

\end{frame}


\begin{frame}[fragile]
\frametitle{Máquinas que calculam estas linguagens: }
\begin{figure}[!ht]
	\centering
	\includegraphics[height =.6\textheight,width=.6\textwidth]
	{figuras/circuitos_a_MT.png}
	%\caption{}
	%\label{ag_01}
\end{figure}

\end{frame}



\begin{frame}[fragile]
\frametitle{Automatos de Estados Finitos}
\begin{figure}[!ht]
	\centering
	
\begin{minipage}[b]{0.475\linewidth}
	\centering
	%		\includegraphics[width=\textwidth]{a.png}
	\includegraphics[width=\textwidth, height =.4\textheight]
{figuras/afd_elementos.png}
		\caption{Uma unidade de controle com estados finitos}
	%		\label{fig:a}
\end{minipage}
\hspace{0.2cm}
\begin{minipage}[b]{0.475\linewidth}
	\centering
	\includegraphics[width=\textwidth, height =.4\textheight]{maq_moedas.jpg}
	\caption{Exemplo: maleiro e moedas}
	%	\label{fig:b}
\end{minipage}
\end{figure}


\end{frame}


\begin{frame}[fragile]
\frametitle{Exemplo}
\begin{figure}[!ht]
	\centering
	
\begin{minipage}[b]{0.475\linewidth}
	\centering
	%		\includegraphics[width=\textwidth]{a.png}
	\includegraphics[width=\textwidth, height =.48\textheight]
{figuras/ilustra_pastor.jpg}
		\caption{Problema do Pastor}
	%\label{fig:a}
\end{minipage}
\hspace{0.2cm}
\begin{minipage}[b]{0.475\linewidth}
	\centering
	\includegraphics[width=\textwidth, height =.6\textheight]{figuras/espaco_estados_PASTOR.pdf}
	\caption{Modelagem via AFD -- determinístico e finito: $\lessdot$ \underline{exponencial} $\gtrdot$}
	%	\label{fig:b}
\end{minipage}
\end{figure}


\end{frame}







\begin{frame}[fragile]
\frametitle{Automatos de Pilha}
\begin{figure}[!ht]
	\centering
	\includegraphics[height =.5\textheight,width=.5\textwidth]
	{figuras/ap_elementos.png}
	%\caption{}
	%\label{ag_01}
\end{figure}

\end{frame}





%-----------------------------------------------------------------------------

\begin{frame}[fragile]

%\frametitle{Precisamos Mensurar esta Complexidade -- Dificuldade}
\frametitle{Uma máquina \textit{ forte} e robusta: Máquina de Turing}
\begin{figure}[!ht]
\centering
\includegraphics[height =.65\textheight,width=.8\textwidth]	{figuras/old_TM.jpeg}
%%\caption{$\Sigma = \{1,2\}$ -- Partida $\models w = 11221 ... 12 $}
%\label{ag_01}
\end{figure}

\end{frame}



%-----------------------------------------------------------------------------


\begin{frame}[fragile]

%\frametitle{Precisamos Mensurar esta Complexidade -- Dificuldade}
\frametitle{ Máquina de Turing  e seus conceitos computacionais}
\begin{figure}[!ht]
	\centering
	
	\includegraphics[height =.65\textheight,width=.8\textwidth]
	{figuras/turing_machine_mecanica.jpg}
	\caption{Memória, entrada, saída, programa armazenado, escrita, etc.}
%	 \animategraphics[loop,controls,width=.8\linewidth]{10}{figuras/move/maquina_movimento_MT01-}{0}{570}
%%
	%\label{ag_01}
\end{figure}

\end{frame}


%-----------------------------------------------------------------------------


\begin{frame}[fragile]

%\frametitle{Precisamos Mensurar esta Complexidade -- Dificuldade}
\frametitle{ Máquina de Turing Universal : calcula tudo?}

\begin{minipage}[t]{0.475\linewidth}
 \begin{figure}[!ht]
	\centering
	\includegraphics[height =.6\textheight,width=.7\textwidth]
	{figuras/ex_calc_TM.png}
	\caption{Se calcula tudo, então vai saber se uma outra máquina vai {\bf parar} ou \textbf{não}  sob um dada entrada?}
	%%\label{ag_01}
\end{figure}
\end{minipage}
\begin{minipage}[t]{0.475\linewidth}
%\begin{block}
\begin{itemize}
\item Esta máquina hipotética tem suas limitações
%%há problemas que ela não calcula.
\item Por exemplo, dizer se ela  própria vai {\bf parar} ou não, quando sua
entrada for a sua própria especificação!
\item Este é o famoso problema da parada da Máquina de Turing ({\em the Halting Turing})
\item Assim os problemas são {\bf decidíveis} ou {\bf reconhecíveis} (as vezes não param)
\item Contudo, as MTs ajudaram a construir modelos  computacionais robustos e 
{\bf consistentes} para computação!      
\end{itemize}
%\end{block}

\end{minipage}


\end{frame}

%%%%%%%%%%%%%%%%%


\begin{frame}[fragile]

%\frametitle{Precisamos Mensurar esta Complexidade -- Dificuldade}
\frametitle{Máquina de Turing: Deterministica  x  Não-Determinística}
\begin{figure}[!ht]
	
	\begin{minipage}[b]{0.475\linewidth}
		\centering
%		\includegraphics[width=\textwidth]{a.png}
			\includegraphics[width=\textwidth, height =.4\textheight]
		{figuras/determ_non_determ.png}
%		\caption{Label for a}
%		\label{fig:a}
	\end{minipage}
	\hspace{0.2cm}
	\begin{minipage}[b]{0.475\linewidth}
		\centering
		\includegraphics[width=\textwidth, height =.4\textheight]{calculo_ND_MT.png}
	%	\caption{Label for b}
	%	\label{fig:b}
	\end{minipage}

	%	\caption{Memória, entrada, saída, programa armazenado, escrita, etc.}
	%	 %\label{ag_01}
\end{figure}

\begin{center}

Buscas   $\lessdot$ \underline{ exponenciais } $\gtrdot$ na fita da MT,  da ordem: $ \gtrapprox  2^{f(n)}$ 
\end{center}

\end{frame}



%-----------------------------------------------------------------------------


\begin{frame}[fragile]

\frametitle{\normalsize{Os algoritmos  consomem \textbf{tempo} e \textbf{memória} sob uma determinada máquina (modelo formal -- abstrato ou físico). }}

\begin{figure}[!ht]
	\centering
	\includegraphics[height =.64\textheight,width=.8\textwidth]
	{figuras/complexidade_algo.jpg}
	\caption{Complexidade dos algoritmos quanto ao \textbf{tempo}.\\Tempo \textbf{não} pode ser reusado, já memória: \textbf{sim}!}
	%\label{ag_01}
\end{figure}

\end{frame}

%%%%%%%%%%%%%%%%%%%%%%%%%%

\begin{frame}[fragile]

\frametitle{Máquinas que calculam sobre linguagens são resumidas em:}

\begin{figure}[!ht]
\centering
\includegraphics[height =.65\textheight,width=.85\textwidth]
{figuras/some+problems+have+no+algorithms_HALT.jpg}
\caption{RE: reconhecíveis $\approx $ linguagens finitas ou não, mas sem garantias da existência de um algortimo que \textbf{SEMPRE} pare!}
%\label{ag_01}
\end{figure}

\end{frame}


%-----------------------------------------------------------------------------
\begin{frame}[fragile]

\frametitle{Ampliando a visão anterior tem-se: linguagens, máquinas -- modelos abstratas  versus problemas resolvidos (total e parcial): }

\begin{figure}[!ht]
\centering
\includegraphics[height =.65\textheight,width=.8\textwidth]
{figuras/Extended+Chomsky+Hierarchy.jpg}
\caption{Modelos, formalismos, máquinas e complexidade!}
%\label{ag_01}
\end{figure}

\end{frame}
%-----------------------------------------------------------------------------

%-----------------------------------------------------------------------------


\begin{frame}[fragile]
\frametitle{Como se relacionam estas linguagens e/ou problemas?}
\begin{figure}[!ht]
	\centering
	\includegraphics[height =.7\textheight,width=.8\textwidth]
	{figuras/reducao01.jpg}
	\caption{Problema B se reduz a A ou $ B \le A$ (Reduções de Karp)}
	%\label{ag_01}
\end{figure}

\end{frame}


%-----------------------------------------------------------------------------
\begin{frame}[fragile]
\frametitle{A \emph{redução} é a \textbf{engrenagem} de se estabelecer propriedades, a existência e dificuldade de soluções entre problemas:}
\begin{figure}[!ht]
	\centering
	\includegraphics[height =.7\textheight,width=.8\textwidth]
	{figuras/reducao-classica.jpg}
	\caption{Problema B se reduz a A ou $ B \le A$}
	%\label{ag_01}
\end{figure}
\end{frame}
%-----------------------------------------------------------------------------


%-----------------------------------------------------------------------------
\begin{frame}[fragile]
\frametitle{O objetivo é tornar esta \emph{redução} (uma \textbf{engrenagem}) entre os problemas sem muita complexidade:}
\begin{figure}[!ht]
	\centering
	\includegraphics[height =.7\textheight,width=.8\textwidth]
	{figuras/reducao-via-mapeamento.jpg}
	\caption{Problema B se reduz a A ou $ B \le A$}
	%\label{ag_01}
\end{figure}
\end{frame}
%-----------------------------------------------------------------------------

%-----------------------------------------------------------------------------
\begin{frame}[fragile]
\frametitle{Exemplo de \emph{redução}:}
\begin{figure}[!ht]
	\centering
	\includegraphics[height =.5\textheight,width=\textwidth]
	{figuras/reducao-3cnf-clique.pdf}
	\caption{Há uma \emph{engrenagem} (uma redução específica) nesta reduçao!}
	%\label{ag_01}
\end{figure}
\end{frame}
%
%-----------------------------------------------------------------------------
\begin{frame}[fragile]
\frametitle{Muitos problemas podem ser reduzidos:}
\begin{figure}[!ht]
	\centering
	\includegraphics[height =.7\textheight,width=.6\textwidth]
	{figuras/reducao-problemas.jpg}
	\caption{Há uma \emph{engrenagem} (uma redução específica) nesta reduçao!}
	%\label{ag_01}
\end{figure}
\end{frame}
%


\begin{frame}[fragile]
\frametitle{Em geral, o cientista da CC está focado em reconhecer e resolver problemas de \emph{classes}:}
\begin{figure}[!ht]
	\centering
	\includegraphics[height =.7\textheight,width=.8\textwidth]
	{figuras/classes_de_problemas.png}
	%\caption{}
	%\label{ag_01}
\end{figure}

\end{frame}
%%%%%%%%%%%%%%%%%%%%%%%%%%%%%%%%%%%%%%%%%%%%%%%%%%%%%%%%%%%%%%%%%%%%%
\begin{frame}[fragile]
\frametitle{Referências}


\begin{figure}[!ht]
	
	\begin{minipage}[b]{0.475\linewidth}
		\centering
%		\includegraphics[width=\textwidth]{a.png}
			\includegraphics[width=0.6\textwidth, height =.6\textheight]{figuras/livro_sipser.jpg}
		\caption{3a. Edição: algumas provas claras!}
%		\label{fig:a}
	\end{minipage}
	\hspace{0.2cm}
	\begin{minipage}[b]{0.475\linewidth}
		\centering
		\includegraphics[width=0.6\textwidth, height =.6\textheight]{livro_NP_complete.jpg}
		\caption{Lista muitos problemas NP-completos. Conteúdo compilado da Wikipédia.}
	%	\label{fig:b}
	\end{minipage}

	%	\caption{Memória, entrada, saída, programa armazenado, escrita, etc.}
	%	 %\label{ag_01}
\end{figure}



\end{frame}

%%%%%%%%%%%%%%%%%%%%%%%%%%%%%%%%%%%%%%%%%%%%%%%%%%%%%%%%%%%%%%%%%%%%%

\begin{frame}[fragile]
\frametitle{Conclusões}
\begin{itemize}
    \item Tudo começou com alfabetos de símbolos, usados para  montar palavras e estas definem uma linguagem
    \item Linguagens devem ser computadas sob máquinas abstractas
    \item Linguagens $\Leftrightarrow$ problemas (se equivalem)
    \item Estas precisam computadas sob máquinas abstractas, as quais definem a complexidade em se computar um problema
    \item Há uma \emph{engrenagem} em se descobrir relações entre problemas: a \emph{redução}
    \item ..........................................
    
	\pause
	\item Área teórica da CC  é paradoxalmente  complicada--difícil, mas, bela e  sedutora!
\end{itemize}

\end{frame}
%%%%%%%%%%%%%%%%%%%%%%%%%%%%%%%%%%%%%%%%%%%%%%%%%%%%%%%%%%%%%%%%%%%%%

\begin{frame}[fragile]
\frametitle{Agradecimentos}
\begin{itemize}

    \item Obrigado 

	\item Comentários e dúvidas são muito bem-vindos

	\item Email:  claudio.sa@udesc.br 

	\item Canal do Youtube: Claudio Cesar de Sá -- ccs1664@gmail.com
	
\end{itemize}

\end{frame}


\end{document}
