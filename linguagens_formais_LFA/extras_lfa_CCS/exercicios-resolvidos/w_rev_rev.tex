\documentclass[a4paper,12pt]{article}
\usepackage[utf8]{inputenc}
\usepackage{color}
\usepackage{amsmath,amssymb}
\usepackage{mathrsfs}


\title{LFA}
\author{Prova por Indução Matemática: Reverso do Reverso}
\date{\today}
\begin{document}

\maketitle


\section{Exercício: $(w^r)^r=w$}

 Sabendo-se que $u^rv^r=(vu)^r$, a propriedade de associativida e definição de reverso, prove: $(w^r)^r=w$.
	
	\vspace{1cm}	
	\textbf{BASE Indutiva}:

\begin{enumerate}
		
	\item Define-se comprimento de $|w|$ por $|w| = n $ ou $k$
	
		\item Da definição de reverso: $k=0 \rightarrow \curlywedge^r = \curlywedge \rightarrow (\curlywedge^r)^r =\curlywedge $
		
	\item Idem $k=1 \Rightarrow a^r = a$ e $(a^r)^r = a$
	\item Logo $k=|w|$ on $k$ é equivalente ao $n$ passo 
			
	\end{enumerate}
	
	\vspace{1cm}
	\textbf{HIPÓTESE INDUTIVA}:
	
	\begin{enumerate}
		\item Em $k$ ou  $n$ tem-se $(w^r)^r=w $
		\item $k=0$  já foi definido anteriormente 
	
	\end{enumerate}

	\vspace{1cm}	
	\textbf{$\rightarrow$ Precisamos provar para $(k+1)$, então digamos uma palavra $aw$, logo $((aw)^r)^r = aw $}. \textcolor{red}{Não experimentei com $wa$ mas deve funcionar também, tarefa do aluno!}  

	
	\vspace{1cm}	
	Passo (ou Prova)  Indutivo:

	\begin{enumerate}

    \item $((aw)^r)^r$ \hspace{2cm} assim, esta é a partida
		
    \item$((aw)^r)^r = (w^ra^r)^r$ \hspace{2cm} aplicar o reverso na parte interna. Usar o teorema $(uv)^r=v^ru^r$, demonstrado anteriormente
		 
	 \item$(w^ra^r)^r = (xy)^r$ \hspace{2cm}  para fins de clclareza	 $x=w^r$ e $y=a^r$ na expressão anterior
		
	\item $(xy)^r = y^rx^r$ \hspace{2cm} aplicar o reverso, do teorema $(uv)^r=v^ru^r$, demonstrado anteriormente
		
				 
	 \item $ y^rx^r=(a^r)^r(w^r)^r$ \hspace{2cm} aplicando a Hipótese Indutiva, e substituindo os valores originais de $y$ e $x$
		 
	 \item $a(w^r)^r = aw$ \hspace{2cm} aplicando a Hipótese Indutiva mais uma vez na 2a. parte na expressão 
		
	\item $aw$ \hspace{2cm} C.Q.D.
		
\end{enumerate}

\textcolor{red}{Cada passo é realizado em relação ao anterior!}  
	


\section{Notas:}
\begin{enumerate}

		 \item Esclarecendo: $a^r \equiv a$ da definição do reverso, pois $a$ é símbolo do alfabeto. 
		 
		 \item Idem quanto $\curlywedge ^r \equiv \curlywedge$
		 
		 \item Digitação inicial: Paula
		 
	
\end{enumerate}



\end{document}
