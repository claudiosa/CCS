% -*- coding: utf-8 -*-

\documentclass[12pt,a4paper]{article}
\usepackage[utf8]{inputenc}
\usepackage[portuges,brazil]{babel}

\usepackage[pdftex]{graphicx}
\usepackage{amsmath}
\usepackage{amsfonts}
\usepackage{amssymb}
\usepackage[normalem]{ulem}
\usepackage{pifont, color, comment, marvosym }

\graphicspath{{figuras/}}

\topmargin       -0.15cm
\headheight      0pt
\headsep         -0.5cm
\textheight      25cm
\textwidth       16.7cm
\oddsidemargin   0mm
\evensidemargin  0mm
\baselineskip     -13pt
\pagestyle{empty}

%\restylefloat{figure,table}
\begin{document}
%%%\framebox[\textwidth][c]{\normalsize
\begin{center}
\fbox{\fbox{
\parbox [c][2.0cm][c]{15cm} {
\centering {\Large ALP}\\
{\bf Exercícios de laços de repetição  (\today)}\\
(Antes-- faça os da aula passada)\\
\uwave{Nível Avançado .... }\\
}
}}
\end{center}

%\vskip13pt Aluno(a): \rule{13cm}{0.4mm}
%\noindent

\section{Série de Taylor e outras séries -- antes de Cálculo III}

\begin{enumerate}

  \item Implemente a seguinte soma com  $n=7$ para um dado $x$
  $$e^x = \sum^\infty_{n=0} {x^n\over n!} =1 + x + {x^2 \over 2!} + {x^3 \over 3!} + {x^4 \over 4!}+\cdots.$$
  \begin{itemize}
    \item Veja são dois valores para ter precisão neste irracional $e^x$
    \item Para conferir o resultado na linguagem C++ este valor é dado pela função $exp$, veja o 
    exemplo abaixo:
       
\begin{verbatim}
  #include <stdio.h>      /* printf */
  #include <math.h>       /* exp  function e outras funcoes */

  int main ()
  {
  double param, result;   %%% veja os tipos ....
  param = 5.0;
  result = exp (param);
  printf ("The exponential value of %f is %f.\n", param, result );
  return 0;
    }
\end{verbatim}
    
      \end{itemize}


\item Casualmente $\cos x$ é dado por:

$$\cos x = 1 - {x^2 \over 2!} + {x^4 \over 4!} - \cdots$$

\item Casualmente $\sin x$ é dado por:

$$\sin x =  x - \frac{x^3}{3!} + \frac{x^5}{5!} - \cdots $$


\item Verifique esta série:

\begin{align}
\pi &\approxeq 768 \sqrt{2 - \sqrt{2 + \sqrt{2 + \sqrt{2 + \sqrt{2 + \sqrt{2 + \sqrt{2 + \sqrt{2 + \sqrt{2+1}}}}}}}}}\\
&\approxeq 3.141590463236763.
\end{align}


\item Verifique esta série:
$$\pi = \sum_{k = 0}^\infty \frac{1}{16^k}
\left( \frac{4}{8k + 1} - \frac{2}{8k + 4} - \frac{1}{8k + 5} - \frac{1}{8k + 6}\right).$$




%\item
\end{enumerate}

\section{Conferindo se voce aprendeu mesmo ....}

Considere os exemplos anteriores, tal que cada um dos valores da
série tem um número de termos $n$, ora $k$. Considere um \texttt{erro = 0.001}
o qual é a diferença entre o valor analítico e o valor da série.
Ou seja:\\


 $erro = \mid$ Valor da série - Valor analítico $\mid$\\


Assim, refaça todos exercícios anteriores, agora retornando valor $n$, ora $k$,
quando o erro for $erro \le 0.001$. Os valores analíticos voce considera
das funções matemáticas do C++, consulte o manual para uso destes.




%\vskip 13pt \noindent Considerações para avaliação: clareza, comentários sobre a
%resolução, isto  é: 0,5 pontos.
\begin{center}
\rule{15cm}{0.1cm}
\end{center}

\end{document}
