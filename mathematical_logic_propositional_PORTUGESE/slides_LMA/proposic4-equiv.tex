% ============================================================================

\subsection{Equivalência Lógica}

\begin{frame}[t]
\vskip 3.5cm
\begin{center}
{\Huge Equivalência Lógica}
\end{center}
\end{frame}

\begin{frame}[t]{Equivalência Lógica} % Título do Frame
	\begin{itemize}
	\item Diz-se que uma fórmula $P_{pqr\ldots}$ é logicamente equivalente a uma fórmula $Q_{pqr\ldots}$, se as tabelas-verdade dessas fórmulas forem idênticas.

	\item Tautologias são sempre equivalentes: $\blacksquare \Leftrightarrow \blacksquare$

	\item Contradições são sempre equivalentes: $\square \Leftrightarrow \square$

	\end{itemize}
\end{frame}



%----------------------------------------------------

\begin{frame}[t]{Equivalência Lógica} % Título do Frame
	
	\begin{block} {Propriedades da relação de equivalência:}
		
	\begin{description}
	\item [Propriedade Reflexiva]: $P_{pqr\ldots} \Leftrightarrow P_{pqr\ldots}$


	\item [Propriedade Simetria:] se $P_{pqr\ldots} \Leftrightarrow Q_{pqr\ldots}$ então
	$Q_{pqr\ldots} \Leftrightarrow P_{pqr\ldots}$
	

	\item [Propriedade Transitiva]: se $P_{pqr\ldots} \Leftrightarrow Q_{pqr\ldots}$ e $Q_{pqr\ldots} \Leftrightarrow R_{pqr\ldots}$ então $P_{pqr\ldots} \Leftrightarrow R_{pqr\ldots}$

	\end{description}
\end{block}	

\pause

\begin{block} {Observações e uso das propriedades:}

  \begin{itemize}
  
    \item A propriedade da simetria \textbf{\textcolor{red}{não é válida}} para relação da ``$\Rightarrow$''

    \item O uso dessas propriedades é para validar novas relações entre as fórmulas
    

  \end{itemize}
  
\end{block}	
	
\end{frame}

%----------------------------------------------------


\begin{frame}[t]{Equivalência Lógica} % Título do Frame
	Propriedades:
	\begin{description}
	\item [Dupla Negação:] $p \Leftrightarrow \sim\sim p$
	\end{description}
\end{frame}

\begin{frame}[t]{Equivalência Lógica} % Título do Frame
	Propriedades:
	\begin{description}
	\item [Dupla Negação:] $p \Leftrightarrow \sim\sim p$

	\item [Regra de CLAVIUS:] $(\sim p \rightarrow p) \Leftrightarrow p$
	\end{description}
\end{frame}
%-------------------------------------------------------------------------------------
\begin{frame}[t]{Equivalência Lógica} % Título do Frame
	Propriedades:
	\begin{description}
	\item [Dupla Negação:] $p \Leftrightarrow \sim\sim p$

	\item [Regra de CLAVIUS:] $(\sim p \rightarrow p) \Leftrightarrow p$

	\item [Regra da Absorção:] $(p \rightarrow p \wedge q) \Leftrightarrow (p \rightarrow q)$
	\end{description}
\end{frame}
%-------------------------------------------------------------------------------------
\begin{frame}[t]{Equivalência Lógica} % Título do Frame
	Propriedades:
	\begin{description}
	\item [Dupla Negação:] $p \Leftrightarrow \sim\sim p$

	\item [Regra de CLAVIUS:] $(\sim p \rightarrow p) \Leftrightarrow p$

	\item [Regra da Absorção:] $(p \rightarrow p \wedge q) \Leftrightarrow (p \rightarrow q)$

	\item $(p \rightarrow q) \Leftrightarrow \sim p \vee q$
	\end{description}
\end{frame}
%-------------------------------------------------------------------------------------
\begin{frame}[t]{Equivalência Lógica} % Título do Frame
	Propriedades:
	\begin{description}
	\item [Dupla Negação:] $p \Leftrightarrow \sim\sim p$

	\item [Regra de CLAVIUS:] $(\sim p \rightarrow p) \Leftrightarrow p$

	\item [Regra da Absorção:] $(p \rightarrow p \wedge q) \Leftrightarrow (p \rightarrow q)$

	\item $(p \rightarrow q) \Leftrightarrow \sim p \vee q$

	\item $(p \leftrightarrow q) \Leftrightarrow (p \rightarrow q) \wedge (q \rightarrow p)$
	\end{description}
\end{frame}
%-------------------------------------------------------------------------------------
\begin{frame}[t]{Equivalência Lógica} % Título do Frame
	Propriedades:
	\begin{description}
	\item [Dupla Negação:] $p \Leftrightarrow \sim\sim p$

	\item [Regra de CLAVIUS:] $(\sim p \rightarrow p) \Leftrightarrow p$

	\item [Regra da Absorção:] $(p \rightarrow p \wedge q) \Leftrightarrow (p \rightarrow q)$

	\item $(p \rightarrow q) \Leftrightarrow \sim p \vee q$

	\item $(p \leftrightarrow q) \Leftrightarrow (p \rightarrow q) \wedge (q \rightarrow p)$

	\item $(p \leftrightarrow q) \Leftrightarrow (\sim p \vee q) \wedge (\sim q \vee p)$
	\end{description}
\end{frame}

\begin{frame}[t]{Equivalência Lógica - Exercício} % Título do Frame
	\begin{center}
	Prove que: $(p \wedge \sim q \rightarrow \Box ) \Leftrightarrow (p \rightarrow q)$
	\end{center}
\end{frame}
%-------------------------------------------------------------------------------------
\begin{frame}[t]{Equivalência Lógica - Exercício} % Título do Frame
	\begin{center}
	Prove que: $(p \wedge \sim q \rightarrow \mathtt{\Box}) \Leftrightarrow (p \rightarrow q)$
	\end{center}

	\begin{center}
	\begin{scriptsize}
	\begin{tabular}{|c|c|c|c|c|c|c|}
	\hline
	$\mathbf{p}$ & $\mathbf{q}$ & $\mathbf{\sim q}$ & $\mathbf{p \wedge\sim q}$ & $\mathbf{p \wedge\sim q \rightarrow\mathtt{\Box}}$ & $\mathbf{p \rightarrow q}$ & $\mathbf{p \wedge\sim q \rightarrow\mathtt{\Box} \Leftrightarrow p \rightarrow q}$\\
	\hline
	V & V & F & F & V & V &  Iguais \\
	\hline
	V & F & V & V & F & F & Iguais \\
	\hline
	F & V & F & F & V & V & Iguais \\
	\hline
	F & F & V & F & V & V & Iguais \\
	\hline
	\end{tabular}
	\end{scriptsize}
	\end{center}
\end{frame}
%-------------------------------------------------------------------------------------
\begin{frame}[t]{Equivalência Lógica -- Observação} % Título do Frame

{\bf Repetindo:}

\begin{itemize}
\item  Relembrando, embora $\Leftrightarrow $ esteja na TV (página anterior),
ele não é um conectivo lógico, sim uma relação
lógica;

\item Relação de  $\Leftrightarrow $ indica uma verdade entre os lados
esquerdo e direito das fórmulas; 

\item  Se $f_1 \Leftrightarrow f_2 $ e estes são respeitados na
definição desta relação, então é uma relação verdadeira
 (poético, não?).
\end{itemize}

\end{frame}
%-------------------------------------------------------------------------------------
\begin{frame}[t]{Equivalência Lógica} % Título do Frame
	Dada a fórmula $p \rightarrow q$ alguns definições:

	\begin{description}
	\item[Fórmula Recíproca:] $q \rightarrow p$
	\item[Fórmula Contrária:] $\sim p \rightarrow \sim q$
	\item[Fórmula Contrapositiva:] $\sim q \rightarrow \sim p$
	\end{description}
\end{frame}
%-------------------------------------------------------------------------------------
\begin{frame}[t]{Equivalência Lógica} % Título do Frame
	\begin{center}
	\begin{tabular}{|c|c|c|c|c|c|}
	\hline
	$\mathbf{p}$ & $\mathbf{q}$ & $\mathbf{p \rightarrow q}$ & $\mathbf{q \rightarrow p}$ & $\mathbf{\sim p \rightarrow\sim q}$ & $\mathbf{\sim q \rightarrow \sim p}$ \\
	\hline
	V & V & V & V & V & V \\
	\hline
	V & F & F & V & V & F \\
	\hline
	F & V & V & F & F & V \\
	\hline
	F & F & V & V & V & V \\
	\hline
	\end{tabular}
	\end{center}

	Logo:

	\begin{itemize}
	\item $p \rightarrow q \Leftrightarrow \sim q \rightarrow \sim p$
	\item $q \rightarrow p \Leftrightarrow \sim p \rightarrow \sim q$
	\end{itemize}
\end{frame}
%-------------------------------------------------------------------------------------
\begin{frame}[t]{Equivalência Lógica - Exercícios} % Título do Frame
	\begin{enumerate}
	\item Determine a proposição contrapositiva
	\begin{enumerate}
	\item Se $X$ é menor que zero então $X$ não é positivo
	\item Se $X^2$ é ímpar então $X$ é ímpar
	\end{enumerate}

	\item Determine:
	\begin{enumerate}
	\item A contrapositiva da contrapositiva de $p \rightarrow q$
	\item A contrapositiva da recíproca de $p \rightarrow q$
	\item A contrapositiva da contrária de $p \rightarrow q$
	\end{enumerate}

	\item Determine:
	\begin{enumerate}
	\item A contrapositiva de $p \rightarrow \sim q$
	\item A contrapositiva de $\sim p \rightarrow q$
	\item A contrapositiva da recíproca de $p \rightarrow \sim q$
	\item A recíproca da contrapositiva de $\sim p \rightarrow \sim q$
	\end{enumerate}

	\end{enumerate}
\end{frame}
%-------------------------------------------------------------------------------------
\begin{frame}[t]{Equivalência Lógica - Conectivos de Scheffer} % Título do Frame
	
	
	\begin{description}
	\item [Negação conjunta $(\sim p \wedge \sim q)$] também indicada por $(p \downarrow q)$, 
	portanto $$(p \downarrow q) \Leftrightarrow  (\sim p \wedge \sim q)$$

	\item [Negação disjunta $(\sim p \vee \sim q)$] também indicada por $(p \uparrow q)$, 
	portanto $$(p \uparrow q) \Leftrightarrow (\sim p \vee \sim q)$$


  \pause
	\item [Mas ...] há outras equivalências para este conectivo de Sheffer:
	
	\begin{description}
	
	\item[Conectivo de Sheffer:] $ p \uparrow q \Leftrightarrow  \:\: \sim (p \wedge q) $

  \item[Conectivo de  Sheffer:] $ p \downarrow q \Leftrightarrow \:\: \sim (p \vee  q) $
  
  
	\end{description}	
	\end{description}
  \pause
 \textcolor{red}{Logo, muitas equivalências estão por vir!}


\end{frame}
%-------------------------------------------------------------------------------------
\begin{frame}[t]{Equivalência Lógica - Exercícios} % Título do Frame
	Demonstrar por tabelas-verdade as seguintes equivalências lógicas:
	\begin{enumerate}
	\item $p \wedge (p \vee q) \Leftrightarrow p$
	\item $q \leftrightarrow p \vee q \Leftrightarrow p \rightarrow q$
	\item $(p \rightarrow q) \vee (p \rightarrow r) \Leftrightarrow p \rightarrow q \vee r$
	\item $p \veebar q \Leftrightarrow (p \vee q) \wedge \sim (p \wedge q)$
	\item $(p \downarrow q) \uparrow (q \downarrow p) \Leftrightarrow q \vee p$
	\end{enumerate}
\end{frame}
