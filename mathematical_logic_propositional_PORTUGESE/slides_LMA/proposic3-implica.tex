% ============================================================================

\subsection{Implicação Lógica}

\begin{frame}[t]
\vskip 3.5cm
\begin{center}
{\Huge Implicação Lógica}
\end{center}
\end{frame}

\begin{frame}[t]%%%[allowframebreaks=0.9]

\frametitle{Implicação Lógica} % Título do Frame

\begin{block}{Definições:}
		\begin{itemize}
		
	\item Diz-se que uma fórmula $P_{pqr\ldots}$ {\em implica logicamente} uma outra fórmula $Q_{pqr\ldots}$, se simultaneamente estas 
	 nunca assumirem os valores lógicos ``V'' e ``F'', respectivamente, em suas tabelas verdades.
	 
	 \item Isto é uma definição: quando a fórmula  $P_{pqr\ldots}$ assumir $V$,
	 $Q_{pqr\ldots}$ nunca poderá ser $F$

	\item Representação: $$P_{pqr\ldots} \Rightarrow Q_{pqr\ldots}$$

	\item Toda fórmula (ora fórmula atômica ou proposição)  implica logicamente em 
	      uma tautologia: $P_{pqr\ldots} \Rightarrow\blacksquare$

	\item Só uma contradição implica logicamente em outra contradição: $\square \Rightarrow \square$
	
		
	\end{itemize}

	\end{block}
\end{frame}



\begin{frame}[t]%%%[allowframebreaks=0.9]

\frametitle{Implicação Lógica} % Título do Frame

\begin{block}{Definições:}
		\begin{itemize}
		
	
	\item \underline{Atenção}: o símbolo ``$\Rightarrow $'' define uma \textbf{relação lógica}    entre duas 
	fórmulas $P_{pqr\ldots}$ e $Q_{pqr\ldots}$

	\item Por outro lado,  o símbolo ``$\rightarrow $'' define uma \textbf{conectivo lógico} entre duas 
		fórmulas $P_{pqr\ldots}$ e $Q_{pqr\ldots}$. Aqui tem tabela-verdade, no caso acima não
		
		
	\item Em resumo: o ``$\Rightarrow $'' é uma  \textbf{relação	matemática}, enquanto o
	 ``$\rightarrow $'' é  um
	  \textbf{conectivo lógico} usado para construir fórmulas
	
	\end{itemize}

	\end{block}
\end{frame}

%--------------------------------------------------------------------




\begin{frame}[t]{Implicação Lógica} % Título do Frame



\begin{block}{Sua definição operacional é dada por:}

\begin{itemize}
\item Para verificar se duas fórmulas se relacionam logicamente entre si, isto é, $P_{pqr\ldots} \Rightarrow Q_{pqr\ldots}$, deve-se contruir as tabelas-verdade de $P_{pqr\ldots}$  e $Q_{pqr\ldots}$

\item Em toda linha de  avaliação de $P_{pqr\ldots}$ for \textbf{Verdade}, então naquela linha em $Q_{pqr\ldots}$ também deverá ser \textbf{Verdade}

\item Em outras palavras, a definição da relação de implicação lógica ($\Rightarrow$), segue a tabela-verdade da operação ou conectivo lógico da implicação ($\rightarrow$)
\end{itemize}
\end{block}
\end{frame}

%--------------------------------------------------------------------------------------------------


\begin{frame}[t]{Implicação Lógica} % Título do Frame
	Propriedades:
	\begin{itemize}
	\item Propriedade Reflexiva: $P_{pqr\ldots} \Rightarrow P_{pqr\ldots}$

	\item Propriedade Transitiva: se $P_{pqr\ldots} \Rightarrow Q_{pqr\ldots}$ e $Q_{pqr\ldots} \Rightarrow R_{pqr\ldots}$ então $P_{pqr\ldots} \Rightarrow R_{pqr\ldots}$

	\item Exemplo 01: para $p \wedge q, p \vee q ~e~ p \leftrightarrow q$:
	\begin{itemize}
	\item $p \wedge q \Rightarrow p \vee q$
	\item $p \wedge q \Rightarrow p \leftrightarrow q$
	\end{itemize}

	\item Exemplo 02: para $p \leftrightarrow q, p \rightarrow q ~e~ q \rightarrow p$:
	\begin{itemize}
	\item $p \leftrightarrow q \Rightarrow p \rightarrow q$
	\item $p \leftrightarrow q \Rightarrow q \rightarrow p$
	\end{itemize}
	
	\item Falta validar todos estes exemplos!
	\end{itemize}
\end{frame}





\begin{frame}[t]{Implicação Lógica} % Título do Frame

\frametitle{Regras de Inferência}

Construindo TVs, construa e valide as Implicações Lógicas que se seguem:

	\begin{description}
	\item[Adição:] 
		$\begin{array}{l}
		   p \Rightarrow p \vee q \\ 
		   q \Rightarrow p \vee q \\
		\end{array}$
	\end{description}
\end{frame}

\begin{frame}[t]{Implicação Lógica} % Título do Frame
	Regras de Inferência:
	\begin{description}
	\item[Adição:] 
		$\begin{array}{l}
		   p \Rightarrow p \vee q \\ 
		   q \Rightarrow p \vee q \\
		\end{array}$

	\item[Simplificação:] 
		$\begin{array}{l}
		  p \wedge q \Rightarrow p \\
		  p \wedge q \Rightarrow q \\
		\end{array}$
	\end{description}
\end{frame}

\begin{frame}[t]{Implicação Lógica} % Título do Frame
	Regras de Inferência:
	\begin{description}
	\item[Adição:] 
		$\begin{array}{l}
		   p \Rightarrow p \vee q \\ 
		   q \Rightarrow p \vee q \\
		\end{array}$

	\item[Simplificação:] 
		$\begin{array}{l}
		  p \wedge q \Rightarrow p \\
		  p \wedge q \Rightarrow q \\
		\end{array}$

	\item[Silogismo Disjuntivo:] 
		$\begin{array}{l}
		  (p \vee q) \wedge \sim p \Rightarrow q \\
		  (p \vee q) \wedge \sim q \Rightarrow p \\
		\end{array}$

	\end{description}
\end{frame}

\begin{frame}[t]{Implicação Lógica} % Título do Frame
	Regras de Inferência:
	\begin{description}
	\item[Adição:] 
		$\begin{array}{l}
		   p \Rightarrow p \vee q \\ 
		   q \Rightarrow p \vee q \\
		\end{array}$

	\item[Simplificação:] 
		$\begin{array}{l}
		  p \wedge q \Rightarrow p \\
		  p \wedge q \Rightarrow q \\
		\end{array}$

	\item[Silogismo Disjuntivo:] 
		$\begin{array}{l}
		  (p \vee q) \wedge \sim p \Rightarrow q \\
		  (p \vee q) \wedge \sim q \Rightarrow p \\
		\end{array}$

	\item[Modus Ponens:] $ (p \rightarrow q) \wedge p \Rightarrow q$

	\end{description}
\end{frame}

\begin{frame}[t]{Implicação Lógica} % Título do Frame
	Regras de Inferência:
	\begin{description}
	\item[Adição:] 
		$\begin{array}{l}
		   p \Rightarrow p \vee q \\ 
		   q \Rightarrow p \vee q \\
		\end{array}$

	\item[Simplificação:] 
		$\begin{array}{l}
		  p \wedge q \Rightarrow p \\
		  p \wedge q \Rightarrow q \\
		\end{array}$

	\item[Silogismo Disjuntivo:] 
		$\begin{array}{l}
		  (p \vee q) \wedge \sim p \Rightarrow q \\
		  (p \vee q) \wedge \sim q \Rightarrow p \\
		\end{array}$

	\item[Modus Ponens:] $ (p \rightarrow q) \wedge p \Rightarrow q$

	\item[Modus Tollens:] $ (p \rightarrow q) \wedge \sim q \Rightarrow \sim p$

	\end{description}
\end{frame}

\begin{frame}[t]{Implicação Lógica} % Título do Frame
	Regras de Inferência:
	\begin{description}
	\item[Adição:] 
		$\begin{array}{l}
		   p \Rightarrow p \vee q \\ 
		   q \Rightarrow p \vee q \\
		\end{array}$

	\item[Simplificação:] 
		$\begin{array}{l}
		  p \wedge q \Rightarrow p \\
		  p \wedge q \Rightarrow q \\
		\end{array}$

	\item[Silogismo Disjuntivo:] 
		$\begin{array}{l}
		  (p \vee q) \wedge \sim p \Rightarrow q \\
		  (p \vee q) \wedge \sim q \Rightarrow p \\
		\end{array}$

	\item[Modus Ponens:] $ (p \rightarrow q) \wedge p \Rightarrow q$

	\item[Modus Tollens:] $ (p \rightarrow q) \wedge \sim q \Rightarrow \sim p$

	\item[Silogismo Hipotético:] $ (p \rightarrow q) \wedge  (q \rightarrow r) \Rightarrow  (p \rightarrow r)$

	\end{description}
\end{frame}

\begin{frame}[t]{Implicação Lógica} % Título do Frame
	Regras de Inferência:
	\begin{description}
	\item[Adição:] 
		$\begin{array}{l}
		   p \Rightarrow p \wedge q \\ 
		   p \Rightarrow p \vee q \\
		\end{array}$

	\item[Simplificação:] 
		$\begin{array}{l}
		  p \wedge q \Rightarrow p \\
		  p \wedge q \Rightarrow q \\
		\end{array}$

	\item[Silogismo Disjuntivo:] 
		$\begin{array}{l}
		  (p \vee q) \wedge \sim p \Rightarrow q \\
		  (p \vee q) \wedge \sim q \Rightarrow p \\
		\end{array}$

	\item[Modus Ponens:] $ (p \rightarrow q) \wedge p \Rightarrow q$

	\item[Modus Tollens:] $ (p \rightarrow q) \wedge \sim q \Rightarrow \sim p$

	\item[Silogismo Hipotético:] $ (p \rightarrow q) \wedge  (q \rightarrow r) \Rightarrow  (p \rightarrow r)$

	\item[Princípio da Inconsistência:] $ \square \Rightarrow  p$
		\begin{itemize} 
		\item $p \wedge \sim p \Rightarrow \square $ 
		\end{itemize}

	\end{description}
\end{frame}

\begin{frame}[t]{Implicação Lógica - Exercicios} % Título do Frame
	\begin{enumerate}
	\item Mostre que:
	   \begin{enumerate}
	   \item $q \Rightarrow p \rightarrow q$
	   \item $q \Rightarrow p \wedge q \leftrightarrow p$
	   \end{enumerate}

	\item Mostre que $p \leftrightarrow\sim q$ não implica $p \rightarrow q$

	\item Mostre que $(X \neq 0 \rightarrow X=Y) \wedge X \neq Y \Rightarrow X=0$
	\end{enumerate}
\end{frame}
