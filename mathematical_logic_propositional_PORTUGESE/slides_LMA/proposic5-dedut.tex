% ============================================================================

\section{Método Dedutivo}

\subsection{Álgebra das Proposições}

\begin{frame}[t]
\vskip 3.5cm
\begin{center}
{\Huge Método Dedutivo}
\end{center}
\end{frame}

\begin{frame}[t]{Álgebra das Proposições}
	Propriedades da Conjunção

	\vskip 0.7cm

	\begin{description}
	\item [Idempotente] $p \wedge p \Leftrightarrow p$
	\item [Comutativa] $p \wedge q \Leftrightarrow q \wedge p$
	\item [Associativa] $(p \wedge q) \wedge r \Leftrightarrow p \wedge (q \wedge r)$
	\item [Identidade] $\begin{array}{l}p \wedge \mbox{true} \Leftrightarrow p\\p \wedge \mbox{false} \Leftrightarrow \mbox{false}\end{array}$
	\end{description}
\end{frame}

\begin{frame}[t]{Álgebra das Proposições}
	Propriedades da Disjunção

	\vskip 0.7cm

	\begin{description}
	\item [Idempotente] $p \vee p \Leftrightarrow p$
	\item [Comutativa] $p \vee q \Leftrightarrow q \vee p$
	\item [Associativa] $(p \vee q) \vee r \Leftrightarrow p \vee (q \vee r)$
	\item [Identidade] $\begin{array}{l}p \vee \mbox{true} \Leftrightarrow \mbox{true}\\p \vee \mbox{false} \Leftrightarrow p\end{array}$
	\end{description}
\end{frame}

\begin{frame}[t]{Álgebra das Proposições}
	Propriedades da Conjunção e da Disjunção

	\vskip 0.7cm

	\begin{description}
	\item [Distributiva] $\begin{array}{l}p \wedge (q \vee r) \Leftrightarrow (p \wedge q)\vee(p \wedge r)\\p \vee (q \wedge r) \Leftrightarrow (p \vee q) \wedge (p \vee r)\end{array}$

	\item [Absorção] $\begin{array}{l}p \wedge (p \vee q) \Leftrightarrow p\\p \vee (p \wedge q) \Leftrightarrow p\end{array}$

	\item [Regra de DE MORGAN] $\begin{array}{l}\sim(p \wedge q) \Leftrightarrow \sim p \vee \sim q\\ \sim (p \vee q) \Leftrightarrow \sim p \wedge \sim q\end{array}$
	\end{description}
\end{frame}

\begin{frame}[t]{Álgebra das Proposições - Exercício}
	Apresente a negação das proposições abaixo:

	\begin{itemize}
	\item É inteligente e estuda
	\item É médico ou professor
	\end{itemize}
\end{frame}

\begin{frame}[t]{Álgebra das Proposições - Exercício}
	Apresente a negação das proposições abaixo:

	\begin{itemize}
	\item É inteligente e estuda
	\item É médico ou professor
	\end{itemize}

	\vskip 0.5cm

	Conclusões:

	\begin{itemize}
	\item $p \wedge q \Leftrightarrow \sim(\sim p \vee \sim q)$
	\item $p \vee q \Leftrightarrow \sim(\sim p \wedge \sim q)$
	\end{itemize}
\end{frame}

\begin{frame}[t]{Álgebra das Proposições}
	Negação da Condicional:

	\begin{itemize}
	\item Se $p \rightarrow q \Leftrightarrow \sim p \vee q$ então sua negação é dada por $$\begin{array}{c}\sim(p \rightarrow q) \Leftrightarrow \sim(\sim p \vee q)\\ \sim(\sim p \vee q) \Leftrightarrow p \wedge \sim q \end{array}$$
	\end{itemize}

	\vskip 2cm

	{\bf CUIDADO!:} Condicional não apresenta as propriedades idempotente, comutativa e associativa.
\end{frame}

\begin{frame}[t]{Álgebra das Proposições}
	Negação da Bicondicional:

	\begin{itemize}
	\item Se $p \leftrightarrow q \Leftrightarrow (p \rightarrow q)\wedge (q \rightarrow p)$ então sua negação é dada por $$\begin{array}{c}\sim(p \leftrightarrow q) \Leftrightarrow \sim((\sim p \vee q)\wedge(\sim q \vee p))\\  \sim((\sim p \vee q)\wedge(\sim q \vee p)) \Leftrightarrow (p \wedge \sim q) \vee (q \wedge\sim p) \end{array}$$
	\end{itemize}

	\vskip 2cm

	{\bf CUIDADO!:} Bicondicional não é idempotente, mas é comutativa e associativa.
\end{frame}

\begin{frame}[t]{Álgebra das Proposições - Exercício}
	Prove:

	\begin{enumerate}
	\item $p \rightarrow q \vee r \Leftrightarrow (p \rightarrow q) \vee (p \rightarrow r)$
	\item $p \rightarrow q \wedge r \Leftrightarrow (p \rightarrow q) \wedge (p \rightarrow r)$
	\end{enumerate}
\end{frame}

\begin{frame}[t]{Método Dedutivo}
	\begin{itemize}
	\item Seja $\mathtt{falso} \rightarrow p$ podemos concluir que
	\item \textcolor{red}{Atenção:} no livro texto, a fórmula $\mathtt{falso}$ é equivalente
	a representação de \textit{quadrado vazio} ou \textcolor{blue}{fórmula sempre falsa}: $\Box $
	
	\item Idem para $\mathtt{true} \equiv \blacksquare $ (\textit{quadrado cheio} ou \textcolor{blue}{fórmula sempre verdade})
	
	\item $\Phi ^n(\blacksquare) = V$ \hspace{1cm}  (interpretação V nas n-linhas da TV)
	
	\item $\Phi ^n(\Box) = F$ \hspace{1cm}  (interpretação F nas n-linhas da TV)
	\end{itemize}
\end{frame}

\begin{frame}[t]{Método Dedutivo}
	\begin{itemize}
	\item Seja $\mathtt{falso} \rightarrow p$ podemos concluir que $$\begin{array}{c}\sim(\mathtt{falso}) \vee p \\ \mathtt{verdade} \vee p \\ \mathtt{verdade}\end{array}$$
	\end{itemize}
\end{frame}

\begin{frame}[t]{Método Dedutivo}
	\begin{itemize}
	\item Seja $\mathtt{falso} \rightarrow p$ podemos concluir que $$\begin{array}{c}\sim(\mathtt{falso}) \vee p \\ \mathtt{verdade} \vee p \\ \mathtt{verdade}\end{array}$$

	\item Seja $p \rightarrow \mathtt{verdade}$ então:
	\end{itemize}
\end{frame}

\begin{frame}[t]{Método Dedutivo}
	\begin{itemize}
	\item Seja $\mathtt{falso} \rightarrow p$ podemos concluir que $$\begin{array}{c}\sim(\mathtt{falso}) \vee p \\ \mathtt{verdade} \vee p \\ \mathtt{verdade}\end{array}$$

	\item Seja $p \rightarrow \mathtt{verdade}$ então: $$\begin{array}{c} \sim p \vee \mathtt{verdade} \\ \mathtt{verdade}\end{array}$$
	\end{itemize}
\end{frame}

%%%%%%%%%% inserido pelo Claudio

\begin{frame}[t]{Método Dedutivo -- Reflexões Parciais}

	\begin{itemize}
	\item Lembrar que: $\mathtt{falso} \equiv  \square$ 
    \item $\mathtt{verdade} \equiv \blacksquare  $

	\item Logo, para que  $P \Rightarrow Q $ seja uma relação
de implicação (ou condicional)
 verdadeira (lembrar da definição desta relação via TV), podemos
assumir como uma relação de equivalência (ou bi-condicional)
também verdadeira

   \item Para isto, basta demonstrar que fórmula $P \rightarrow Q $ é uma fórmula
tautológica, isto é: $P \rightarrow Q \equiv \blacksquare  $

 \item Precisamente: $P \rightarrow Q \Leftrightarrow \blacksquare  $

\item Há uma relação  de equivalência verdadeira!

	\end{itemize}

\end{frame}







\begin{frame}[t]{Método Dedutivo}
	Demonstre pelo método dedutivo que a implicação de simplificação $$p \wedge q \Rightarrow p$$
\end{frame}

\begin{frame}[t]{Método Dedutivo}
	Demonstre pelo método dedutivo que a implicação de simplificação $$p \wedge q \Rightarrow p$$
	\vskip 0.2cm
	$$\begin{array}{cl}
	p \wedge q \rightarrow p & \mbox{(hipótese inicial)}
	\end{array}$$
\end{frame}

\begin{frame}[t]{Método Dedutivo}
	Demonstre pelo método dedutivo que a implicação de simplificação $$p \wedge q \Rightarrow p$$
	\vskip 0.2cm
	$$\begin{array}{cl}
	p \wedge q \rightarrow p & \mbox{(hipótese inicial)} \\
	\sim(p \wedge q) \vee p & \mbox{(equivalência condicional)} 
	\end{array}$$
\end{frame}

\begin{frame}[t]{Método Dedutivo}
	Demonstre pelo método dedutivo que a implicação de simplificação $$p \wedge q \Rightarrow p$$
	\vskip 0.2cm
	$$\begin{array}{cl}
	p \wedge q \rightarrow p & \mbox{(hipótese inicial)} \\
	\sim(p \wedge q) \vee p & \mbox{(equivalência condicional)} \\
	(\sim p \vee\sim q) \vee p & \mbox{(regra de DE MORGAN)} 
	\end{array}$$
\end{frame}

\begin{frame}[t]{Método Dedutivo}
	Demonstre pelo método dedutivo que a implicação de simplificação $$p \wedge q \Rightarrow p$$
	\vskip 0.2cm
	$$\begin{array}{cl}
	p \wedge q \rightarrow p & \mbox{(hipótese inicial)} \\
	\sim(p \wedge q) \vee p & \mbox{(equivalência condicional)} \\
	(\sim p \vee\sim q) \vee p & \mbox{(regra de DE MORGAN)} \\
	\sim p \vee (\sim q \vee p) & \mbox{(associativa)} 
	\end{array}$$
\end{frame}

\begin{frame}[t]{Método Dedutivo}
	Demonstre pelo método dedutivo que a implicação de simplificação $$p \wedge q \Rightarrow p$$
	\vskip 0.2cm
	$$\begin{array}{cl}
	p \wedge q \rightarrow p & \mbox{(hipótese inicial)} \\
	\sim(p \wedge q) \vee p & \mbox{(equivalência condicional)} \\
	(\sim p \vee\sim q) \vee p & \mbox{(regra de DE MORGAN)} \\
	\sim p \vee (\sim q \vee p) & \mbox{(associativa)} \\
	\sim p \vee (p \vee\sim q) & \mbox{(comutativa)} 
	\end{array}$$
\end{frame}

\begin{frame}[t]{Método Dedutivo}
	Demonstre pelo método dedutivo que a implicação de simplificação $$p \wedge q \Rightarrow p$$
	\vskip 0.2cm
	$$\begin{array}{cl}
	p \wedge q \rightarrow p & \mbox{(hipótese inicial)} \\
	\sim(p \wedge q) \vee p & \mbox{(equivalência condicional)} \\
	(\sim p \vee\sim q) \vee p & \mbox{(regra de DE MORGAN)} \\
	\sim p \vee (\sim q \vee p) & \mbox{(associativa)} \\
	\sim p \vee (p \vee\sim q) & \mbox{(comutativa)} \\
	(\sim p \vee p) \vee\sim q & \mbox{(associativa)} 
	\end{array}$$
\end{frame}

\begin{frame}[t]{Método Dedutivo}
	Demonstre pelo método dedutivo que a implicação de simplificação $$p \wedge q \Rightarrow p$$
	\vskip 0.2cm
	$$\begin{array}{cl}
	p \wedge q \rightarrow p & \mbox{(hipótese inicial)} \\
	\sim(p \wedge q) \vee p & \mbox{(equivalência condicional)} \\
	(\sim p \vee\sim q) \vee p & \mbox{(regra de DE MORGAN)} \\
	\sim p \vee (\sim q \vee p) & \mbox{(associativa)} \\
	\sim p \vee (p \vee\sim q) & \mbox{(comutativa)} \\
	(\sim p \vee p) \vee\sim q & \mbox{(associativa)} \\
	(\mathtt{verdade}) \vee\sim q & \mbox{(tautologia)}
	\end{array}$$
\end{frame}

\begin{frame}[t]{Método Dedutivo}
	Demonstre pelo método dedutivo que a implicação de simplificação $$p \wedge q \Rightarrow p$$
	\vskip 0.2cm
	$$\begin{array}{cl}
	p \wedge q \rightarrow p & \mbox{(hipótese inicial)} \\
	\sim(p \wedge q) \vee p & \mbox{(equivalência condicional)} \\
	(\sim p \vee\sim q) \vee p & \mbox{(regra de DE MORGAN)} \\
	\sim p \vee (\sim q \vee p) & \mbox{(associativa)} \\
	\sim p \vee (p \vee\sim q) & \mbox{(comutativa)} \\
	(\sim p \vee p) \vee\sim q & \mbox{(associativa)} \\
	(\mathtt{verdade}) \vee\sim q & \mbox{(tautologia)} \\
	\mathtt{verdade} & \mbox{(identidade)}
	\end{array}$$
	
	Lembrar que: $\mathtt{verdade} \equiv  \blacksquare $
	
\end{frame}

\begin{frame}[t]{Método Dedutivo}
	Demonstre pelo método dedutivo que a implicação de adição $$p \Rightarrow p \vee q$$
	\vskip 0.2cm
	$$\begin{array}{cl}
	p \rightarrow p \vee q & \mbox{(hipótese inicial)}
	\end{array}$$
\end{frame}

\begin{frame}[t]{Método Dedutivo}
	Demonstre pelo método dedutivo que a implicação de adição $$p \Rightarrow p \vee q$$
	\vskip 0.2cm
	$$\begin{array}{cl}
	p \rightarrow p \vee q & \mbox{(hipótese inicial)} \\
	\sim p \vee (p \vee q) & \mbox{(equivalência condicional)} \\
	(\sim p \vee p) \vee q & \mbox{(associativa)} \\
	(\mathtt{verdade}) \vee q & \mbox{(tautologia)} \\
	\mathtt{verdade} & \mbox{(identidade)}
	\end{array}$$
\end{frame}

\begin{frame}[t]{Método Dedutivo}
	Demonstre pelo método dedutivo a regra {\em Modus Ponens}: $$(p \rightarrow q) \wedge p \Rightarrow q$$
\end{frame}

\begin{frame}[t]{Método Dedutivo}
	Demonstre pelo método dedutivo a regra {\em Modus Ponens}: $$(p \rightarrow q) \wedge p \Rightarrow q$$
	\vskip 0.2cm
	$$\begin{array}{cl}
	(p \rightarrow q) \wedge p \rightarrow q & \mbox{(hipótese inicial)} \\
	(\sim p \vee q) \wedge p \rightarrow q & \mbox{(equivalência condicional)} \\
	p \wedge (\sim p \vee q) \rightarrow q & \mbox{(comutativa)} \\
	(p \wedge \sim p) \vee (p \wedge q) \rightarrow q & \mbox{(distributiva)} \\
	(\square) \vee (p \wedge q) \rightarrow q & \mbox{(contradição)} \\
	(p \wedge q) \rightarrow q & \mbox{(identidade)} \\
	\blacksquare & 
	\end{array}$$
\end{frame}

\begin{frame}[t]{Método Dedutivo}
	Demonstre pelo método dedutivo a regra {\em Modus Tollens}: $$(p \rightarrow q) \wedge \sim q \Rightarrow \sim p$$
\end{frame}

\begin{frame}[t]{Método Dedutivo}
	Demonstre pelo método dedutivo a regra {\em Modus Tollens}: $$(p \rightarrow q) \wedge \sim q \Rightarrow \sim p$$
	\vskip 0.2cm
	$$\begin{array}{cl}
	(p \rightarrow q) \wedge \sim q \rightarrow \sim p & \mbox{(hipótese inicial)} \\
	(\sim p \vee q) \wedge \sim q \rightarrow \sim p & \mbox{(equivalência condicional)} \\
	(\sim p \wedge \sim q) \vee (q \wedge\sim q) \rightarrow\sim p & \mbox{(distributiva)} \\
	(\sim p \wedge\sim q) \vee \square \rightarrow\sim p & \mbox{(contradição)} \\
	(\sim p \wedge\sim q) \rightarrow\sim p & \mbox{(identidade)} \\
	\blacksquare & 
	\end{array}$$
\end{frame}

\begin{frame}[t]{Método Dedutivo}
	Demonstre pelo método dedutivo a regra {\em Silogismo Disjuntivo}: $$(p \vee q) \wedge \sim p \Rightarrow q$$
\end{frame}

\begin{frame}[t]{Método Dedutivo}
	Demonstre pelo método dedutivo a regra {\em Silogismo Disjuntivo}: $$(p \vee q) \wedge \sim p \Rightarrow q$$
	\vskip 0.2cm
	$$\begin{array}{cl}
	(p \vee q) \wedge \sim p \rightarrow q & \mbox{(hipótese inicial)} \\
	(p \wedge \sim p) \vee (q \wedge\sim p) \rightarrow q & \mbox{(distributiva)} \\
	\square \vee (q \wedge\sim p) \rightarrow q & \mbox{(contradição)} \\
	(q \wedge\sim p) \rightarrow q & \mbox{(identidade)} \\
	\blacksquare & 
	\end{array}$$
\end{frame}


\begin{frame}[t]{Método Dedutivo}

  Resumo até o momento:
  \begin{itemize}
  	\item Se $F_{pqrs...} \Leftrightarrow G_{pqrs...} $ significa que :
  	
  	\begin{enumerate}
  	
  	\item Que estas duas fórmulas apresentam uma \textcolor{red}{relação de equivalência verdadeira} entre si,
  	      a TV de $F$ é igual a TV de $G$, logo:
  	
  	\item $F_{pqrs...} \leftrightarrow G_{pqrs...} \Leftrightarrow \blacksquare  $\\
  	Lembre que:  $ p \leftrightarrow q \Leftrightarrow (p \rightarrow q ) \wedge  (q \rightarrow p) $
  	  	  	      
  	\item E ainda, podemos transformar   $F_{pqrs...}$ em $G_{pqrs...}$  usando outras regras
  	de equivalência, e vice-versa $G_{pqrs...}$ em $F_{pqrs...}$, leia-se:
  	
  	\item $F_{pqrs...} \underrightarrow{\ast}  G_{pqrs...}$ ou
  	
  	\item ou $G_{pqrs...} \underrightarrow{\ast}  F_{pqrs...}$ 
  	
  	\item Além os uso das propriedades: simétrica, reflexividade e transitividade
  	
  	
  	\end{enumerate}
  	\end{itemize}

  	
\end{frame}


\begin{frame}[t]{Método Dedutivo}

 E ainda:
  \begin{itemize}
  	
\item Quanto $F_{pqrs...}  \Rightarrow G_{pqrs...} $ significa:
  	 
  	\begin{enumerate}
  	\item Significa que $F_{pqrs...}$ tem uma relação de implicação verdadeira para fórmula $G_{pqrs...}$ 
  	
  	\item Para mostrar que isto é verdadeiro, demonstra-se que a implicação é uma
  	      tautologia, isto é: \\
  	      $F_{pqrs...} \rightarrow G_{pqrs...}$ terá que levar a $\blacksquare $
  	      
  	\item Ou   $F_{pqrs...} \rightarrow G_{pqrs...} \Leftrightarrow \blacksquare $  usando outras regras	de equivalências (verdades entre duas fórmulas)
  	
  	\item Aqui:  $(p \rightarrow q )  \Leftrightarrow  (\sim p \vee q )$
  	
  	
  	\item Além os uso das propriedades:  reflexividade e transitividade
  	
  	
  	\end{enumerate}

	\end{itemize}

  	
\end{frame}



\begin{frame}[t]{Método Dedutivo}

  Onde vamos usar esta formulação no curso?

 \begin{enumerate}

\item Faz parte o conceito fundamental do que é um \underline{teorema lógico}

\item Antecipando (mas não muito), um conjunto de fórmulas vai confirmar ou
não uma conclusão ($C$), isto é:
$$
\{ F_1, F_2, F_3, .... F_n  \}  \vdash C
$$
  
   \item Essencialmente, isto é perguntar se: 
$$\{ F_1, F_2, F_3, .... F_n  \} \Rightarrow C$$

   \item Ou se: 
$$\{ F_1, F_2, F_3, .... F_n  \} \rightarrow C \Leftrightarrow \blacksquare $$

  \item Das notações acima, troque as vírgulas 
  por  conectivos $\wedge$ e reflita sobre esta 
  fórmula canônica.

	\end{enumerate}

\end{frame}



\begin{frame}[t]{Método Dedutivo}

\begin{itemize}

\item Mas para chegar lá, faltam alguns detalhes de como transformar fórmulas entre si
\item Uma fórmula transformada em outra ... mantém-se equivalente!

\item Então vamos reescrever fórmulas:

\end{itemize}


\end{frame}

\begin{frame}[t]{Método Dedutivo}
  Transformação de conectivos equivalentes, levando a redução do números de conectivos:

	\begin{enumerate}
	\item Transformar $\wedge, \rightarrow, \leftrightarrow$ em termos de $\sim, \vee$:
	\begin{itemize}
	\item $p \wedge q \Leftrightarrow \sim(\sim p \vee\sim q)$
	\item $p \rightarrow q \Leftrightarrow \sim p \vee q$
	\item $p \leftrightarrow q \Leftrightarrow \sim (\sim (\sim p \vee q) \vee\sim (\sim q \vee p))$
	\end{itemize}

		\item Transformar  $\vee, \rightarrow, \leftrightarrow$ em termos de $\sim, \wedge$:
	\begin{itemize}
	\item $p \vee q \Leftrightarrow \sim (\sim p \wedge\sim q)$
	\item $p \rightarrow q \Leftrightarrow \sim (p \wedge\sim q)$
	\item $p \leftrightarrow q \Leftrightarrow \sim (p \wedge\sim q) \wedge\sim (\sim p \wedge q)$
	\end{itemize}

		\item Transformar  $\wedge, \vee, \leftrightarrow$ em termos de $\sim, \rightarrow$:
	\begin{itemize}
	\item $p \vee q \Leftrightarrow \sim p \rightarrow q$
	\item $p \wedge q \Leftrightarrow \sim (p \rightarrow\sim q)$
	\item $p \leftrightarrow q \Leftrightarrow \sim ((p \rightarrow q) \rightarrow\sim (q \rightarrow p))$
	\end{itemize}
	\end{enumerate}
\end{frame}




\begin{frame}[t]{Forma Normal das Proposições}
	São proposições que (no máximo) contém os conectivos $\sim, \wedge, \vee$.
	
	Tipos:

	\begin{description}
	\item [FNC] Formal Normal Conjuntiva
	\item [FND] Formal Normal Disjuntiva
	\end{description}
\end{frame}

\begin{frame}[t]{Forma Normal Conjuntiva - FNC}
	\begin{itemize}
	\item contém (no máximo) os conectivos $\sim, \wedge, \vee$
	\item $\sim$ não aparece repetido ($\sim\sim$)
	\item $\sim$ não tem alcance sobre $\wedge, \vee$, ou seja, só afeta proposições simples
	\item $\vee$ não tem alcance sobre $\wedge$ como em $(p \vee (q \wedge r))$
	\end{itemize}

	Exemplos de FNC:

	\begin{itemize}
	\item $p$
	\item $\sim p \wedge\sim q$
	\item $\sim p \wedge q \wedge r$
	\item $\sim q \vee r$
	\item $(\sim p \vee q) \wedge (\sim q \vee\sim r)$
	\end{itemize}
\end{frame}

\begin{frame}[t]{Forma Normal Conjuntiva - FNC}
	Como determinar a FNC equivalente?

	\begin{enumerate}
	\item Substituir os conectivos $\leftrightarrow$
	\item Substituir os conectivos $\rightarrow$
	\item Substituir dupla negações ($\sim\sim$)
	\item Substituir negações de parênteses $\sim(X \wedge Y)$ e $\sim(X \vee Y)$
	\item Aplicar a regra da distributiva onde $\vee$ tem alcance sobre $\wedge$
	\item Simplificar as expressões equivalentes a $\square$ ou $\blacksquare$
	\end{enumerate}
\end{frame}

\begin{frame}[t]{Forma Normal Conjuntiva - FNC}
	Converter a expressão abaixo para sua forma normal conjuntiva $$\sim (((p \vee q) \wedge\sim q) \vee (q \wedge r))$$
\end{frame}

\begin{frame}[t]{Forma Normal Conjuntiva - FNC}
	Converter a expressão abaixo para sua forma normal conjuntiva $$\sim (((p \vee q) \wedge\sim q) \vee (q \wedge r))$$

	\vskip 0.5cm

	\begin{enumerate}
	\item $\sim ((p \vee q) \wedge\sim q) \wedge\sim (q \wedge r)$
	\item $( \sim (p \vee q) \vee \sim\sim q) \wedge (\sim q \vee\sim r)$
	\item $((\sim p \wedge\sim q) \vee q) \wedge (\sim q \vee\sim r)$
	\item $(\sim p \vee q) \wedge (\sim q \vee q) \wedge (\sim q \vee\sim r)$
	\item $(\sim p \vee q) \wedge (\blacksquare) \wedge (\sim q \vee\sim r)$
	\item $(\sim p \vee q) \wedge (\sim q \vee\sim r)$
	\end{enumerate}
\end{frame}

\begin{frame}[t]{Forma Normal Conjuntiva - FNC}
	Converter a expressão abaixo para sua forma normal conjuntiva $$(p \rightarrow q) \leftrightarrow (\sim q \rightarrow\sim p)$$

	\vskip 0.25cm

	\begin{enumerate}
	\item $(\sim p \vee q) \leftrightarrow (\sim\sim q \vee\sim p)$
	\item $(\sim p \vee q) \leftrightarrow (q \vee\sim p)$
	\item $(\sim(\sim p \vee q) \vee (q \vee\sim p)) \wedge (\sim (q \vee\sim p) \vee (\sim p \vee q))$
	\item $((\sim\sim p \wedge\sim q) \vee (q \vee\sim p)) \wedge ((\sim q \wedge\sim\sim p) \vee (\sim p \vee q))$
	\item $((p \wedge\sim q) \vee (q \vee\sim p)) \wedge ((\sim q \wedge p) \vee (\sim p \vee q))$
	\item $(p\vee (q \vee\sim p)) \wedge (\sim q \vee (q \vee\sim p)) \wedge (\sim q \vee (\sim p \vee q)) \wedge (p \vee (\sim p \vee q))$
	\item $(p\vee (\sim p \vee q)) \wedge ((\sim q \vee q) \vee\sim p) \wedge (\sim q \vee (q \vee \sim p)) \wedge ((p \vee\sim p) \vee q)$
	\item $((p\vee \sim p) \vee q) \wedge ((\sim q \vee q) \vee\sim p) \wedge ((\sim q \vee q) \vee \sim p) \wedge ((p \vee\sim p) \vee q)$
	\item $((\blacksquare) \vee q) \wedge ((\blacksquare) \vee\sim p) \wedge ((\blacksquare) \vee \sim p) \wedge ((\blacksquare) \vee q)$
	\item $(\blacksquare) \wedge (\blacksquare) \wedge (\blacksquare) \wedge (\blacksquare)$
	\item $\blacksquare$
	\end{enumerate}
\end{frame}

\begin{frame}[t]{Forma Normal Conjuntiva - FNC}
	Converter a expressão abaixo para sua forma normal conjuntiva $$p \leftrightarrow q \vee\sim r$$

	\vskip 0.25cm

	\begin{enumerate}
	\item $(\sim p \vee q\vee\sim r) \wedge (\sim(q \vee\sim r) \vee p)$
	\item $(\sim p \vee q\vee\sim r) \wedge ((\sim q \wedge\sim\sim r) \vee p)$
	\item $(\sim p \vee q\vee\sim r) \wedge ((\sim q \wedge r) \vee p)$
	\item $(\sim p \vee q\vee\sim r) \wedge (\sim q \vee p) \wedge (r \vee p)$
	\end{enumerate}
\end{frame}

\begin{frame}[t]{Forma Normal Disjuntiva - FND}
	\begin{itemize}
	\item contém (no máximo) os conectivos $\sim, \wedge, \vee$
	\item $\sim$ não aparece repetido ($\sim\sim$)
	\item $\sim$ não tem alcance sobre $\wedge, \vee$, ou seja, só afeta proposições simples
	\item $\wedge$ não tem alcance sobre $\vee$ como em $(p \wedge (q \vee r))$
	\end{itemize}

	Exemplos de FND:

	\begin{itemize}
	\item $p$
	\item $\sim p \vee\sim q$
	\item $\sim p \vee q \vee r$
	\item $\sim q \wedge r$
	\item $(\sim p \wedge q) \vee (\sim q \wedge\sim r)$
	\end{itemize}
\end{frame}

\begin{frame}[t]{Forma Normal Disjuntiva - FND}
	Converter a expressão abaixo para sua forma normal disjuntiva $$(p \rightarrow q) \wedge (q \rightarrow p)$$

	\vskip 0.25cm

	\begin{enumerate}
	\item $(\sim p \vee q) \wedge (\sim q \vee p)$
	\item $((\sim p \vee q) \wedge \sim q) \vee ((\sim p \vee q) \wedge p)$
	\item $(\sim p \wedge \sim q) \vee (q \wedge \sim q) \vee (\sim p \wedge p) \vee (q \wedge p)$
	\item $(\sim p \wedge \sim q) \vee (\square) \vee (\square) \vee (q \wedge p)$
	\item $(\sim p \wedge \sim q) \vee (q \wedge p)$
	\end{enumerate}
\end{frame}

\begin{frame}[t]{Forma Normal Disjuntiva - FND}
	Converter a expressão abaixo para sua forma normal disjuntiva $$\sim(((p \vee q) \wedge \sim q) \vee (q \wedge r))$$

	\vskip 0.25cm

	\begin{enumerate}
	\item $\sim((p \vee q) \wedge \sim q) \wedge \sim(q \wedge r)$
	\item $(\sim(p \vee q) \vee \sim\sim q) \wedge (\sim q \vee \sim r)$
	\item $((\sim p \wedge \sim q) \vee q) \wedge (\sim q \vee \sim r)$
	\item $(((\sim p \wedge \sim q) \vee q) \wedge \sim q) \vee (((\sim p \wedge \sim q) \vee q) \wedge \sim r)$
	\item $(\sim p \wedge \sim q \wedge \sim q) \vee (q \wedge \sim q) \vee (\sim p \wedge \sim q \wedge \sim r) \vee (q \wedge \sim r)$
	\item $(\sim p \wedge (\sim q \wedge \sim q)) \vee (\square) \vee (\sim p \wedge \sim q \wedge \sim r) \vee (q \wedge \sim r)$
	\item $(\sim p \wedge \sim q) \vee (\sim p \wedge \sim q \wedge \sim r) \vee (q \wedge \sim r)$	\end{enumerate}
\end{frame}

\begin{frame}[t]{Método Dedutivo}
	\begin{itemize}
	\item Se uma proposição só contém os conectivos $\sim, \wedge, \vee$ então se trocarmos cada símbolo $\wedge$ por $\vee$ e vice-versa, dá-se o nome de {\em\bf dual} à proposição resultante.
	
	\item {\em Exemplo:} $\sim ((p \wedge q) \vee\sim r)$ e sua dual $\sim ((p \vee q) \wedge\sim r)$ 
	
	\item {\bf Princípio da Dualidade:} se $\mathcal{F}_1$ e $\mathcal{F}_2$ são fórmulas equivalentes que só contenham conectivos $\sim$, $\wedge$, e $\vee$, então suas duais $\mathcal{G}_1$ e $\mathcal{G}_2$ respectivamente, também são  equivalentes. Isto é:
	
	\begin{itemize}
	  \item se \hspace{1.5cm} $\mathcal{F}_1 \equiv \mathcal{F}_2$
	  \item então \hspace{1cm} $\mathcal{G}_1 \equiv \mathcal{G}_2$
	\end{itemize}
	
	
	\item Comprove o {\bf Princípio da Dualidade} com exemplos, construindo TV ou transformando
	fórmulas
	
	\end{itemize}
\end{frame}

%%%%%%%%%%%%%%%%%%%%%%%%%%%%%%%%%%%%%%%%%%%%%%%%%%%%
\begin{frame}[t]{Fórmulas Duais -- Fixando}

Basicamente, por definição, obter uma fórmula dual de $\mathcal{F}$ , efetuamos:
\begin{itemize}
  \item Trocamos $\vee $ por $\wedge $,  e vice-versa;
  \item A negação tem como  dual ela própria ($\sim p$ tem como dual $\sim p$);
  \item E se algum termo da fórmula $\mathcal{F}$ for $\blacksquare $ troque por $\Box $,  e vice-versa.
\end{itemize}

\end{frame}
%%%%%%%%%%%%%%%%%%%%%%%%%%%%%%%%%%%%%%%%%%%%%%%%%%%%

\begin{frame}[fragile]{Exemplos de Fixação}

\begin{description}
\itemsep 13pt

\item[Ex1:] $ X \vee (X \wedge Y) \equiv \mathcal{F}_1 $\\
sua dual é:\\
 $ X \wedge (X \vee Y) \equiv \mathcal{F}_2 $

\item[Ex2:] $ X \vee  \square  \equiv X $\\
sua dual é:\\
 $ X \wedge \blacksquare \equiv X $

 \item[Ex3:] $ X \vee \blacksquare  \equiv \blacksquare $ \\
 sua dual é:\\
 $ X \wedge \square \equiv \square $
\end{description}

\end{frame}


%{\bf \textcolor{red}{PS: Rever estes exemplos -- algo %estranho do \LaTeX}}
%%%%%%%%%%%%%%%%%%%%%%%%%%%%%%%%%%%%%%%%%%%%%%%%%%%%
\begin{frame}[t]{Método Dedutivo - Exercícios}
	\begin{enumerate}
	\item Simplifique as proposições
	\begin{enumerate}
	\item $\sim(\sim p \rightarrow \sim q)$
	\item $\sim(p \vee q) \vee (\sim p \wedge q)$
	\end{enumerate}
	
	\item Use o método dedutivo para demonstrar
	\begin{enumerate}
	\item $(p \rightarrow q) \wedge (p \rightarrow r) \Leftrightarrow (p \rightarrow q \wedge r)$
	\end{enumerate}
	
	\item Determine a FNC
	\begin{enumerate}
	\item $p \rightarrow q$
	\item $p \leftrightarrow\sim p$
	\item $\sim p \downarrow (q \veebar p)$
	\end{enumerate}
	
	\item Determine a FND
	\begin{enumerate}
	\item $\sim(\sim p \vee\sim q)$
	\item $(p \rightarrow q) \vee \sim p$
	\end{enumerate}
	\end{enumerate}

\end{frame}

